\documentclass[12pt, a4paper]{article}

% !TeX program = lualatex

\usepackage{pacchetti}

%%%%%%%%%%%%%%%%%%%%%%%%%%%%%%%%%%%%%%% Colori 
\definecolor{darkturquoise}{rgb}{0.0, 0.81, 0.82}
\definecolor{cerisepink}{rgb}{0.93, 0.23, 0.51}
\definecolor{brilliantlavender}{rgb}{0.96, 0.73, 1.0}
\definecolor{fuchsiapink}{rgb}{1.0, 0.47, 1.0}

%%%%%%%%%%%%%%%%%%%%%%%%%%%%%%%%%%%%%%% HEAD COMMANDS	
%\newtheorem{theorem}{Theorem}[section]
%
%\newtheorem{corollary}{Corollary}[theorem]
%
%\newtheorem{lemma}[theorem]{Lemma}
%
%\theoremstyle{definition}
%\newtheorem{definition}{Definition}[section]
%
%\theoremstyle{remark}
%\newtheorem*{remark}{Remark}

%%%%%%%%%%%%%%%%%%%%%%%%%%%%%%%%%%%%%%% MATH SYMBOLS
\newcommand{\R}{{\mathbb{R}}}
\DeclareMathOperator{\Tr}{Tr}

%%%%%%%%%%%%%%%%%%%%%%%%%%%%%%%%%%%%%%%%box carini
\newenvironment<>{ideablock}[1]{%
	\setbeamercolor{block title}{fg=white,bg=idea}%
	\begin{block}#2{#1}}{\end{block}}

\newenvironment<>{defblock}[1]{%
	\setbeamercolor{block title}{fg=white,bg=definition}%
	\begin{block}#2{#1}}{\end{block}}

\newenvironment<>{theoblock}[1]{%
	\setbeamercolor{block title}{fg=white,bg=theorem}%
	\begin{block}#2{#1}}{\end{block}}


%%%%%%%%%%%%%%%%%%%%%%%%%%%%%%%%%%%%%%%%% due nuovi comandi per allineare le cose bene
\newcommand\parallelcontent[2]{
	\begin{columns}[t]
		\column{0.5\textwidth} #1
		\column{0.5\textwidth} #2
	\end{columns}
}
\newcommand\parallelitem[2]{
	\parallelcontent
	{\begin{itemize} \item #1 \end{itemize}}
	{\begin{itemize} \item #2 \end{itemize}}
}

%per fare le parentesi sulla destra di una lista

%%%%%%%%%%%%%%%%%%%%%%%%%%%%%%%%%%%%%%%% Graffe verticali per il testo
\tikzset{My Node Style/.style={midway, right, xshift=3.0ex, align=left, font=\small, draw=none, thin, text=black}}

\newcommand{\VerticalBrace}[4][]{%
	% #1 = draw options
	% #2 = top mark
	% #2 = bottom mark
	% #4 = label
	\begin{tikzpicture}[overlay,remember picture]
	\tikzmath{coordinate \p,\q;\p=(pic cs:#2);\q=(pic cs:#3);\maxx=max(\px,\qx);}
	\draw[xshift=1ex,decorate,decoration={brace, amplitude=1.5ex}, #1] 
	([yshift=1.5ex]{{pic cs:#2} -| \maxx pt,0})  -- ([yshift=-.5ex]{{pic cs:#3} -| \maxx pt,0})
	node[My Node Style] {#4};
	\end{tikzpicture}
}


% Bibliografia
\usepackage[backend=bibtex,doi=false,isbn=false,url=false]{biblatex}
\addbibresource{bibliografia.bib}

%todo: settare un nuovo font, ora non ci riesco

\title{PhD Project Proposal}

\author{Veronica Sacchi\thanks{veronica.sacchi@sns.it}}


\begin{document}

\maketitle

\section{Introduction}
This document aims to illustrate the intrests I developed in Theoretical Physics during these past few years of study, and how I would intend to pursue those intrests in the following years.

As a prosecutions of my studies I would like to undertake a resarch degree in Theoretical Physics, with a specific focus on some applications and developments of General Relativity.
The Gravitational force has been the first fundamental force that has been given a coherent modern treatment - thanks to the well known work due to Newton - but today it's the one that we struggle the most with when we try to extend its regime of validity.
The fact that it deals with so many fundamental concepts -  it is even able to deform time itself! - but, despite the incredibly elegant treatment of General Relativity, with its powerful geometrical insight, most of the questions raisen are not answered yet, fascinates me a lot.

Among others, black holes are one of the most astonishing predictions of General Relativity and, as any great discovery in Physics, they have the double role of being a milestone of the analysis developed so far and one of the most difficult challenges this very same analysis has to face.

As their existence has now been firmly established (see for example the very famous \cite{falke}) questions concerning their structure and characteristics are not anymore purely academic but find new applications, especially in light of the latest developments of more powerful interferometers, whose measures hopefully in the near futures will be able to prove or reject many long standing hypthesis regarding these mysterious objects.

The renewed energy, directly injected into this field by gravitational waves coming from so far away, is then one of the driving force of my enthusiasm in approaching this topic, and in posing the questions I would like to research on and which are going to be presented in the next sections.

\section{Singularity Theorems}

The first work on Singularity theorem is due to Hawking \cite{hawking} and Penrose \cite{penrose}: is is astonishing how powerfully a few, simple assumtpions can lead to the proof of the existence of a singularity.
Of particular notice is the absence of the need of a specifc symmetry of spacetime to develop such theorems; they, in fact, aimed to answer the question whether a singularity would form even after a gravitational collapse in a non perfectly symmetric background
\clearpage


\printbibliography

\end{document}          
