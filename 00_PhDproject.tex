\documentclass[12pt, a4paper]{article}

% !TeX program = lualatex

\usepackage{pacchetti}



\definecolor{turquoise}{RGB}{0, 247, 230}
\definecolor{goldenyellow}{RGB}{255, 218, 66}
\definecolor{fuchsia}{RGB}{255, 0, 172}

%%%%%%%%%%%%%%%%%%%%%%%%%%%%%%%%%%%%%%% HEAD COMMANDS	
\newtheorem{theorem}{Theorem}[section]

\newtheorem{corollary}[theorem]{Corollary}

\newtheorem{lemma}[theorem]{Lemma}

\newtheorem{prop}[theorem]{Proposition}

\theoremstyle{definition}
\newtheorem{definition}{Definition}[section]

\theoremstyle{remark}
\newtheorem*{remark}{Remark}

\newcommand{\EAK}[1]{\textcolor{red}{EAK: #1}}
\newcommand{\VS}[1]{\textcolor{cyan}{VS: #1}}

%%%%%%%%%%%%%%%%%%%%%%%%%%%%%%%%%%%%%%% MATH SYMBOLS
\newcommand{\R}{{\mathbb{R}}}
\newcommand{\N}{{\mathbb{N}}}

 \newcommand{\pprec}{\prec\mathrel{\mkern-5mu}\prec}






% Bibliografia
\usepackage[backend=bibtex,doi=false,isbn=false,url=false]{biblatex}
\addbibresource{bibliografia.bib}

\title{PhD Project Proposal}

\author{Veronica Sacchi\thanks{veronica.sacchi@sns.it}}

\begin{document}

\maketitle

\section{Introduction}
This document aims to illustrate the interests I developed in Theoretical Physics during these past few years of study, and how I would intend to pursue those interests in the following years.

As a prosecution of my studies I would like to undertake a research degree in Theoretical Physics, with a specific focus on some applications and developments of General Relativity.
The Gravitational force has been the first fundamental force that has been given a coherent modern treatment - thanks to the well-known work due to Newton - but today it's the one that we struggle the most with when we try to extend the regime of validity of our theory.
It deals with so many fundamental concepts -  it is even able to deform time itself! - but, despite the incredibly elegant treatment of General Relativity, with its powerful geometrical insight, most of the questions raised are not answered yet: this fascinates me a lot.

Among others, black holes are one of the most astonishing predictions of General Relativity and, as any great discovery in Physics, they have the double role of being a milestone of the analysis developed so far and one of the most difficult challenges this very same analysis has to face.

As their existence has now been firmly established (see for example the very famous \cite{falcke1999viewing}) questions concerning their structure and characteristics are not anymore purely academic but find new applications, especially in light of the latest developments of more powerful interferometers, whose measurements hopefully will be able to prove or reject many long-standing hypothesis regarding these mysterious objects in the near future.

The renewed energy, directly injected into this field by gravitational waves coming from so far away, is then one of the driving force of my enthusiasm in approaching this topic, and in posing the questions I would like to research on and which are going to be presented in the next sections.
\clearpage

\section{Singularity Theorems}

The first work on Singularity theorem is due to Hawking \cite{hawking1966occurrence} and Penrose \cite{penrose1965gravitational}: it is astonishing how powerfully a few, simple assumptions can lead to the proof of existence of a singularity.
It is of particular notice the absence of need for a specific symmetry of spacetime to develop such theorems; they, in fact, aimed to realize whether a singularity would form even after a gravitational collapse in a non perfectly symmetric background. 

Although very minimal, the hypotheses required for the classical singularity theorems are violated by classical fields - even by rather innocent looking ones, for a review see \cite{kontou2020energy} - and therefore it's natural to raise the question whether singularities should form even in the presence of quantum fields.

In order to answer these questions a few fundamental steps must be made. These include:
\begin{itemize}
	\item Identifying some reasonable energy conditions.
	\item Check what fields satisfy the above mentioned conditions.
	\item Identify what are the minimal requirements we need in order to deduce the creation of a singularity. 
	The energy condition is usually given as a bound on a contraction of the stress tensor of the theory under consideration: by means of Einstein equations this can be translated into a condition on the Ricci tensor, which in turn leads to the proof of a singularity formation using some geometrical procedures (the only physical input is in the energy and initial conditions).
\end{itemize}

Over my PhD I would really like to keep on studying this problem, focusing especially on the third point. 

It has been noticed how often local violations of a pointwise energy condition are compensated for, or even overcompensated for, in other regions of the spacetime: this hinted the idea of \emph{averaged} energy conditions, where the contraction of the stress tensor is averaged over a suitable region of spacetime.

This new class of conditions opens a new wide range of possibilities, and recently a very promising one has been identified, called AANEC, but its geometrical implications are not well understood yet. Hopefully the work by Fewster and Kontou in \cite{fewster2020new} can drive the analysis of the implications of this class of conditions, and might be a source of inspiration to identify an even better one (for example one from which a Singularity Theorem might be properly derived).

Moreover, very close to the proof of Penrose's Singularity theorem is the black hole area theorem, which states that the area of the black hole horizion can never decrease; at a classical level they are both valid, though semiclassically we know that the latter must be violated (due to Hawking Radiation) while we still expect the former to hold. Where the tipping point is, and how these two proofs should separate from each other is not clear yet, and would be an extremely exciting topic onto which I'd like to work.

It is worth pointing out that this is not at all a purely academic question; despite the need of formalization and precision that often require rather abstract and mathematical definitions, the question whether a singularity would form inside a black hole or at the origin of time has been a long standing problem, and already back in the 1960s Hawking himself realized how the then  recently discovered CMB was supporting the dignity of this question and the validity of his theorem \cite{hawking1968cosmic}.

\clearpage

\section{Energy Conditions}

The need for formulating a satisfying energy condition has been already illustrated in the previous section.

Although this is certainly the most urgent question to answer, little effort seems to have been put in understanding what physical meaning they carry. This is rather surprising, especially considering that one of the first, and most aggressive critic that has been moved against them is exactly their obscurity in terms of physical interpretation.

Lately it has been claimed they embody the ``attractiveness''  of gravity, or, in a more general sense, the stability of the system; even if this seems rather established now - at least for the classical energy conditions such as NEC (Null Energy Condition) and SEC (Strong Energy Condition) - some work still needs to be done in regard to averaged energy conditions, such as AANEC. 
This condition has been proved to hold at least in some special cases (see \cite{wall2010proving} and \cite{verch2000averaged}) and has been given an interpretation in \cite{curiel2017primer} as positivity of the energy flow through a geodesic, but to my understanding this has not been related yet to more microscopic properties of the theory.

Specifically, it would be interesting to understand the link between AANEC and an eventual singularity theorem derived from an averaged energy condition, and how that relates to the fact that the associated black hole (a singularity needs to be hidden beyond an horizon for Cosmic Censorship) is intrinsically an unstable object (due to Hawking Radiation).

More generally, as pointed out in the introduction of \cite{kontou2020energy}, the derivation of AANEC from the generalized second law seems paradoxical, given that the indipendence of averaged energy conditions from time-orientation suggests that they are related to microscopic physics, unlike thermodynamic entropy.

\section{Cosmic Censorship Conjecture}

In order to translate the singularity theorems, which only really prove the incompleteness of a timelike (Hawking) or null (Penrose) geodesic, an additional \emph{physical} hypothesis is needed \cite{witten2020light}: this is called Cosmic Censorship.

An established, or widely accepted version of the Cosmic Censorship Conjecture does not exist yet. Different formulations have been proposed up to now, the most important ones being the Strong Cosmic Censorship and the Weak Cosmic Censorship.

The \emph{Strong} Cosmic Censorship says that in an arbitrary space time, not necessarily asymptotic to a Minkowski space, no observer can see a naked singularity.

Instead the \emph{Weak} Cosmic Censorship requires that in a globally hyperbolic spacetime, asymptotic at spatial infinity to Minkowski space, in any localized process, the region to the far distance and to the far future continues to exist, just as in Minkowski space. Moreover, the evolution seen by an outside observer is supposed to be predictable based on the Classical Einstein Equations: any singularity must be hidden behind an horizon and must not affect the outside evolution.
Notice that the weak formulation is not a special case of the Strong Cosmic Censorship, due to this requirement of ``future continuation'' of evolution.

Strong Cosmic Censorship has been proven wrong by the analysis of the Reissner-Nordstr\"om black hole solution, and Reall \emph{et al.} have shown as, to recover it, one needs to allow for rough initial data \cite{dias2018strong}. They also investigate whether the coupling with other fields might rescue this condition, for example considering a charged scalar field \cite{Dias:2018ufh}.
This result is probably of pure academical interest, but it expresses the concern for the very outstanding question about whether Weak Cosmic Censorship (or a better formulation of it) is true or not. 
A ``hope'' in this direction has been given by the recent LIGO/Virgo observation, as if anything more than a bigger black hole was emerging from the collapse of two black holes, they would have perhaps seen it. By this I refer to well-posedness of initial value problems and the causality structure of the universe, which are topics recently investigated in \cite{Kovacs:2020pns} and \cite{Reall:2021voz}, among others.

Both problems concerning a good formulation of this conjecture and its dynamical implications are topics I am very attracted by, and even if my knowledge is probably very limited this is a horizon I would really like to explore.

	\section{A final remark}
Although the topics I have tried to resume before are something I think I'd like to  explore, I wouldn't like them to become my limits form this very moment. 

I am very well aware that, as I am approaching right now the opportunity to do some research, my interests are highly influenced by the problems I had a chance to encounter in my previous academic life, which is definetley rather limited if compared to the huge and rich universe Maths and Physics offer to us.

I would like to remark then that, more important than the exact topics my PhD will focus on, what I'd really like to gain is an \emph{attitude}: I want to be exposed to many different topics and refine that spirit of curiosity that has guided me through all these years up to here, to become able to orient myself in this marvelous universe and pursue my own interests while still influencing other people's work, in a collaborative environment where everyone is enriched by the presence of the others.

\clearpage


\printbibliography

\end{document}          
