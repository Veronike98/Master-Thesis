\documentclass[12pt, a4paper]{article}

% !TeX program = lualatex

\usepackage{pacchetti}



\definecolor{turquoise}{RGB}{0, 247, 230}
\definecolor{goldenyellow}{RGB}{255, 218, 66}
\definecolor{fuchsia}{RGB}{255, 0, 172}

%%%%%%%%%%%%%%%%%%%%%%%%%%%%%%%%%%%%%%% HEAD COMMANDS	
\newtheorem{theorem}{Theorem}[section]

\newtheorem{corollary}[theorem]{Corollary}

\newtheorem{lemma}[theorem]{Lemma}

\newtheorem{prop}[theorem]{Proposition}

\theoremstyle{definition}
\newtheorem{definition}{Definition}[section]

\theoremstyle{remark}
\newtheorem*{remark}{Remark}

\newcommand{\EAK}[1]{\textcolor{red}{EAK: #1}}
\newcommand{\VS}[1]{\textcolor{cyan}{VS: #1}}

%%%%%%%%%%%%%%%%%%%%%%%%%%%%%%%%%%%%%%% MATH SYMBOLS
\newcommand{\R}{{\mathbb{R}}}
\newcommand{\N}{{\mathbb{N}}}

 \newcommand{\pprec}{\prec\mathrel{\mkern-5mu}\prec}






% Bibliografia
\usepackage[backend=bibtex,doi=false,isbn=false,url=false]{biblatex}
\addbibresource{bibliografia.bib}

%todo: settare un nuovo font, ora non ci riesco

\title{Master Thesis -- Abstract}

\author{Veronica Sacchi\thanks{veronica.sacchi@sns.it}\\
Supervisor: Eleni A. Kontou\thanks{e.a.kontou@uva.nl}}


\begin{document}

\maketitle

\section{Introduction}
The aim of this project is to review and present some new developments about Singularity Theorems.
Singularity theorems have been first proved in the \(1960\)s by Hawking \cite{hawking1966occurrence} and Penrose \cite{penrose1965gravitational} and represent a major step forward in the study of gravitational collapse.
They, in fact, are able to prove  the existence of incomplete geodesics, within the spacetime manifold, making use of only very general assumptions.
Any singularity theorems in fact mainly needs \(3\) ingredients:
\begin{itemize}
	\item A causality condition: this is needed to make sure there are no closed causal curves
	\item A boundary or initial condition: we need a surface which is \emph{trapped}, which sort of means that all the future directed geodesics leaving this surface are initially focusing.
	\item An energy conditions: this should embody the ``attractiveness'' of gravity.
\end{itemize}

It is particularly imprtant to notice that no assumption on symmetries of the spacetime is required; this allows to answer the question singularity theorems were born for, namely if the singularity formation in the Swarzschild and Friedmann metrics were a consequence of the very high degree of symmetry of the solution, or were a hint to a more general behaviour.

\section{Classical Singularity Theorems}
I refer to classical singularity theorems to indicate Penrose's and Hawking's (or slightly modified versions of them), characterized by the requirement of a pointwise energy condition.
They do require this specific class of energy conditions because of the structure of the proof; here we provide a sketch of it to single out why that's necessary.

We start from a submanifold (a Cauchy hypersurface in the case of Hawking's theorem) whose orthogonal, future directed geodesics are initially focusing (the existence of such a submanifold is one of the hypotheses of the theorem).

Given a congruence of geodesics it is possible to compute the expansion, namely how the transversial area of the congruence does infinitesimally evolve. This is usually called \(\theta\) and is defined as 
\[
\theta \coloneqq \nabla_{\mu} U^{\mu} = \frac{1}{2} \text{Tr}\left[g^{-1}\dot{g}\right]
\]
where we indicate with \(\nabla_{\mu}\) the covariant derivative, with \(U^{\mu}\) the tangent field of the congurence, and \(g\) is the metric tensor.
It is not difficult then to derive the Riccati inequality, which puts some constraints on how this quantity evloves along the geodesics:
\[
\frac{D}{d\tau} \theta \le -\frac{1}{3} \theta ^2 - R_{\mu\nu}U^{\mu}U^{\nu}
\]
Starting from a negative initial value for \(\theta\), using an energy condition that guarantees us 
\[
R_{\mu\nu}U^{\mu}U^{\nu} >=0
\]
\emph{everywhere}, with some analysis gymnastic, it is possible to prove that \(theta\) needs to blow up within a finite amount of proper time.
We can then see why we required the energy conditions to hold everywhere.

\section{Extended singularity theorems}
``Unluckly'' the energy conditions most widely used (NEC for Penrose's and SEC for Hawking's theorem) are violeted by a large collection of (even classical) scalar fields, as for example the cosmological constant.

It is therefore natural the question whether these theorems can be extended to be applicable to the spcetime we live in or not.

\clearpage

\printbibliography

\end{document}          
