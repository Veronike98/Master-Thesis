% !TeX spellcheck = <none>
\documentclass[12pt, a4paper]{article}

% !TeX program = lualatex

\usepackage{pacchetti}



\definecolor{turquoise}{RGB}{0, 247, 230}
\definecolor{goldenyellow}{RGB}{255, 218, 66}
\definecolor{fuchsia}{RGB}{255, 0, 172}

%%%%%%%%%%%%%%%%%%%%%%%%%%%%%%%%%%%%%%% HEAD COMMANDS	
\newtheorem{theorem}{Theorem}[section]

\newtheorem{corollary}[theorem]{Corollary}

\newtheorem{lemma}[theorem]{Lemma}

\newtheorem{prop}[theorem]{Proposition}

\theoremstyle{definition}
\newtheorem{definition}{Definition}[section]

\theoremstyle{remark}
\newtheorem*{remark}{Remark}

\newcommand{\EAK}[1]{\textcolor{red}{EAK: #1}}
\newcommand{\VS}[1]{\textcolor{cyan}{VS: #1}}

%%%%%%%%%%%%%%%%%%%%%%%%%%%%%%%%%%%%%%% MATH SYMBOLS
\newcommand{\R}{{\mathbb{R}}}
\newcommand{\N}{{\mathbb{N}}}

 \newcommand{\pprec}{\prec\mathrel{\mkern-5mu}\prec}






% Bibliografia
\usepackage[backend=bibtex,doi=false,isbn=false,url=false]{biblatex}
\addbibresource{bibliografia.bib}

%todo: settare un nuovo font, ora non ci riesco

\title{Master Thesis -- Abstract}

\author{Veronica Sacchi\thanks{veronica.sacchi@sns.it}\\
Supervisor: Eleni A. Kontou\thanks{e.a.kontou@uva.nl}}


\begin{document}

\maketitle

\section{Introduction}
The aim of this project is to review and present some new developments about Singularity Theorems.
Singularity theorems have been first proved in the \(1960\)s by Hawking \cite{hawking1966occurrence} and Penrose \cite{penrose1965gravitational} and represent a major step forward in the study of gravitational collapse.
They, in fact, are able to prove  the existence of incomplete geodesics, within the spacetime manifold, making use of only very general assumptions.
Any singularity theorem in fact mainly needs \(3\) ingredients:
\begin{itemize}
	\item A causality condition: this is needed to make sure there are no closed causal curves.
	\item A boundary or initial condition: we need a surface which is \emph{trapped}, which sort of means that all the future directed geodesics leaving this surface are initially focusing.
	\item An energy condition: this should embody the ``attractiveness'' of gravity.
\end{itemize}

It is particularly important to notice that no assumption on symmetries of the spacetime is required; this allows to answer the question singularity theorems were born for, namely if the singularity formation in the Schwarzschild and Friedmann metrics were a consequence of the very high degree of symmetry of the solution, or were a hint to a more general behaviour.

\section{Classical Singularity Theorems}
I refer to classical singularity theorems to indicate Penrose's and Hawking's (or slightly modified versions of them), characterized by the requirement of a pointwise energy condition.
They do require this specific class of energy conditions because of the structure of the proof; here we provide a sketch of it to single out why that's necessary.

We start from a submanifold (a Cauchy hypersurface in the case of Hawking's theorem) whose orthogonal, future directed geodesics are initially focusing (the existence of such a submanifold is one of the hypotheses of the theorem).

Given a congruence of geodesics it is possible to compute the expansion, namely how the transversial area of the congruence does infinitesimally evolve. This is usually called \(\theta\) and is defined as 
\[
\theta \coloneqq \nabla_{\mu} U^{\mu} = \frac{1}{2} \text{Tr}\left[g^{-1}\dot{g}\right]
\]
where we indicate with \(\nabla_{\mu}\) the covariant derivative, with \(U^{\mu}\) the tangent field of the congurence, and \(g\) is the metric tensor.
It is not difficult then to derive the Riccati inequality, which puts some constraints on how this quantity evolves along the geodesics:
\[
\frac{D\theta }{d\tau} \le -\frac{1}{3} \theta ^2 - R_{\mu\nu}U^{\mu}U^{\nu}
\]
Starting from a negative initial value for \(\theta\), an energy condition that guarantees 
\[
R_{\mu\nu}U^{\mu}U^{\nu} \ge 0
\]
\emph{everywhere}, with some analysis gymnastic, leads us to the fact that \(\theta\) needs to blow up within a finite amount of proper time.
We can then see why we required the energy condition to hold everywhere.

\section{Extended singularity theorems}
``Unluckily'' the energy conditions most widely used (Null Energy Condition for Penrose's and Strong Energy Condition for Hawking's theorem) are violeted by a large collection of (even classical) scalar fields. 
Just to quickly mention a couple of important examples, SEC is violated by the cosmological constant, while NEC can be violated by non-minimally coupled scalar fields.

Contracting the stress-energy tensor with a null-vector \(\textbf{k}\) we obtain:
\begin{align*}
	T_{ab}k^ak^b &= (k^a\nabla_a\phi)^2 + \xi G_{ab}k^ak^b\phi^2 -\xi(k^a\nabla_a)^2\phi^2 \\
	&= (1 - 2\xi)(k^a\nabla_a\phi)^2 + \xi G_{ab}k^ak^b\phi^2 - 2\xi\phi(k^a\nabla_a)^2\phi
\end{align*}

For any point \(x\) on a null geodesics with tangent field \(\textbf{k}\), we can choose \(\phi(x) = c_1\), \(k^a\nabla_a \phi(x) = 0\) and \((k^a\nabla_a)^2\phi = c_2\).

Since \(\xi \neq 0 \) for each \(c_1\) it is possible to find \(c_2\) so that the above expression becomes negative, which means that NEC is violated.
One may then show that NEC can be violated in all cases with \(m \neq 0 \) and explicit examples of violations, including naked singularities and wormholes, were found in \cite{barcelo1999traversable} and \cite{barcelo2000scalar}.

It is therefore natural ro raise the question whether these theorems can be extended to be applicable to the spacetime we live in or not.
Several different energy conditions have been derived throughout the years (for a review see \cite{kontou2020energy} while \cite{tipler1978energy} for an early article)
but all of them can be violeted (counterexamples can be built with Lagrangians with second order or higher derivatives interactions, but second order equations of motions).
In front of this evident weakness, it has been though noticed how violations of some energy conditions at some points or regions of spacetime are usually compensated for - or even overcompensated for - at other points, both for quantum and classical fields \cite{ford1999quantum}.
This is a major hint for \emph{averaged} energy conditions, where we require that a contraction of the stress tensor of a theory, averaged over a suitable spacetime region, is non negative.
This is definitely weakening the energy condition hypothesis, as local violations are allowed, but it's not clear whether they will be strong enough to prove a singularity theorem.

A first attempt in making use of these condition was made in \cite{fewster2011singularity},
where the energy condition is weakened to an exponential average of a contraction of the stress tensor along causal geodesics.
However, this result is archived with some difficulties and relying heavily on technical mathematical analysis tools, partially loosing the physical picture of the system.
An analogous result was obtained recently \cite{fewster2020new}, but employing much more powerful tools from differential geometry, that embed very naturally the averaged conditons into the proof, by constructing an integral equation rather than a differential one.

To make this method work though, quite some new tools and geometry machinery needs to be introduced: for this purpose, a specific section of the thesis will be dedicated to the introduction of Jacobi fields, Index forms and their properties.

Here is a sketch of how such a proof is working in the timelike case: the main idea is that a geodesic from a point \( A\) is lenght maximizing up to a focal point, where the lenght is defined as:
\[
L[\gamma] \coloneqq \int_{0}^{\tau} |\dot{\gamma}(t) |dt
\]
We proceed then to take a family of geodesics emanating from \(A\) and parametrized by a parameter \(s\): we can compute the first and second deriviative of \(L\) with respect to \(s\), the latter one being related to the index form of this transversial field (it would be the second fundamental form in the null case).
Depending on the sign of this object we can tell whether there is a focal point or not, and hence if \(\gamma\) fails to be lenght maximizing within a fine amount of proper time.

Conversely to what we had done before, we didn't solve any differential equation here, instead we naturally intoduced an integral inequality as a criteria for focal points' existence, which brings all the advantages we have been talking on before.

As the proof of singularity theorems follows way more naturally after introducing these objects, there is hope in the ability to extend them to much more promising energy conditions, potentially the AANEC (achronal averaged null energy condition) which has recently attracted lots of attention both for a possible physical interpretation (\cite{curiel2017primer}) and for its wide regime of validity (up to know, no existing quantum field is known to violate it).

\section{Black Hole Area Theorem}
Very tied to the proof of Penrose's Singularity Theorem is the proof of the very famous Black Hole area theorem.
It has been already noticed that the proof provided in \cite{fewster2011singularity}
for weaker energy conditions might have been extended to the black hole area theorem (see \cite{lesourd2018remark}) but no proof has ever been provided, to my current knowledge, of this theorem making use of the methods developed in \cite{fewster2020new}.
In this work, we aim to create such a proof, possibly extend it to weaker energy conditions and gain an understanding about whether it needs a stronger condition or not than singularity theorems.

This last question is raisen by the fact that we mildly expect Singularity Theorems to hold at least at a semiclassical level, while we know - due to Hawking radiation - that the Black Hole area theorem must be violated. Where the tipping point that separates this two proofs is, is unknown yet, and we hope that this investigation my shed some light on it.



\clearpage

\printbibliography

\end{document}          
