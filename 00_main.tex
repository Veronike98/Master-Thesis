\documentclass[12pt, a4paper]{book}

% !TeX program = lualatex

\usepackage{pacchetti}



\definecolor{turquoise}{RGB}{0, 247, 230}
\definecolor{goldenyellow}{RGB}{255, 218, 66}
\definecolor{fuchsia}{RGB}{255, 0, 172}

%%%%%%%%%%%%%%%%%%%%%%%%%%%%%%%%%%%%%%% HEAD COMMANDS	
\newtheorem{theorem}{Theorem}[section]

\newtheorem{corollary}[theorem]{Corollary}

\newtheorem{lemma}[theorem]{Lemma}

\newtheorem{prop}[theorem]{Proposition}

\theoremstyle{definition}
\newtheorem{definition}{Definition}[section]

\theoremstyle{remark}
\newtheorem*{remark}{Remark}

\newcommand{\EAK}[1]{\textcolor{red}{EAK: #1}}
\newcommand{\VS}[1]{\textcolor{cyan}{VS: #1}}

%%%%%%%%%%%%%%%%%%%%%%%%%%%%%%%%%%%%%%% MATH SYMBOLS
\newcommand{\R}{{\mathbb{R}}}
\newcommand{\N}{{\mathbb{N}}}

 \newcommand{\pprec}{\prec\mathrel{\mkern-5mu}\prec}






\title{Master Thesis}

\author{Veronica Sacchi\thanks{veronica.sacchi@sns.it}\\
Supervisor: Eleni A. Kontou\thanks{e.a.kontou@uva.nl}}

% Bibliografia
\usepackage[backend=bibtex,doi=false,isbn=false,url=false]{biblatex}
\addbibresource{bibliografia.bib}

\begin{document}

\maketitle

\tableofcontents
\clearpage

\chapter*{Notation and Conventions}
\thispagestyle{empty}
%nice frame
\usetikzlibrary{math,calc,decorations.pathmorphing}
\tikzset{
	fancy corner/.pic={
\textit{}		\draw (0,4) -- (0,2.5) to[bend left] (2.5,0) -- (4,0);
		\draw (0.5,4) -- (0.5,0.5) -- (4,0.5);
		\draw (1,4) -- (1,0) -- (0,0) -- (0,1) -- (4,1);
	},
	fancy edge/.pic={
		\draw (1,0) -- (0,1) -- (-1,0) -- (0,-1) -- cycle;
		\fill[red] (1,0) circle(.5pt);
		\def\a{.7}
		\foreach \xs in {1,-1} {
			\begin{scope}[xscale=\xs,shift={(.2,0)}]
				\draw (4,0) -- ({1+\a},0) -- (1,\a) -- ({1-\a},0) -- (1,{-\a}) -- ({1+\a},0);
				\foreach \ys in {1,-1} {
					\draw[yscale=\ys] (4,0.5) -- ({0.5*(3+\a)},0.5) -- +(135:0.6);
				}
			\end{scope}
		}
		\fill (0,0) circle (.2);
	},
	pics/fancy rectangle/.style n args={4}{code={
			\pic at (#1,#3) {fancy corner};
			\pic[yscale=-1] at (#1,#4) {fancy corner};
			\pic[xscale=-1] at (#2,#3) {fancy corner};
			\pic[xscale=-1,yscale=-1] at (#2,#4) {fancy corner};
			\pic at ({(#1+#2)/2},{#3+.5}) {fancy edge};
			\pic at ({(#1+#2)/2},{#4-.5}) {fancy edge};
			\foreach \i in {0,0.5,1} {
				\foreach \y in {#3,#4-1} {
					\draw ({#1+3.9},{\y+\i}) -- ({#1+#2)/2-3.9},{\y+\i});
					\draw ({#2-3.9},{\y+\i}) -- ({#1+#2)/2+3.9},{\y+\i});
				}
				\foreach \x in {#1,#2-1} {
					\draw ({\x+\i},{#3+3.9}) -- ({\x+\i},{#4-3.9});
			}}
	}},
}

\def\decorationscale{.5} % decoration scale
\colorlet{decoration color}{cyan!70} % decoration color

	\begin{tikzpicture}[draw=decoration color,fill=decoration color,line width=2pt,remember picture,overlay,x={(\decorationscale,0)},y={(0,\decorationscale)}]
	\tikzmath{
		coordinate \ne,\sw;
		\ne=(current page.north east)+(-1cm,-1cm);
		\sw=(current page.south west)+(1cm,1cm);
		\xx=(\swx/1cm)/\decorationscale;
		\xxx=\nex/1cm/\decorationscale;
		\yy=\swy/1cm/\decorationscale;
		\yyy=\ney/1cm/\decorationscale;
	}
	\pic{fancy rectangle={\xx}{\xxx}{\yy}{\yyy}};
	\end{tikzpicture}

In this work we are going to use the \([-, + , +]\) conventions as labelled in the Misner, Thorne and Wheeler \cite{misner1973gravitation}. That means:
\begin{itemize}
	\item[\ding{99}] The metric signature is \((+, -, -, -)\).
	\item[\ding{99}]  The Riemann tensor in components is defined as:
	\[
	R\indices{^{\mu}_{\nu\alpha\beta}} = \partial_{\alpha}\Gamma^{\mu}_{\nu\beta} - \partial_{\beta}\Gamma^{\mu}_{\nu\alpha} + \Gamma^{\mu}_{\alpha\tau}\Gamma^{\tau}_{\nu\beta} - \Gamma^{\mu}_{\beta\tau}\Gamma^{\tau}_{\nu\alpha}.
	\]
	In particular, it's true that:
	\[
	[\nabla_U, \nabla_V] W^{\mu} = R\indices{^{\mu}_{\nu\alpha\beta}}W^{\nu}U^{\alpha}V^{\beta}.
	\]
	\item[\ding{99}]  Einstein equations are \(R_{\mu\nu} - \frac{1}{2}Rg_{\mu\nu} = 8\pi T_{\mu\nu}\).
\end{itemize}

\chapter{Fundamental Concepts}
\section{Geodesics}
The main source of inspiration for this first chapter is the book by B. O'Neill \cite{o1983semi}. We shall refer mainly to chapters \(10\) and \(14\); when needed, more specific references will be made.

Let's start off with some basic concepts derived from differential geometry. We will generally address the spacetime manifold as \(M\), \(g\) its Lorentizan metric and \(\nabla\) the Levi-Civita connection.
We will generally assume that a \emph{spacetime} is a smooth, time-oriented Lorentzian manifold; we will use the terms manifold and spacetime intercheangeably, unless otherwise stated.
The first fundamental definiton we need is:
\begin{definition}
	Let \((M, g)\) be a spacetime and \(I \subseteq \R\) an interval; a smooth curve \(\gamma : I \rightarrow M\) is a \emph{geodesic} if its tangent field is parallel transported along \(\gamma\).\\
\end{definition}    

Calling \(U\indices{^{\mu}} \coloneqq \dot{\gamma}^{\mu}\) the tangent field, this definition is equivalent to the equation:
\[
\nabla_U U^{\mu} \coloneqq \frac{D\tensor{U}{^\mu} }{dt} \equiv 0.
\]


A \emph{curve} is a \(1\)-parameter map; similarly we can define a \emph{family of curves} as a \(2\)-parameter map \(\zeta: I \times J \rightarrow M\), where \(I, J \subseteq \R\) are intervals. This gives us \(2\) tangent fields, which we call respectively \emph{longitudinal} and \emph{transverse}:
\[
U^{\mu} \coloneqq \frac{\partial}{\partial t} \zeta^{\mu}(t,s); \quad \quad \quad 
V^{\mu} \coloneqq \frac{\partial}{\partial s} \zeta^{\mu}(t,s). 
\]

\begin{definition}
	Given a curve \(\gamma\), a \emph{(smooth) variation of \(\gamma\)} is any (smooth) \(2\)-parameter map \(\zeta(s,t)\) such that 
	\[
	\left. \zeta(s, t) \right\vert_{s = 0} = \gamma(t).
	\]
\end{definition}

%todo: metti un bel disegno di una famigla di curve

\begin{remark}
	Commutation of partial derivatives immediately gives 
	\[
	[U, V] = 0 \implies \nabla_U V^{\mu} = \nabla_V U^{\mu}
	\]
	and, by definition of the Riemann tensor
	\[
	[\nabla_U, \nabla_V]W^{\mu} = R\indices{^{\mu}_{\nu\alpha\beta}}W^{\nu}U^{\alpha}V^{\beta}
	\]
\end{remark}

By this remark we can compute the second derivative of the transverse field:
\[
\frac{D^2\tensor{V}{^\mu} }{dt^2} = \nabla_U\nabla_V U^{\mu} = \nabla_V\nabla_U U^{\mu} + R\indices{^{\mu}_{\nu\alpha\beta}}U^{\nu}U^{\alpha}V^{\beta}
\]

which his leads us to the second fundamental objects we need to define. 
\begin{definition}
	Given a family of \emph{geodesics} \(\gamma_s(t) \coloneqq \zeta(s,t)\), the transversal field \(V^{\mu} \coloneqq \frac{\partial}{\partial s} \zeta^{\mu}(t,s)\) is said to be a \emph{Jacobi field}.
\end{definition}

This vector field has a very important characterization, which here we'll only state:
\begin{lemma}
\(V^{\mu}\) is a \emph{Jacobi field} if and only if it satisfies the equation
	\[
	\frac{D^2\tensor{V}{^\mu} }{dt^2} = \nabla_U\nabla_V U^{\mu} =  R\indices{^{\mu}_{\nu\alpha\beta}}U^{\nu}U^{\alpha}V^{\beta}
	\].
\end{lemma}

\section{Submanifolds}
\label{sec:submanifolds}

\begin{definition}
	Let \(M\) be a submanifold of the semi-Riemannian manifold \(\bar{M}\), and \(j:M\rightarrow\bar{M}\) the inclusion map. Then \(M\) is a \emph{semi-Riemannian submanifold} if the pullback of the metric tensor \(j^*(g)\) is a metric tensor on \(M\).
\end{definition}

This definiton makes it clear that observers on the submanifold \(M\) agree on the notion of distance as if they were outside the submanifold. Nevertheless, observers within \(M\) see the world differently than observers from the outside: they in fact will inherit \(2\) different connections from their respective notion of metric. 

The comparison between these \(2\) different Levi-Civita connections gives surge to the \emph{second fundamental form}, which provides an infinitesimal description of the shape of \(M\) within \(\bar{M}\).

In order to introduce this important object notice first of all that:
\[
\forall p \in M \quad T_p\bar{M} = \underbrace{T_pM}_{\text{vectors tangent to }M}+ \underbrace{T_p(M)^{\perp}}_{\text{vectors othogonal to } M}
\]

\noindent and we will generally call \(\Pi_p^{\parallel}\coloneqq d_pj\) and \(\Pi_p^{\perp}\coloneqq \mathbb{1} - d_pj\) the projecive maps on these \(2\) subspaces.
Adopting a similar notation, we can decompose the set of all smooth fields in \(T\bar{M}\) restricted to \(M\) as:
\[
\bar{\mathfrak{X}}(M) = \mathfrak{X}(M) + \mathfrak{X}(M)^{\perp}.
\]

Now, the connection \(\bar{D}\) on \(\bar{M}\) will naturally give raise to a connection on \(M\)
\begin{align*}
\bar{D} : \mathfrak{X}(M) \times \bar{\mathfrak{X}}(M) & \rightarrow \bar{\mathfrak{X}}(M) \\
	 V \times X &\mapsto \bar{D}_V X
\end{align*}

by taking any smooth extension of the fields \(V\) and \(X\) to \(\mathfrak{X}(\bar{M})\), and then restricting again on \(M\). It can be proved that \(\bar{D}_V X\) is a well-defined smooth vector field on \(M\). In particular
\begin{lemma} 
	if \(V, W \in \mathfrak{X}(M)\) and \(D\) is the Levi-Civita connection on \(M\), it holds that
	\[
	D_V W = \Pi^{\parallel}\left(\bar{D}_V W\right).
	\]
\end{lemma}

It's evident then that the Levi-Civita connection on \(M\) is loosing something repect to the induced connection from \(\bar{M}\). This is precisely the object we have been looking for:
\begin{definition}
	given \(M \subset \bar{M}\) a semi-Riemannian submanifold, the \emph{shape tensor} (or \emph{second fundamental form}) is defined as:
	\begin{align*}
		\mathbb{I} : \mathfrak{X}(M) \times \mathfrak{X}(M) &\longrightarrow \mathfrak{X}(M)^{\perp}\\
							V \times W &\mapsto \Pi^{\perp}\left(\bar{D}_V W\right)
	\end{align*}
	\noindent and in particular is bilinear and symmetric.
\end{definition}

In order to gain a better understanding of what this object encodes, let's think for a moment about the following example. Given a field \(Y \in \mathfrak{X}(M)\) tangent to a curve \(\alpha\) of \(M\), parametrized by \(s\), we indicate:
\[
\dot{Y} \coloneqq \frac{\bar{D}Y}{ds} \quad \quad Y' \coloneqq \frac{DY}{ds}
\]
It is then easy to prove that:
\[
\ddot{\alpha} = \alpha'' + \mathbb{I}(\alpha', \alpha')
\]

and hence, we can think of \(\mathbb{I}\) as the additional external ``force'' needed to keep a point moving on \(M \subset \bar{M}\), a sort of costraining force. An equivalent interpretation is by how much a vector paralleled transported in \(M\) according to \(D\) is actually changing from the point of view of \(\bar{D}\).

%todo: inserisci disegno pag.103 O'Neill


The second fundamental form induces a vector field on \(M\) called \emph{mean curvature}.
\begin{definition}
		Let\(M \subset \bar{M}\) be an  \(n\)-semi-Riemannian submanifold, with \emph{shape tensor} \(\mathbb{I}\), and \(\{\textbf{e}_i\}_{i \in \{1, \ldots, n\}}\) any frame on \(M\) at \(p\). Then we call \emph{mean curvature vector field} the field \(\mathfrak{H} \in \mathfrak{X}(M)^{\perp} \) defined by:
		\[
		\mathfrak{H}^{\mu}(p) = \frac{1}{n} \sum_{i=1}^{n} \epsilon_i \mathbb{I}^{\mu}(\textbf{e}_i, \textbf{e}_i).
		\]
		where 
		\[
		\epsilon_i = 
		\begin{cases}
		+1 \quad \text{se } \textbf{e}_i \text{\`e di tipo \emph{tempo}} \\
		-1 \quad \text{se } \textbf{e}_i \text{\`e di tipo \emph{spazio}}
		\end{cases}
		\]
\end{definition}

This definition will be particularly important for the proof of Null Singularity Theorems, as they are instrumental to define the concept of a \emph{trapped surface}, the initial condition needed to eventually develop a singularity (in the classical limit). 

\section{Trapped surfaces}

\begin{definition}
	We say that a spacelike submanifold \(M\) is \emph{future-converging} provided its mean curvature vetor field \(\mathfrak{H}\) is past-pointing timelike.
\end{definition}



It is only a matter of simple linear algebra in each normal space \(T_p(M)^{\perp}\) to show that
\begin{lemma} \label{lemma:charact-trapped}
	assuming \(M\) is a \emph{spacelike} submanifold, the follwoing statements are equivalent:
	\begin{enumerate}
		\item  \(k(v) =g(\mathfrak{H}, v) < 0 \) for all future pointing \emph{null} vectors \(v\) normal to M.
		\item  \(k(w) =g(\mathfrak{H}, w) < 0 \) for all future pointing \emph{causal} vectors \(v\) normal to M.
		\item \(\mathfrak{H}\) is past-pointing timelike.
	\end{enumerate}
\end{lemma}

\begin{proof}
	We will prove a chain of implications to gain the equivalence.
	
	\(\mathbf{2) \implies 1)]}\) this is trivial because any null vector is a causal vector.
	
	\(\mathbf{1) \implies 3)]}\) Let's write \(\mathfrak{H}^{\mu} = (\mathfrak{H}^0, \vec{\mathfrak{H}})\). It is possible to choose a spacelike \(3-\)vector \(\vec{v}\) such that \(\vert\vec{v}\cdot\vec{\mathfrak{H}}\vert = - \vert\vec{v}\vert\cdot\vert\vec{\mathfrak{H}}\vert\) and  then complete it to the future-pointing null vector \(v^{\mu} = (v^0, \vec{v})\) (where \(v^0 = \vert \vec{v}\vert > 0\)).
	Now:
	\[
	\mathfrak{H}^{\mu}v_{\mu} = \mathfrak{H}^0v^0 - \vec{\mathfrak{H}}\cdot\vec{v} < 0 \quad \implies 
	\quad v^0 \mathfrak{H}^0 < - \vert\vec{v}\vert\cdot\vert\vec{\mathfrak{H}}\vert < 0
	\]
	which, for the positivity of \(v^0\), implies that \(H^{\mu}\) is timelike and past-pointing.
	
	\(\mathbf{3) \implies 2)]}\) first of all observe that any causal vector can be written as the sum of \(2\) future-pointing vectors such that \(w^{\mu}= v^{\mu} + t^{\mu}\) where \(v\) is null, and \(t^{\mu} = (t^0, \vec{0})\) (with \(t^0 \ge 0\)). At this point:
	\[
	w^{\mu}\mathfrak{H}^{\mu} = v^{\mu}\mathfrak{H}^{\mu} + \underbrace{t^0 \mathfrak{H}^0}_{\le0}
	\]
	Moreover, by Cauchy-Schwartz we have:
	\[
	v_{\mu}\mathfrak{H}^{\mu} = v^0 \mathfrak{H}^0 - \vec{\mathfrak{H}}\cdot\vec{v} \le v^0\mathfrak{H}^0 +\vert \vec{\mathfrak{H}}\vert\cdot\underbrace{\vert\vec{v}\vert}_{v^0} = v^0\underbrace{(\mathfrak{H}^0 + \vert \vec{\mathfrak{H}}\vert)}_{<0} < 0.
	\]
\end{proof}

\subsection{Causality conditions}

The notion of \emph{causality} has to do with the fact that in a Lorentzian manifold, starting from a  point \(p\) it is not true that an observer or a light ray can travel to any other point \(q\), as they are allowed only to move along (future pointing) causal geodesics (i.e. whose tangent field is never spacelike).

We will denote causality relations with the following notation, following \cite{o1983semi}.
\begin{enumerate}
	\item  \(p \guillemotleft q\) means there is a future-pointing \emph{timelike} curve in \(M\) connecting \(p\) to \(q\).
	\item \(p \prec q\) means there is a future-pointing \emph{causal} curve from \(p\) to \(q\).
\end{enumerate}

Evidently \(p\guillemotleft q\) implies \(p\prec q\) but not vice versa. For a subset \(A \subseteq M\) we also define:
\begin{enumerate}
	\item the \emph{chronological future} of \(A\)
	\[
	I^+(A) =\{ q\in M :\quad\exists p \in A\text{ with } p\guillemotleft q\}.
	\]
	\item the \emph{causal future} of \(A\)
	\[
	I^+(A) =\{ q\in M : \quad\exists p \in A\text{ with } p\prec q\}.
	\]
\end{enumerate}

%\clearpage

\chapter{Focal Points}

Now that we know what a submanifold is, an important question we might like to answer is \emph{given a submanifold \(P\subset M\) and a point \(q \notin P\), what is the ``best'' path from \(P\) to \(q\)?} 

Here ``best'' means a path that maximizes (or minimizes) some relevant functional of the curves joining \(P\) to \(q\). The most important functional we would like to study is the curve-lenght \(L[\gamma]\): it carries a powerful physical meaning but it turns out to be well-defined and sufficiently differentiable only for families of timelike curves. 

%Let's not worry for a moment about it: for the study of critical points of the lenght functional the relevant objects are focal points.

\section{Variations of \(E\)}

In this work we would like to study null, or even causal curves: \(L\) is not smooth though so a better choice for what functional to study is the \emph{energy} or \emph{action}. Given a curve \(\gamma : [0, l] \rightarrow M\) affinely parametrized by \(\lambda\), \(E\) is defined as
\begin{equation*}
E[\gamma] \coloneqq \frac{1}{2}\int_{0}^{l} g(\gamma'(\lambda), \gamma'(\lambda))d\lambda;
\end{equation*}
this quantity can be proven to vary smoothly for any picewise smooth variation of \(\gamma\), regardless of its causal nature.

%Let \(L[\gamma]\) be the lenght functional of the curve, defined as:
%\begin{equation*}
%	L[\gamma] \coloneqq \int_{a}^{b} \vert\vert \dot{\gamma}(\tau) \vert\vert d\tau;
%\end{equation*}
%where \(\vert\vert \dot{\gamma}(\tau) \vert\vert = \sqrt{g_{\mu\nu}U^{\mu}U^{\nu}}\) and \(\tau\) is the proper time of the curve.
%
%Given a picewise smooth family of timelike curves \(\zeta(t,s) \equiv \gamma_s(t)\) (with constant speed \(\vert U \vert = 1\)) it is only a matter of simple algebra to compute the first and second variation of \(L\) along the family.
%\begin{lemma}
%	\label{lemma:first-L-var}
%	If \(L\) is the lenght functional of the family \(\zeta:[a,b] \times (-\delta, \delta) \rightarrow M\) of timelike picewise smooth curves, then (take \(U\) and \(V\) as defined in \ref{eq:tang-trans-fields})
%	\begin{align*}
%		\frac{dL}{ds}\Big\vert_{s = 0} &= \int_{a}^{b} g_{\mu\nu} U^{\mu}\frac{D V^{\nu}}{D t} dt  \\
%		&=- \int_{a}^{b}  g_{\mu\nu} \frac{D U^{\mu}}{D t} V^{\nu} dt - \sum_{i = 1}^{k} g_{\mu\nu} \Delta U^{\mu}V^{\nu}\Big\vert_{t_i} + g_{\mu\nu} U^{\mu}V^{\nu}\Big\vert^b_a.
%	\end{align*}
%where \(t_i\) with \(i\in \{1, \ldots, k\}\) are the breaks of \(\zeta\).
%\end{lemma}

We are now interested in studying the set of all picewise smooth curves joining \(P\) to \(q\), \(\Omega(P, q)\). By means of a rather simple computation we can work out the formula for the first and second variation of \(E\) on a family of curves \(\gamma_s(\lambda) \coloneqq
\zeta(\lambda, s)\) in \(\Omega(P, q)\) (which adds the restriction that \(\forall s \quad \zeta(0, s) \in M\)).

\begin{lemma}
	Let \(\zeta\) be a picewise smooth variation of the curve \(\gamma: [0, l] \rightarrow M\)  in \(\Omega(P, q)\), and call \(U\) and \(V\) the tangent and the transverse fields, as in section \ref{sec:geodesics}. Then
	\begin{align}
		E'[\gamma_s]\vert_{s = 0} \coloneqq \frac{dE[\gamma_s]}{ds}\Big\vert_{s = 0} &= 
		\int_{0}^{l} U_{\mu}\nabla_UV^{\mu}(\lambda) d\lambda = \\
		&= - \int_{0}^{l}  V_{\mu}\nabla_UU^{\mu}(\lambda) d\lambda - \sum_{i = 1}^{k} \Delta U^{\mu}V_{\mu}\Big\vert_{\lambda_i} + U^{\mu}V_{\mu}\Big\vert^l_0.
		\label{eq:first-E-var}
	\end{align}
	If \(\gamma\) is a geodesic we also have
	\begin{equation}
		\label{eq:second-E-var}
		E''[\gamma_s]\vert_{s = 0} \coloneqq \frac{d^2E[\gamma_s]}{ds^2}\Big\vert_{s = 0} = 
		\int_{0}^{l} \left[(\nabla_UV_{\mu})(\nabla_UV^{\mu}) + R_{\mu\nu\alpha\beta}U^{\mu}V^{\nu}V^{\alpha}U^{\beta}\right] d\lambda + (U_{\mu}\nabla_VV^{\mu})\Big\vert_0^l.
	\end{equation}
	
\end{lemma}

\begin{proof}
	For the first identity we have
	\begin{equation*}
		\frac{dE[\gamma_s]}{ds}\Big\vert_{s = 0}  = \int_{0}^{l} U_{\mu}\nabla_VU^{\mu}(\lambda) d\lambda = \int_{0}^{l} U_{\mu}\nabla_UV^{\mu}(\lambda) d\lambda
	\end{equation*}
Then, integrating by parts we get 
\begin{equation*}
	\frac{dE[\gamma_s]}{ds} = \int_{0}^{l} \nabla_U\left(U_{\mu}(\lambda)V^{\mu}\right)(\lambda) d\lambda - \int_{0}^{l} V_{\mu}\nabla_UU^{\mu}(\lambda) d\lambda
\end{equation*}
 and remembering that the transversial field is always continuous (all the curves are unbroken) we get \eqref{eq:first-E-var}.
 
 For the second variation we can keep on deriving, and remembering the definition of the Riemann tensor:
 \begin{align*}
 	\frac{d^2E[\gamma_s]}{ds^2} &= \frac{d}{ds} \left[\int_{0}^{l} U_{\mu}\nabla_UV^{\mu}(\lambda) d\lambda\right] = \\
 	&= \int_{0}^{l} (\nabla_VU^{\mu})(\nabla_UV_{\mu})(\lambda) + U^{\mu} \nabla_V\nabla_UV_{\mu} (\lambda) d\lambda=\\
 	&= \int_{0}^{l} \left[(\nabla_UV_{\mu})(\nabla_UV^{\mu}) + R_{\mu\nu\alpha\beta}U^{\mu}V^{\nu}V^{\alpha}U^{\beta}\right] d\lambda + U^{\mu} \nabla_V\nabla_UV_{\mu} \Big\vert_0^l +\\
 	&+  \int_{0}^{l} \nabla_UU^{\mu} \nabla_VV_{\mu} (\lambda) d\lambda
 \end{align*}
	the last term being zero when we reduce to \(s = 0\), if \(\gamma\) is a geodesic.

\end{proof}

We are now able to formally prove a rather intuitive property:
\begin{prop}
	\label{prop:perp-critical-gamma}
	The critical points of \(E\), defined as the curves \(\gamma\) such that for any variation \(\gamma_s\), \(E'[\gamma_s]\Big\vert_{s = 0} = 0\), are exactly the normal geodesics from \(P\) to \(q\).
\end{prop}
\begin{proof}
	If \(\gamma\) is an (unbroken) null geodesic we have \(\nabla_UU = 0\) and there are no breaking points. Moreover for fixed endpoint variations \(V^{\mu}(l) = 0\), while for orthogonal geodesics
	\(U^{\mu}V_{\mu}(0) = 0\), hence the last term of \eqref{eq:first-E-var} null, leaving \(E'[\gamma] = 0\) for any variation.
	
	Conversely, if we suppose that \(E'(s =0) = 0\) for every variation of \(\gamma\) in \(\Omega (P,q)\), it is possible to show that each segment \(\gamma[\lambda_i, \lambda_{i + 1}]\) is a geodesic.
	In fact, take \(\textbf{v}\) any tangent vector in \(T_{\gamma(\lambda_i)}M\)  extended on \(\gamma\) by parallel transport, and \(f\) a bump function with support in \([\lambda_i, \lambda_{i + 1}]\); \(V = f\textbf{v}\) produces a fixed endpoint variation of \(\gamma\) and hence for any \(f\) we have 
	\[
	0 = E'[\gamma_s]\vert_{s = 0} = -\int_{\lambda_i}^{\lambda_{i+1}} g_{\mu\nu}\nabla_UU^{\mu} f v^{\nu} d\lambda
	\]
	which leads to \(\nabla_UU^{\mu} = 0\).
	
	We then need to show that the breaks are trivial: let \(v\) be an arbitrary tangent vector in \(\gamma(\lambda_i)\) extended on \(\gamma[\lambda_{i - 1}, \lambda_{i + 1}]\) and \(f\) is a bump function in \([\lambda_{i - 1}, \lambda_{i + 1}]\); we get for any \(v\):
	\[
	0 = E'[\gamma_s]\vert_{s = 0} = -g_{\mu\nu}v^{\nu}\Delta U^{\mu}(\lambda_i)
	\]
	so \(\Delta U^{\mu}(\lambda_i) = 0\).
	
	Finally we are left to show that \(\gamma\) needs to be orthogonal to \(P\). Take any vector \(v \in T_{\gamma(0)}P\) and extend it to a field \(V\) on \(\gamma\) by parallel transport, inducing a variation of \(\gamma\).
	
	\noindent As \(E'[\gamma] = 0\) for any variation in \(\Omega(P, q)\), and since \(\gamma\) is an unbroken null geodesic, we have
	\[
	0 = E'[\gamma_s] \Big\vert_{s = 0} = U^{\mu}V_{\mu} (0) = U^{\mu}v_{\mu},
	\]
	which concludes.
\end{proof}

By definition \(E''[\gamma]\) is exactly the Hessian \(\textbf{H}_{\gamma}\): the latter in fact is the unique \(\R\)-bilinear from \(T_{\gamma}\Omega\) such that 
\[
\textbf{H}_{\gamma}(V, V) = \frac{d^2E[\gamma_s]}{ds^2}\Big\vert_{s = 0}
\]
where \(\gamma_s\) is any variation of \(\gamma\) in \(\Omega(P, q)\) with transverial field \(V\).
As we now know that \(\gamma\) needs to be orthogonal to \(P\) we could also write the second variation of \(E\) employing the second fundamental form:
\begin{equation}
	\label{eq:hessian}
	\textbf{H}(V) \equiv\frac{d^2E[\gamma_s]}{ds^2}\Big\vert_{s = 0} = 
	\int_{0}^{l} \left[(\nabla_UV_{\mu})(\nabla_UV^{\mu}) + R_{\mu\nu\alpha\beta}U^{\mu}V^{\nu}V^{\alpha}U^{\beta}\right] d\lambda - (U_{\mu}\mathrm{I\!I}^{\mu})(V, V)\Big\vert_{\gamma(0)}.
\end{equation}

\section{Focal points along null geodesics}
\label{sec:fp-index-forms}
We see then that the ``relevant'' set of curves to consider in \(\Omega(P, q)\) when studying \(E\) is the one of geodesics normal to \(P\); however this is not enough to guarantee that we found a minimum of \(E\) and therefore we want to develop a method to better distinguish elements in this subset of curves. Besides the study of the second derivative, which is what an analytic approach would suggest, we find a deep relationship with a new geometrical object: focal points.

\vskip 4pt

%%%% pag.281 O'Neill ma secondo me non serve davvero
%If \(V\) is such a field it is characterized by the fact that 
%\[
%\begin{cases}
%V(0) \text{ is tangent to } P; \\
%\Pi_{P}^{\parallel} \frac{dV}{d\lambda} = \Pi_{P}^{\parallel}
%\end{cases}
%\]

\begin{definition}
	Let \(\gamma\) be a geodesic of \(M\) normal to \(P \subset M\). Then \(\gamma(r)\), where \(r \neq 0\) is a \emph{focal point} of \(P\) along \(\gamma\) provided there is a nonzero \(P\)-Jacobi field \(V\) on \(\gamma\), with \(V(r) = 0\).
\end{definition}

A focal point of a submanifold \(P\) along a normal geodesic \(\gamma\) is an \emph{almost}-meeting point of nearby \(P\)-normal geodesics \underline{of the same causal character} of \(\gamma\). If \(\gamma\) is nonnull this is obvious by continuity, while if \(\gamma\) is null this is assured by the following:
\begin{corollary}
	\label{cor:same-causal-character}
	Let \(\gamma\) be a null geodesic normal to a submanifold \(P\) of a Lorentizian manifold \(M\). A \(P\)-Jacobi field \(V\) on \(\gamma\) is the vector field of a variation of \(\gamma\) through null geodesics through \(P\) if and only if \(V \perp \gamma\).
\end{corollary}

\begin{proof}
	Suppose \(\gamma_s\) is such a variation: then \(U_{\mu}U^{\mu}\equiv 0\) by definition, and then
	\[
	0 = \frac{1}{2}\nabla_V(U_{\mu}U^{\mu}) = \nabla_UV^{\mu}U_{\mu}
	\]
	Hence \(\nabla_UV(0) \perp \gamma\); but \(V(0)\) is tangent to \(P\), then we also have \(V(0) \perp \gamma\), which implies, by \ref{lemma:Jacobi-fields-properties}, that \(V \perp \gamma\).
	
	The converse is also true but a bit more complicated to prove, and a formal proof is provided by O'Neill in \cite{o1983semi} in Corollary 10.40.
\end{proof}

As we have that \(V(0) \perp P\) and \(V(l) = 0\), because we are taking a family of normal geodesics in \(\Omega(P, q)\), we get \(V \perp \gamma\), so we know that when we study variation of \(E\) on a null geodesic \(\gamma\) we can restrict our attention to familys of normal null geodesics.

We are finally ready to prove the following key result:
\begin{prop}
	\label{prop:H-positivity-criteria}
	Let \(P\) be a spacelike sumbanifold of a Lorentz manifold. If there are no focal points of \(P\) along a normal null geodesic \(\gamma\in\Omega(P,q)\), then \(\textbf{H}_\gamma\) is negative semidefinite on \(T_{\gamma}^{\perp}\Omega = \{V \in T_{\gamma}\Omega \vert V \perp \gamma\}\). Furthermore if \(V \in T_{\gamma}^{\perp}\Omega \) and \(\textbf{H}_\gamma(V, V) = 0\) then \(V\) is tangent to \(\gamma\).
\end{prop}

\begin{proof}
	Let \(e_1, \ldots, e_{n - 1}\) be a basis for the space of perpendicular \(P\)- Jacobi fields on \(\gamma\). Without loss of generality we can suppose \(e_1\) to be tangent to \(\gamma\), and as there are no focal points on \(\gamma\) we can write \(V = \sum_{i = 1}^{n - 1} f_ie_i\).
	Now, call \(A \coloneqq \sum_{i = 1}^{n - 1} \nabla_U(f_i)e_i\) and \(B \coloneqq \sum_{i = 1}^{n - 1} f_i\nabla_Ue_i\); then
	\[
	\nabla_UV = A + B
	\]
	so
	\begin{align*}
		\nabla_U(V_{\mu}B^{\mu}) &= (\nabla_UV_{\mu})B^{\mu} + V_{\mu}(\nabla_UB^{\mu}) = \\
		 &= A_{\mu}B^{\mu} + B_{\mu}B^{\mu} + \sum_{i = 1}^{n - 1}f'_iV_{\mu}\nabla_Ue_i^{\mu} + \sum_{i = 1}^{n - 1}f_iV_{\mu}\nabla_U\nabla_Ue_i^{\mu}.
	\end{align*}
	By the Jacobi equation \eqref{eq:Jacobi} we have \(\nabla_U\nabla_Ue_i^{\mu} = R\indices{^{\mu}_{\nu\alpha\beta}}U^{\nu}U^{\alpha}e_i^{\beta}\). Moreover, a simple computation leads to \((e_i)_{\mu}\nabla_Ue_j^{\mu} - (e_j)_{\mu}\nabla_Ue_i^{\mu} = 0\) and \(e_i\) is  a Jacobi fields tangent to \(P\) so, by \ref{lemma:shape-identity}
	\[
	(\Pi^{\parallel}\nabla_Ue_i)_{\mu}e_j^{\mu} = - \mathrm{I\!I}(e_i, e_j)^{\mu}U_{\mu}
	\]
	and by symmetry of the shape tensor we then have \((\nabla_Ue_i)_{\mu}e_j^{\mu} = (\nabla_Ue_j)_{\mu}e_i^{\mu}\). This identity implies:
	\[
	\sum_{i = 1}^{n - 1}f'_iV_{\mu}\nabla_Ue_i^{\mu} = \sum_{i, j = 1}^{n - 1}f'_if_j(e_j)_{\mu}\nabla_Ue_i^{\mu} = \sum_{i, j = 1}^{n - 1}f'_if_j(e_i)_{\mu}\nabla_Ue_j^{\mu} = A_{\mu}B^{\mu}
	\]
	So, as \((\nabla_UV^{\mu})(\nabla_UV_{\mu}) = (A^{\mu} + B^{\mu}) (A_{\mu} + B_{\mu})\)
	\[
	\implies \nabla_U(V_{\mu}B^{\mu}) = 2A_{\mu}B^{\mu} + B_{\mu}B^{\mu} - R_{\mu\nu\alpha\beta}V^{\mu}U^{\nu}V^{\alpha}U^{\beta}
	\]
	and we finally have
	\begin{equation}
	\label{eq:V-identity}
		(\nabla_UV^{\mu})(\nabla_UV_{\mu}) + R_{\mu\nu\alpha\beta}U^{\mu}V^{\nu}V^{\alpha}U^{\beta} =  \nabla_U(V_{\mu}B^{\mu})  + A_{\mu}A^{\mu}.
	\end{equation}
	Now, substituting \eqref{eq:V-identity} into \eqref{eq:second-E-var} we get
	\[
	\textbf{H}_{\gamma} = \int_{0}^{l} A_{\mu}A^{\mu}(\lambda) d\lambda + V_{\mu}B^{\mu}\Big\vert_{s = 0} - U_{\mu}\mathrm{I\!I}^{\mu}(V, V)\vert_{\gamma(0)}.
	\]
	The last \(2\) terms cancel with each other because \(V^{\mu}(l) = 0\) and 
	\begin{align*}
		-V_{\mu}B^{\mu}(0) &= - \sum_{i = 1}^{n - 1}f_i V_{\mu}\nabla_Ue_i^{\mu}(0) =\\
		& = - \sum_{i = 1}^{n - 1}f_i V_{\mu}\Pi^{\parallel}\left(\nabla_Ue_i^{\mu}\right)(0) =\\
		& = \sum_{i = 1}^{n - 1}f_i U_{\mu}\mathrm{I\!I}^{\mu}(e_i, V)(0) = U_{\mu}\mathrm{I\!I}^{\mu}(V, V)(0).
	\end{align*}
	At this point we can observe that, as \(A^{\mu}\) is orthogonal to the null vector \(U^{\mu}\) \(A_{\mu}A^{\mu} \le 0\) (\(A^{\mu}\) needs to be spacelike, it's a simple consequence of Cauchy-Schwarts inequality), and also
	\[
	A_{\mu}A^{\mu} = 0 \iff A^{\mu}\parallel U^{\mu}
	\]
	Since \(e_1\) is the only basis vector tangent to \(\gamma\), if \(\textbf{H} \equiv 0\) we must have \(\nabla_Uf_i = 0\) \(\forall i > 1\). But
	\[
	V^{\mu}(l) = 0 \implies f_i(l) = 0 \quad \forall i > 1 \implies f_i\equiv 0 \forall i > 1.
	\]
	Then \(V = f_1e_1\) which is tangent to \(\gamma\), which concludes.
\end{proof}

Proposition \ref{prop:H-positivity-criteria} is already establishing a criteria for the existence of focal points; however vector fields are rather complicated objects, and we can weaken this criteria a little more by ``averaging'' over many vector fields, and be left working only with a scalar trial function (instead of trial fields). This idea, and its consequences have been developed by Fewster and Kontou in \cite{fewster2020new}, and we are now going to follow their observations.

As always, taken the null geodesic \(\gamma\perp P\), let now \(e_i\) with \(i = 1, \ldots, n - 2\) be an orthonormal basis of \(T_{\gamma(0)}P\), and parallel transport them along \(\gamma\) to generate \(\{E_i\}_{i = 1, \ldots, n-2}\). Then, take \(f\) a smooth function with \(f(0) = 1\) and \(f(l) = 0\):
\begin{equation*}
	\textbf{H}(fE_i, fE_i) = \int_{0}^{l} -f'^2(\lambda) + f^2R_{\mu\nu\alpha\beta}U^{\mu}E_i^{\nu}E_i^{\alpha}U^{\beta} d\lambda- U_{\mu}\mathrm{I\!I}^{\mu}(E_i, E_i)\Big\vert_{\gamma(0)}.
\end{equation*}

We then sum over all \(i = 1, \ldots, n - 2\) to get:
\begin{equation}
	\label{eq:hessian-averagded}
	\sum_{i=1}^{n - 2}\textbf{H}(fE_i, fE_i) = - \int_{0}^{l} \left((n - 2)f'^2(\lambda) - f^2R_{\nu\beta}U^{\nu}U^{\beta}\right) d\lambda - (n - 2)f^2U_{\mu}\mathfrak{H}^{\mu}\Big\vert_{\gamma(0)}.
\end{equation}

We have used \(g^{\mu\alpha} = U^{\mu}W^{\alpha} + U^{\alpha}W^{\mu} - \sum_{i=1}^{n - 2}E_i^{\mu}E_i^{\alpha}\) (with \(W^{\mu}\) a suitable null vector, such that \(W^{\mu}U_{\mu} = 1\) and \(W_{\mu}E_i^{\mu} = 0\), a simple linear algebra exercise), which means:
	\[
	R_{\nu\beta}U^{\nu}U^{\beta} = g^{\mu\alpha}R_{\mu\nu\alpha\beta}U^{\nu}U^{\beta} = - \sum_{i=1}^{n - 2}R_{\mu\nu\alpha\beta}E_i^{\mu}U^{\nu}E_i^{\alpha}U^{\beta}
	\]
	
	and \(\mathrm{H}\) is the mean normal curvature as defined in section \ref{sec:submanifolds}.

	We can now state a sufficient condition for the existence of focal points, but some care is required by the fact that no natural parametrization of null geodesics exist.
	The best way to state the result in an invariant form is to look at \(\gamma\) as an unparametrized \(1\)-dimensional submanifold of \(M\).
	Then, each affine parametrization of \(\gamma\) induces a \(1-\)form \(d\gamma_{\mu}\) and a tangent vector \(U^{\mu}\) such that:
	\[
	d\gamma(U) = d\gamma_{\mu}U^{\mu} \equiv 1
	\]
	and the result proved in \ref{prop:H-positivity-criteria} can be stated as:
	\begin{prop}
		\label{prop:fp-criteria}
		Let \(P\) be a spacelike submanifold of \(M\) of co-dimension \(2\) and let \(\gamma\) be a null geodesic joining \(p \in P\) to \(q\in J^+(P)\). If there exist a a smooth \((-\frac{1}{2})\)-density which is non vanishing at \(p\) but is null at \(q\), and such that
		\begin{equation}
		\label{eq:fp-criteria}
		\int_{\gamma} \big((n -2)(\nabla_Uf)^2 + f^2\text{Ric}(U, U) \big)d\gamma\le -(n -2) g(f^2 U, \mathfrak{H})\Big\vert_{p},
		\end{equation}
		then there is a focal point to \(P\) along \(\gamma\). If the inequality is strict then the focal point is located before \(q\).
	\end{prop}

	In the end, we are finally able to prove this important proposition - also present in \cite{o1983semi} in (10.43) but we state it in the form of \cite{fewster2020new}- which will practically allow us to make use of these methods.
	\begin{corollary}
		\label{cor:fp-criteria}
		Let \(P\) be a spacelike submanifold of codimension \(2\) as usual, and additionally suppose it is future converging; we also ask for the null-converging condition \(\text{Ric}(U, U) \le 0\) to hold everywhere on \(\gamma\). Then, refer to the \(V-\)lenght of \(\gamma\), \(L_V(\gamma)\), as the maximum value reached by an affine parameter \(\lambda\) such that in \(p\) it holds \(V_{\mu}\frac{d\gamma^{\mu}}{d\lambda} = 1\) and \(\gamma(\lambda = 0) = p\). It's true that if \(L_{\hat{\mathfrak{H}}} \ge \frac{1}{|\mathfrak{H}|}\) then there is a focal point to \(P\) along \(\gamma\). (We have written \(\mathfrak{H} = - |\mathfrak{H}|\hat{\mathfrak{H}}\), with \(\hat{\mathfrak{H}}\) future pointing timelike versor).
	\end{corollary}

	\begin{proof}
		Choose an affine coordinate \(\lambda\) on \(\gamma\) such that \(\hat{\mathfrak{H}}_{\mu}\frac{d\gamma^{\mu}}{d\lambda} = 1\); this does exist because \(\hat{\mathfrak{H}}\) is past-pointing timelike, and it can be proven similarly to the \(\mathbf{3) \implies 2)]}\) implication in the proof of \ref{lemma:charact-trapped}, remembering \(\hat{\mathfrak{H}}\) is future-pointing and timelike.
		
		Then we can call \(q = \gamma(l)\) with \(\ell = L_{\hat{\mathfrak{H}}}\); in these coordinates take \(f(\lambda) = 1 - \frac{\lambda}{\ell}\). The right hand side of \eqref{eq:fp-criteria} is equal to \((n-2)|\mathfrak{H}|\), while \(\text{LHS} \le \frac{n - 2}{\ell}\), and the thesis follows from \ref{prop:fp-criteria}.
	\end{proof}
	
	\section{The Raychaudhuri's equation}
	The method presented in section \ref{sec:fp-index-forms} is an alternative way to prove the existence of focal points along geodesics: the much more common procedure - used in all the classical proofs of Singularity theorems \cite{penrose1965gravitational} - goes through the Raychadhuri's equation. We are going to present this other method in this section, and then compare it to the one in the previous section, indeed arguing that the former is strictly more powerful than the latter.
	
	A discussion similar to the following can be found in many articles and textbooks; for consistecy of notation we will refer to chapter \(9\) of \cite{wald2010general} \VS{and to some lectures given by professor Fewster in Spring \(2019\) for the Advanced GR course at the University of York}.
	Usually the discussion is conducted for the timelike case, and then generalized to the null case by claming it's analogous. Here instead we shall start directly from the null case and give the full details of the analysis.
	
	As usal, start from a codimension \(2\) spacelike submanifold \(P\) and consider \(P\)-Jacobi fields on a normal null curve \(\gamma \perp P\) (parametrized by the affine parameter \(\lambda\)), with transverse field \(V^{\mu}\) and tangent field \(U^{\mu}\).
	\begin{remark}
		In \cite{wald2010general} Wald, when referring to the null case, specifies that if \(V^{\mu}\) is a Jacobi field, then also \(V^{\mu} + (a + b\lambda) U^{\mu}\) is a Jacobi field - where \(a\) and \(b\) are some real constants - and wants to consider the quotient of the space of Jacobi fields \(\Omega (P,q)\) by this equivalence relation. Notice that the space of \(P\)-Jacobi fields is already a set of good representatives, and then can be reconducted to the quotient above mentioned; by restricting to study only \(P-\)Jacobi fields from the beginning we can simply forget about the equivalence relation and the discussion won't loose its general character.
	\end{remark}

	Consider the velocity gradient (with the Levi-Civita connection): this forms a tensor field
	\[
	B_{\mu\nu} = \nabla_{\nu}U_{\mu}
	\]
	Clearly \(B_{\mu\nu}U^{\nu} = 0\) as \(\gamma_s\) is a family of geodesics. We also have that \(\gamma_s\) are all \emph{null} geodesics (recall corollary \ref{cor:same-causal-character}), so also \(B_{\mu\nu}U^{\mu} = 0\). Now, we may observe that
	\[
	\frac{D}{D\lambda} V^{\mu} = \nabla_U V^{\mu} \equiv \nabla_V U^{\mu} = B\indices{^{\mu}_{\nu}} V^{\nu}
	\]
	which immediately gives the interpretation of \(B_{\mu\nu}\).
	
	It's now useful to decompose \(B_{\mu\nu}\) in its irreducible components:
	\[
	\begin{cases}
	\omega_{\mu\nu} \coloneqq B_{[\mu\nu]} \hfill \text{the vorticity.} \\
	\theta \coloneqq g^{\mu\nu}B_{\mu\nu}  \hfill \text{the expansion.} \\
	\sigma_{\mu\nu} \coloneqq B_{(\mu\nu)} - \frac{\theta}{d}g_{\mu\nu} \hfill \quad \text{ the shear.}
	\end{cases}
	\]
	
	\begin{remark}
		As in \ref{eq:hessian-averagded} let \(e_i\) with \(i = 1, \ldots, n - 2\) be an orthonormal basis of \(T_{\gamma(0)}P\), and parallel transport them along \(\gamma\) to generate \(\{E_i\}_{i = 1, \ldots, n-2}\). We can again decompose the metric as \(g^{\mu\nu} = U^{\mu}W^{\nu} + U^{\nu}W^{\mu} - \sum_{i=1}^{n - 2}E_i^{\mu}E_i^{\nu}\), hence 
		\[
		\theta = - \sum_{i=1}^{n - 2}E_i^{\mu}E_i^{\nu} B_{\mu\nu}.
		\]
	\end{remark}
	
	At this point we can compute the evolution equations for these \(3\) quantities; such equations are called Raychadhuri's equation.
	We are particularly interested in the ones for the expansion and the vorticity, so we will omit the one for the shear.
	
	\begin{align}
		\label{eq:Raychadhuri-theta}
		&\frac{D}{D\lambda}\theta = -\frac{\theta^2}{d - 2} - \sigma_{\mu\nu}\sigma^{\mu\nu} + \omega_{\mu\nu}\omega^{\mu\nu}  - R_{\mu\nu}U^{\mu}U^{\nu}; \\
		\label{eq:Raychadhuri-vorticity}
		&\frac{D}{D\lambda}\omega_{\mu\nu} = \frac{2}{3}\theta\omega_{\mu\nu} + 2\sigma\indices{^{\alpha}_{[\nu}}\omega\indices{_{\mu]\alpha}}.
	\end{align}
	
	From \eqref{eq:Raychadhuri-vorticity} we can immediately see that if \(\omega_{\mu\nu}\) is null at some point, it will be null everywhere on the congruence. As for geodesics orthogonal to an hypersurface it can be shown that \(\omega_{\mu\nu}(0) = 0\) at the intersection point, we can immediately infer that we can neglect the vorticity term in the evolution of \(\theta\) \eqref{eq:Raychadhuri-theta}.
	
	\noindent Moreover \(\sigma_{\mu\nu}\sigma^{\mu\nu}\) is quadratic, hence \(\sigma_{\mu\nu}\sigma^{\mu\nu} \ge 0\), and we get to the Riccati inequality
	\begin{equation}
		\label{eq:Riccati-ineq}
		\frac{D}{D\lambda}\theta \le -\frac{\theta^2}{d - 2} - R_{\mu\nu}U^{\mu}U^{\nu}.
	\end{equation}
	

	
	
	
	
	




%\clearpage

\chapter{Black Holes}
\section{Asymptotically flat spacetime}
For this section we refer mainly to chapter \(11\) of \cite{wald1991general}, entirely dedicated to this class of spacetimes.
Here we are interested only to grasp the idea of the definition and understand why it's a fundamental concept, needed to lay the basis for good definitions of black holes and related objects, even if the idea of asymptotic flatness is interesting by itself.

Once again, the definition of this concept is rather subtle: the intuitive idea would be to specify some fall-off rates with which the metric \(g_{\mu\nu}\) ``tends to'' the flat metric  \(\eta_{\mu\nu}\). However, we no longer have a background flat metric, in terms of which the fall-off rates can be specified; in particular, generally there is no global inertial reference frame where to define a preferred radial coordinate \(r\).

One way to work around this problems is to ask for the existence of \emph{any} system of coordinates \(x^0, x^1, x^2, x^3\), such that the metric components behave appropriately at large coordinate values - e.g. \(g_{\mu\nu} = \eta_{\mu\nu} + O(1/r)\), along causal directions.
Even if this definition  is capturing the main idea, it is difficult to work with it because the coordinate invariance of any statement is not immediate, and must be carefully verified.
Furthermore, usually we are interested in taking the limit ``\(r \rightarrow +\infty\)'', but the above notion of asymptotic flatness does not specify precisely how such limits are to be taken.

These difficulties have been solved by formulating the notion of asymptotic flatness making use of ``points at infinity'' that can be ``added'' to the spacetime in a suitable way. This, indeed, is manifestly coordinate independent and, providing some definite boundary points that represent the limit to infinity, eliminates the difficulties of defining a direction where to take such a limit.

\subsection{A useful example: Radiation in Minkowski spacetime}
Let's start analyzing an example, to make it clear what are the difficulties for the formulation of this concept, and to understand why the idea of ``adding a point to infinity'' is particularly clever.

In spherical coordinates the metric of flat space takes the form:
\[
ds^2 = dt^2 - dr^2 - r^2(d\theta^2 + \sin^2\theta d\phi^2)
\]
Closely following what Wald \cite{wald2010general} shows, suppose we want to study properties of radiation carried to infinity by a massless field, such as a Klein-Gordon scalar field \(\varphi\). Since this means taking limits, going to infinity along null directions, it is convenient to introduce advanced and retarded null coordinates
\[
\begin{cases}
	 v = t + r\\
	 u = t - r
\end{cases}
\implies 
ds^2 = dudv - \frac{(v - u)^2}{4}(d\theta^2 + \sin^2\theta d\phi^2).
\]
Suppose we are concerned in analyzing outgoing radiation for example: then, keeping \(u\) fixed, we would like to compute what happens to out physical field \(\varphi\) in the limit \(v \rightarrow +\infty\), and extract information about the radiation.

However, taking such limit is a procedure that does not easily generalize to curved spacetime: it would be a lot easier if infinity was some definite ``place''. Well - no problem, you might say - it would be enough to perform the change of coordinates \(V\coloneqq \frac{1}{v}\), and then evaluate the quantities of interest at \(V = 0\). However, the spacetime metric components now look like:
\[
ds^2 = -\frac{1}{V^2}dudV - \frac{1}{4}\left(\frac{1}{V} - u\right)^2 (d\theta^2 + \sin^2\theta d\phi^2)
\]

These components are singular at \(V = 0\), so we cannot extend the spacetime metric there, and then we are not able to perform any tensor analysis at \(V = 0\) as though it was an ordinary ``place''. 

Now, consider instead a new, unphysical, metric \(\bar{g}_{\mu\nu}\), conformally flat with conformal factor \(\Omega = V\). Then, in these coordinates, the line element becomes
\[
d\bar{s}^2 = -dudV - \frac{1}{4}\left(1- uV\right)^2 (d\theta^2 + \sin^2\theta d\phi^2)
\]
and these components are well behaved in \(V = 0\). Thus, it makes sense to extend the Minkowski manifold, by ''adding in'' the points represented by \(V = 0\).
\begin{remark}
	The originally flat space \((\R^4, \eta_{\mu\nu})\) is unextendable as a spacetime, and in fact cannot be smoothly continued to \(V = 0\). This new unphysical spacetime \((\bar{M},\bar{g}_{\mu\nu} )\), is a different spacetime, only conformally equivalent to Minkowski, and can be extended at \(V = 0\).
\end{remark}
The idea is that here we have brought infinity to a ``finite'' distance by means of a conformal transformation, so we have become able to state precisely ``where infinity lies'' and we can perform tensor analysis.

Is that the end of all our problems? Can we now evaluate any tensor at \(V = 0\) using tensors built by \(\bar{g}_{\mu\nu}\), and forget about the original Minkowski space?
Well, not exactly...

\noindent As said, this new space is unphysical, so we need to translate back any result in the original spacetime, and during such translation we would crash into the fact that the conversion factor \(\Omega = V= \frac{1}{v}\) hopelessly blows up for \(v \rightarrow 0\).

Furthermore, we have found a new way to take the limit to ``future null infinity'', but it is not clear how to generalize it in the case we also want to go towards ``past null infinity'' (\(u \rightarrow -\infty\) and \(v\) fixed). However, all of these drawbacks can be remedied by a more judicious choice of the conformal transformation. Let's consider:
\[
\tilde{g}_{\mu\nu} = \Omega^2 g_{\mu\nu} \quad\quad \Omega^2 = \frac{4}{(1 + v^2)^{-1}(1 + u^2)^{-1}} 
\]
Now \(\tilde{g}_{\mu\nu}\) can be smoothly extended to a ``larger'' spacetime, such that the boundary of the Minkowski region in this larger spacetime, provides us with an appropriate representation of ``infinity''.
It can be easily seen by taking the change of coordinates
\[
\begin{cases}
T = \tan^{-1}v +\ tan^{-1}u \text{ with } -\pi \le T + R \le \pi\\
R = \tan^{-1}v - \tan^{-1}u \text{ with } -\pi \le T - R \le \pi \text{ and } R \ge 0
\end{cases}
\]

\begin{equation}
\label{eq:Einstein-static-metric}
	\implies
	d\tilde{s}^2 = dT^2 - dR^2 + \sin^2R(d\theta^2 + \sin^2\theta d\phi^2).
\end{equation}

\subsection{Definition of asymptotic flatness}
Remarkably, the line element in \eqref{eq:Einstein-static-metric} is exactly the natural Lorentz metric on \(S^3 \times \R\), better known as \emph{Einstein static universe}. Then we have learned the following, interesting fact:
\begin{prop}
	There exists a conformal isometry of Minkowski spacetime \((\R^4, \eta_{\mu\nu})\) into an open region \(O\) of the Einstein static universe \((S^3 \times \R, \tilde{g}_{\mu\nu})\)
\end{prop}

This allows us to define what we mean by ``conformal infinity'' of Minkowski space, with which we refer to the boundary \(\partial O\) of the open region \(O\).
%todo: add drawing  pag. 273 of Wald
\begin{figure}
	\centering
	\includegraphics[scale=1.5]{example-image-duck}
	\caption{A spacetime diagram of Einstein static universe, and Minkowski's immersion.}
	\label{fig:Einstein-static-universe}
\end{figure}

As illustrated in figure \ref{fig:Einstein-static-universe}, this boundary \(\partial O\) can be naturally divided into \(5\) parts:
\begin{enumerate}[label=(\arabic*)]
	\item \emph{Past timelike infinity} \(i^{-}\), the ``bottom vertex point'' given by the coordinates \((R = 0, T = -\pi)\).
	\item \emph{Past null infinity} \(\mathscr{I}^-\), the \(3\)-dimensional null hypersurface given by \((R, T = -\pi + R)\), with \(R \in (0, \pi)\).
	\item \emph{Spatial infinity} \(i^0\), the point at \((R=\pi, T = 0)\).
	\item \emph{Future null infinity} \(\mathscr{I}^+\), the \(3\)-dimensional null hypersurface given by \((R, T = \pi - R)\), with \(R \in (0, \pi)\).
	\item \emph{Future timelike infinity} \(i^{+}\), the ``top vertex point'' given by the coordinates \((R = 0, T = \pi)\).
\end{enumerate}

It can be noticed that all timelike geodesics of Minkowski spacetime begin at \(i^-\) and end at \(i^+\), null geodesics go from \(\mathscr{I}^-\) to \(\mathscr{I}^+\), while spacelike geodesics begin and end at \(i^0\).

From this definition of conformal infinity of Minkowski spacetime it's possible formulate precise asymptotic conditions on physical fields representing the exterior field resulting from a localized source, simply requiring that a suitable power of the conformal factor \(\Omega^{-1}\) (the exact power depends on the stringency of the asymptotic condition and on the physical field under consideration), times the field itself can be extended to conformal infinity \(\partial O\) in a suitably well-behaved manner.

With these conditions, quantities which used to be the result of limits such as \(r\rightarrow +\infty\) or \(v\rightarrow \infty\) are now represented as ordinary tensor fields on \(\partial O\). This is a very satisfactory solution to the second problem that was pointed out in the introduction of this section, namely in which direction the limit to spatial infinity should be taken.

We then turn to the first problem, defining the notion of \emph{asymptotically flat} curved spacetime.

The key idea is exactly that the ability to perform an immersion of Minkowski spacetime into Einstein static universe via a conformal isometry crucially depends on the property ``at infinity'' of the spacetime. The idea is then to \emph{define} a spacetime to be \emph{asymptotically flat} if we can perform any similar construction, namely if there is a conformal isometry that maps the original spacetime into an ``unphysical'' manifold \((\tilde{M}, \tilde{g}_{\mu\nu})\) with properties similar to the Minkowski case.
In this way we would clearly solve both problems mentioned at the beginning of the section, since we would have a manifestly invariant formulation of asymptotic flatness, and we have a precise framework to define the concept of ``infinity''.

\begin{remark}
	However, there are some differences that must be pointed out with respect to the Minkowski spacetime. There are mainly \(2\) important remarks:
	\begin{enumerate}[label=(\Roman*)]
		\item We don't want to impose that a spacetime becomes flat at a ``\emph{fixed} position at early or late times'', therefore we cannot expect timelike future \(i^{+}\) or past infinity \(i^{-}\) to be similar to those of Minkowski. Then, we won't ask for any restriction on the structure of curved spacetime related to the presence of these \(2\) points.
		\item  Even if we want to impose the metric to become flat at spatial infinity, smoothness - or even differentiability- are too strong requirements. Then, even if the conformal infinity of asymptotically flat spacetimes is required to contain \(i^0\), the smoothness properties that hold in the Minkowski case must be significantly weakened.
	\end{enumerate}
\end{remark}

All what has been said up to here is enough to give us the fundamental ideas with which we will need to work, but for the sake of completeness we conclude this section providing a formal definition of asymptotic flatness. The reader interested in further comments can refer to the already cited chapter \(11\) of \cite{wald2010general}, or even to the rather technical discussion of Ashtekar \emph{et.al} \cite{ashtekar1978unified}.
\begin{definition}
	A vacuum spacetime is called \emph{asymptotically flat at null and spatial infinity} if there exists a spacetime \((\tilde{M}, \tilde{g}_{\mu\nu})\), with \(\tilde{g}_{\mu\nu}\) \(C^{\infty}\) everywhere except possibly at a point \(i^0\) where it needs to be \(C^{>0}\), and a conformal isometry \(\psi: M \rightarrow \psi(M)\subset \tilde{M}\) with conformal factor \(\Omega\), so that the following conditions are satisfied:
	\begin{enumerate}[label=(\arabic*)]
		\item The first condition is needed to state that in fact \(i^0\) represents exactly spatial infinity: \(\bar{J^+}(i^0) \cup \bar{J^-}(i^0) = \tilde{M} \setminus M\). Here \(\bar{J^+}(i^0)\) is the closure of \(J^+(i^0)\) and for notational simplicity we write \(M\) instead of \(\psi(M)\). Then \(i^0\) is spacelike related to any point in \(\dot{M}\), and the boundary \(\partial M\) is made of the union  of \(i^0\), \(\mathscr{I}^+ \coloneqq \partial J^+(i^0) \setminus i^0\) and \(\mathscr{I}^- \coloneqq \partial J^-(i^0) \setminus i^0\).
		\item We need to require that no causal pathologies occur near infinity; namely there exists an open neighborhood \(V\) of \(\partial M = i^0 \cup \mathscr{I}^+ \cup \mathscr{I}^-\) such that the spacetime \((V, \tilde{g}_{\mu\nu})\) is strongly causal.
		\item We want \(\Omega\) to be well defined near infinity, so we ask it to be able to be extended to a function on all \(\tilde{M}\) so that it is \(C^2\) at \(i^0\) and \(C^{\infty}\) anywhere else.
		\item \begin{enumerate}
			\item \(\Omega = 0\) at \(\mathscr{I}^+\) and \(\mathscr{I}^-\), with \(\tilde{\nabla}_{\mu} \Omega = 0\).
			\item \(\Omega = 0\) at \(i^0\), with \(\lim_{i^0} \tilde{\nabla}_{\mu} \Omega = 0\) and \(\lim_{i^0} \tilde{\nabla}_{\mu} \tilde{\nabla}_{\nu}\Omega = 2\tilde{g}_{\mu\nu}(i^0)\).
			\end{enumerate}
			The requirement of \(\Omega\) vanishing on \(\partial M\) is saying that at those points an ``infinite stretching'' is involved in going from the unphysical \(\tilde{g}_{\mu\nu}\) to the physical \(g_{\mu\nu}\), so \(i^0\), \(\mathscr{I}^-\) and \(\mathscr{I}^+\) truly represent the infinity of the physical spacetime. Furthermore the requirements on the derivatives of \(\Omega\) imply that the physical metric becomes flat at as one goes to infinity.
		\item Finally, the last requirement is really a technical hypothesis, which is sort of stating that \(\mathscr{I}^+\) and \(\mathscr{I}^-\) have the ``right size''.
		\begin{enumerate}
			\item The map of null directions at \(i^0\) into the space of integral curves of \(n^{\mu} = \tilde{g}^{\mu\nu} \tilde{\nabla}_{\nu}\Omega\) on  \(\mathscr{I}^+\) and \(\mathscr{I}^-\) is a diffeomorphism. This is needed to say that \(\mathscr{I}^+\) and \(\mathscr{I}^-\) have topology \(S^2\times\R\).
			\item For any smooth function \(\omega\) on \(\tilde{M} \setminus i^0\), with \(\omega > 0\) on \(M \cup \mathscr{I}^+ \cup \mathscr{I}^- \) and such that \(\tilde{\nabla}_{\mu}(\omega^4n^{\mu}) = 0\) on \(\mathscr{I}^+ \cup \mathscr{I}^-\), the associated vector field \(\omega^{-1}n^{\mu}\) is complete over \(\mathscr{I}^+ \cup \mathscr{I}^-\). This condition is saying that ``all \(\mathscr{I}^+\) and \(\mathscr{I}^-\) are present'' in the conformally completed spacetime.
			\end{enumerate}
	\end{enumerate}
\end{definition}

As last observation of these section, we remark that we can separately define asymptotically flat spacetimes only at \emph{spatial} or \emph{null} infinity, by simply eliminating the irrelevant parts on the definition above.
Finally, as pointed out by Wald, the definition of \emph{weak asymptotically simplicity} given by Penrose in \cite{penrose1965zero}, and often taken as a criterion for asymptotic flatness, is formally different but implies all the conditions of the above definition, apart from \((5b)\); this last one is needed in order to make sure that asymptotic flatness does not cease to hold even at finite retarded time \cite{geroch1978asymptotically}.

\section{Strongly Asymptotically Predictable Spacetimes}
In order to study some properties of black holes we first need a precise definition of these objects. The first idea would be to translate the concept of ``regions of no-escape'' into the requirement \(B \coloneqq \{p \in M\vert J^+(p)\subseteq B \}\).
However, this is not really satisfying, as in this way \emph{any} causal future, of \emph{any} set, in \emph{any} spacetime would be called a black hole. The key point is that we must take greater care in specifying what regions of spacetime is impossible to ``escape towards'' when trapped in a black hole.

The clever idea is that for asymptotically flat spacetimes, the impossibility of escaping to future null infinity \(\mathscr{I}^+\) provides an appropriate definition of black hole. Going into more detail, we are saying that \(J^-(\mathscr{I}^+)\) is ``well behaved'' but doesn't include the all physical spacetime. This intuition leads to the following definition:
\begin{definition}
	Let \((M, g_{\mu\nu})\) be an asymptotically flat spacetime with associated unphysical spacetime \((\tilde{M}, \tilde{g}_{\mu\nu})\). We say that \((M, g_{\mu\nu})\)  is \emph{strongly asymptotically predictable} if in the unphysical spacetime there is an open region \(\tilde{V} \subset \tilde{M}\), with the closure \(\overline{M \cap J^-(\mathscr{I}^+)}\subset \tilde{V}\) such that \((\tilde{V}, \tilde{g}_{\mu\nu})\) is globally hyperbolic.
\end{definition}

\begin{remark}
	Here the closure is taken in \(\tilde{M}\), so in particular, \(i^0 \in \tilde{V}\). 
\end{remark}

Now,
\begin{definition}
	A strongly asymptotically predictable spacetime \((M, g_{\mu\nu})\) is said to contain a \emph{black hole} if \(M\) is not contained in \(J^-(\mathscr{I}^+)\). Consequently the \emph{black hole region} is defined as 
	\[
	B \coloneqq M \setminus J^-(\mathscr{I}^+)
	\]
	and the boundary of \(B\) is called the \emph{event horizon} \(H\coloneqq \partial J^-(\mathscr{I}^+) \cap M \).
\end{definition}

Let us make a couple of remarks of why it is important to make some assumptions to give the definition of such objects.

\begin{remark}
	\begin{enumerate}[label=(\Roman*)]
		\item The notion of asymptotic flatness is needed to specify a definition of the ``infinity'' that observers trapped in a black hole don't have any hope to reach anymore. The definition might be extended in some cases where a suitable notion of ``infinity'' is provided, but we are not able to say anything in general.
		\item The requirement for strong asymptotic predictability is more an instrumental physical hypothesis, rather than a mathematical one, needed for well-posedness. It is really meant to ensure that physics is predictable on and outside \(H\).
		
		Indeed, asymptotically flat spacetimes which fail to be strongly predictable are said to contain a \emph{naked singularity}, but these cases are believed not to be physically relevant. This statement is the main content of the Cosmic Censorship Conjecture: the curious reader may find its definition in chapter \(12\) of \cite{wald2010general} and some endless literature browsing the internet, starting from \cite{dias2018strong}.
	\end{enumerate}
\end{remark}

%todo: si potrebbe fare una pagina di disegnini, tipo pag. 300 del wald

\section{Genral properties of Black Holes}
%todo: bestiario di proprieta' carine, tipo che non possono splittarsi

\section{Black Holes evaporation}
\label{sec:black-holes-evaporation}




%\clearpage

\printbibliography


\end{document}          
