\documentclass[12pt, a4paper]{article}

% !TeX program = lualatex

\usepackage{pacchetti}

%%%%%%%%%%%%%%%%%%%%%%%%%%%%%%%%%%%%%%% Colori 
\definecolor{darkturquoise}{rgb}{0.0, 0.81, 0.82}
\definecolor{cerisepink}{rgb}{0.93, 0.23, 0.51}
\definecolor{brilliantlavender}{rgb}{0.96, 0.73, 1.0}
\definecolor{fuchsiapink}{rgb}{1.0, 0.47, 1.0}

%%%%%%%%%%%%%%%%%%%%%%%%%%%%%%%%%%%%%%% HEAD COMMANDS	
%\newtheorem{theorem}{Theorem}[section]
%
%\newtheorem{corollary}{Corollary}[theorem]
%
%\newtheorem{lemma}[theorem]{Lemma}
%
%\theoremstyle{definition}
%\newtheorem{definition}{Definition}[section]
%
%\theoremstyle{remark}
%\newtheorem*{remark}{Remark}

%%%%%%%%%%%%%%%%%%%%%%%%%%%%%%%%%%%%%%% MATH SYMBOLS
\newcommand{\R}{{\mathbb{R}}}
\DeclareMathOperator{\Tr}{Tr}

%%%%%%%%%%%%%%%%%%%%%%%%%%%%%%%%%%%%%%%%box carini
\newenvironment<>{ideablock}[1]{%
	\setbeamercolor{block title}{fg=white,bg=idea}%
	\begin{block}#2{#1}}{\end{block}}

\newenvironment<>{defblock}[1]{%
	\setbeamercolor{block title}{fg=white,bg=definition}%
	\begin{block}#2{#1}}{\end{block}}

\newenvironment<>{theoblock}[1]{%
	\setbeamercolor{block title}{fg=white,bg=theorem}%
	\begin{block}#2{#1}}{\end{block}}


%%%%%%%%%%%%%%%%%%%%%%%%%%%%%%%%%%%%%%%%% due nuovi comandi per allineare le cose bene
\newcommand\parallelcontent[2]{
	\begin{columns}[t]
		\column{0.5\textwidth} #1
		\column{0.5\textwidth} #2
	\end{columns}
}
\newcommand\parallelitem[2]{
	\parallelcontent
	{\begin{itemize} \item #1 \end{itemize}}
	{\begin{itemize} \item #2 \end{itemize}}
}

%per fare le parentesi sulla destra di una lista

%%%%%%%%%%%%%%%%%%%%%%%%%%%%%%%%%%%%%%%% Graffe verticali per il testo
\tikzset{My Node Style/.style={midway, right, xshift=3.0ex, align=left, font=\small, draw=none, thin, text=black}}

\newcommand{\VerticalBrace}[4][]{%
	% #1 = draw options
	% #2 = top mark
	% #2 = bottom mark
	% #4 = label
	\begin{tikzpicture}[overlay,remember picture]
	\tikzmath{coordinate \p,\q;\p=(pic cs:#2);\q=(pic cs:#3);\maxx=max(\px,\qx);}
	\draw[xshift=1ex,decorate,decoration={brace, amplitude=1.5ex}, #1] 
	([yshift=1.5ex]{{pic cs:#2} -| \maxx pt,0})  -- ([yshift=-.5ex]{{pic cs:#3} -| \maxx pt,0})
	node[My Node Style] {#4};
	\end{tikzpicture}
}


% Bibliografia
\usepackage[backend=bibtex,doi=false,isbn=false,url=false]{biblatex}
\addbibresource{bibliografia.bib}

\title{PhD Project Proposal -- abstract}

\author{Veronica Sacchi\thanks{veronica.sacchi@sns.it}}

\begin{document}

\maketitle

My PhD Project proposal express what interests I developed through university, and how I would like to continue my studies. 

I have been very fascinated by gravitation, and the very elegant treatment General Relativity provides us with: in these past few months my attention has been focused on some Relativity theorems, which is the starting point of the project. 
I am interested into them because they are able to make powerful predictions from very few hypotheses about objects such as black holes. Even if they have been regarded as a purely academical problem for a rather long time, it is now evident that this is not the case, given the many different proofs recent technology has been able to show us.
I have divided the topics I would like to work on into \(3\) different main sections: they are all connected by the common thread of improving out current effective description of gravity by observing some pathological behaviour around singularities, but tackle the problem from slightly different starting points, and might require different tools. They include:
\begin{itemize}
	\item Singularity theorems: classical singularity theorems are valid only at a classical level because energy conditions are violated by all sort of quantum fields (and even some classical fields); it is interesting to try to prove new singularity theorems that extend the classical versions, and realizing what is the minimal contraction required by the stress energy tensor of a field to give strong enough geometric restrictions on the Ricci tensor would be a big leap forward.
	\item Energy conditions: for the development of a modern singularity theorem it is very important to develop a convincing energy condition, strong enough to have important geometrical implications, but weak enough not to be violated by the known fields present in our universe. The search for such a condition has been lasting several decades, and there have been some new important developments lately, even if a deep understanding of the physical meaning they carry is still lacking; an investigation about the direct implications these conditions have on more ``macroscopic'' features of a theory, such as the stability of its solutions, is then needed and would be the object of this part of my research.
	\item Cosmic Censorship Conjecture: in order to properly translate the thesis of singularity theorems, which only really prove geodesics' incompleteness, into something closer to what we intend as a singularity in our spacetime, an additional hypothesis is needed. Even the formulation of such a conjecture has been longly debated, and several versions of it have proved their strengths and weaknesses. Of particular notice are the Strong Cosmic Censorship and the Weak Cosmic Censorship: the first version has been proved wrong in the Reissner-Nordst\"om solution, but there is a lot of ongoing study about whether the coupling of gravity with some matter fields would be able to recover it. 
\end{itemize}


\end{document}          
