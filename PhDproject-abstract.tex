\documentclass[12pt, a4paper]{article}

% !TeX program = lualatex

\usepackage{pacchetti}



\definecolor{turquoise}{RGB}{0, 247, 230}
\definecolor{goldenyellow}{RGB}{255, 218, 66}
\definecolor{fuchsia}{RGB}{255, 0, 172}

%%%%%%%%%%%%%%%%%%%%%%%%%%%%%%%%%%%%%%% HEAD COMMANDS	
\newtheorem{theorem}{Theorem}[section]

\newtheorem{corollary}[theorem]{Corollary}

\newtheorem{lemma}[theorem]{Lemma}

\newtheorem{prop}[theorem]{Proposition}

\theoremstyle{definition}
\newtheorem{definition}{Definition}[section]

\theoremstyle{remark}
\newtheorem*{remark}{Remark}

\newcommand{\EAK}[1]{\textcolor{red}{EAK: #1}}
\newcommand{\VS}[1]{\textcolor{cyan}{VS: #1}}

%%%%%%%%%%%%%%%%%%%%%%%%%%%%%%%%%%%%%%% MATH SYMBOLS
\newcommand{\R}{{\mathbb{R}}}
\newcommand{\N}{{\mathbb{N}}}

 \newcommand{\pprec}{\prec\mathrel{\mkern-5mu}\prec}






% Bibliografia
\usepackage[backend=bibtex,doi=false,isbn=false,url=false]{biblatex}
\addbibresource{bibliografia.bib}

\title{PhD Project Proposal -- abstract}

\author{Veronica Sacchi\thanks{veronica.sacchi@sns.it}}

\begin{document}

\maketitle

My PhD Project proposal expresses what interests I developed throughout university, and how I would like to continue my studies. 

I have been very fascinated by gravitation, and the very elegant treatment General Relativity provides us with: in these past few months my attention has been focused on some Relativity theorems, which are the starting point of my master project. 
I am interested into them because from very few hypotheses they are able to make powerful predictions  on pretty complicated objects, such as black holes. Even if they have been regarded as a purely academical problem for a rather long time, it is now evident that this is not the case, given the many different proofs recent technology has been able to show us.
I have divided the topics I would like to work on into \(3\) different main sections: they are all connected by the common thread of improving our current effective description of gravity by observing some pathological behaviour around singularities, but tackle the problem from slightly different starting points, and might require different tools. They include:
\begin{itemize}
	\item Singularity theorems: classical singularity theorems are valid only at a classical level because energy conditions are violated by all sort of quantum fields (and even some classical fields); it is interesting to try to prove new singularity theorems that extend the classical versions, and realizing what is the minimal contraction, required by the stress energy tensor of a field theory, to get strong enough geometric restrictions on the Ricci tensor would be a big leap forward.
	\item Energy conditions: for the development of a modern singularity theorem it is very important to develop a convincing energy condition, strong enough to have important geometrical implications, but weak enough not to be violated by the known fields present in our universe. The search for such a condition has been lasting several decades, and there have been some new important developments lately, even if a deep understanding of the physical meaning they carry is still lacking; an investigation about the direct implications these conditions have on both ``macroscopic'' and ``microscopic'' features of a theory is then still needed and would be the object of this part of my research.
	\item Cosmic Censorship Conjecture: in order to properly translate the thesis of singularity theorems, which only really prove geodesics' incompleteness, into something closer to what we intend as a singularity in our spacetime, an additional hypothesis is needed. Even the formulation of such a conjecture has been longly debated, and several versions of it have proved their strengths and weaknesses. Of particular notice are the Strong Cosmic Censorship and the Weak Cosmic Censorship: the first version has been proved wrong in the Reissner-Nordst\"om solution, but there is a lot of ongoing study about whether the coupling of gravity with some matter fields would be able to recover it. 
\end{itemize}

Although the topics I have been writing about are somethink I believe I'd love to explore, I wouldn't like them to become my limits from this very moment.

I am very well aware that, as I am approaching right now the opportunity to do some research, my interests are highly influenced by the problems I had a chance to encounter in my previous academic life, which is definetley rather limited if compared to the huge and rich universe Maths and Physics offer to us.

I would like to remark then that what I really want from my PhD is to be exposed to many different topics and refine that spirit of curiosity that has guided me through all these years up to here.

Among few other themes I think I would really like to explore, but I haven't had the chance yet, there are CFTs and the Ads/CFT correspondance, the Information Paradox and more generally connections between Thermodynamics and Gravity.


\end{document}          
