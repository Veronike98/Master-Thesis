This project represents a milestone for a journey that started a long time ago, and there are many people I wish to thank for having taken part into it.

The first person I wish to thank is someone this adventure would have never started without: I will never forget the many nights my father dedicated to me when I first struggled in approaching Physics with a quantitative language, and he is the first who taught me the pleasure of solving riddles \emph{together}. In front of any problem, the first thing my mind recalls is the echo of his most frequent question ``E ora, come si fa?''. The voice of that enquiring spirit which never surrenders in front of any obstacle, and is always curious about how to climb over or to get around it. A spirit that I admire very much, and a fundamental skill for anyone who seeks for new developments in any field of knowledge.

If my father was the Moon, shining glimpses of light into dark nights, at the same time my mum has been the Sun, the most important source of energy for the continuous work endured. It always takes a long practice to master any ability, even the simplest one, and she has always set the highest example of perseverance for me. I believe her tenancy due to her ability of marveling in front of the smallest little things, that many others would take for granted. That marvel is what keeps passions alive, and finally pushes us through any daily struggle, so I would like to especially thank her - among everything she has done for me - for showing me this way of powering all my activities.

More generally, I am deeply grateful to my all family, for the tireless support they always granted me, for always believing in me and in my passions, all the same with reserving me a cozy little place where I could feel home anytime I needed. As for that, a special part is played by my brother, who always has a joke to make me smile, even in the most cloudy day.

I would now wish to thank very much my supervisor, Eleni A. Kontou, first of all for giving me the possibility of exploring this project, something I always dreamed about and that eventually she gave me the possibility to realize. I could never describe the joy I felt when she told me ``I like working with students, especially the ones with similar interests'', that marked the starting point of our collaboration, and how it was renewed by her patience in listening to all the -- sometime foolish -- ideas I have been proposing the all time. The credit she gave to any new proposal I had, and the time she invested in finding references for it, are expressions of a trust that I feel very lucky for, and I deeply wish to thank her for the careful supervision and the almost day-to-day discussions I benefited from.

Related to that, I would like to address many thanks to professor E. Trincherini and professor G. Pimentel for introducing me to her, and for checking from time to time that everything was working well. The former has also been my main advisor for the long search of a master project that could passionate me, a rare care I am very grateful for, and I am very honored by his interest in my activities. 
Moreover, I would like to thank the University of Amsterdam for making possible the close collaboration with my advisor. In particular, I will always remember the herd of office 172.B, for hosting me with the warmest welcome, and for turning this working project into a life experience.
Additionally, a very special thank is due to Gimmy, for being the most enthusiast supporter of this thesis, for the most attentive reading he reserved to this document, and for all the mid-afternoon chats on the terrace, where I could always redefine the big picture of this work, while admiring  the green fields and the fresh air of Science Park.
Together with Gimmy, I would like to deeply thank also Alberto and Elia, for making me feel home in a foreign country as the Netherlands, and for enforcing that very special feeling of community, created in Pisa but enduring anywhere.

If that is the community that welcomed me during the last part of this journey, I am going to never forget the many years that preceded it, populated by the most amazing personalities, and some of the brightest people I could ever hope to meet. In particular, the ``Confraternita del Kebab'' has been a second family for me in the past 5 years, and I would like to thank them very much for the many afternoons and nights spent together.
Among them the two Luca who managed to live with me cover a very special place, and I am very grateful to them for coping with all my moments of terrible desperation, all the same with sharing the expansive joy of other times.

I am very obliged to the other Physicists of my year for all the patience with which they always listened to me and help me out with the many questions arose during the studying of our courses. 

Moreover, I need to thank Jon very much, for making sure this document would have been readable also by a native speaker, and for being a very patient friend while I am often busy with the many tasks of university life.

Finally, I would like to thank a lot Francesca for being a constant presence in these five years, for our close friendship and our fruitful collaboration, that made us realise so many objectives I am extremely proud of, without which everything would have been different.

There are undoubtedly many people that I missed to mention in these acknowledgments; this is not because they are not significant to me, but it must be attributed to my deceptive memory, and I thank very much all of them for being there anyway, remembering me how important is to take care of our relationships. This might seem a lonely job, a solitaire journey within a vast sea, while nothing could be further than that: the incredibly complex twine composed of our connections is -- for me at least -- an immense source of ideas and motivation, the roots that keep me connected to this planet and at the same time the foliage from which fruits develop, a web of interactions capable of powering up the skills of any individual far more that what a single person could do, and indeed I shall never forget about that.