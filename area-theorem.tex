%!TEX root = 00_main.tex

\section{The classical Black Hole Area Theorem}
\label{sec:classical-bh-area}

%todo; scrivere che BH non possono separarsi
Our statement of the Black Hole Area Theorem comes from \cite{wald2010general} (theorem \(12.2.6\)):
%todo: restate the tehorem only using \mathscr{H}, non serve distinguere tra 1 e 2 secondo me.
\begin{theorem}[Black Hole Area Theorem]
	\label{th:classical-bh-area}
	Let \((M, g_{\mu\nu})\) be a strongly asymptotically predictable spacetime satisfying \(R_{\mu\nu}k^{\mu}k^{\nu} \ge 0\) for all null vectors \(k^{\mu}\) (\emph{null convergence condition}). Let \(\Sigma_1\) and \(\Sigma_2\) be spacelike Cauchy surfaces for the globally hyperbolic region \(\tilde{V}\) such that \(\Sigma_2 \subset I^+(\Sigma_1)\), and given \(H\) the event horizon we define
	\[
	\mathscr{H}_1 = H \cap \Sigma_1 \quad \quad \mathscr{H}_2 = H \cap \Sigma_2
	\]
	Then the area of \(\mathscr{H}_2\) is greater or equal than the area of \(\mathscr{H}_1\).
\end{theorem}

\begin{proof}
	This theorem is usually proven by showing, through a reductio ad absurdum, that the expansion \(\theta\) of the null geodesic generators is everywhere non-negative. Here we wish to follow a slightly different path, inspired by the index form methods, that makes use of another object, defined in section \ref{sec:submanifolds}: the mean curvature.
	
	Let \(\Sigma_1\) be any Cauchy hypersurface for \(\tilde{V}\) through \(p\), and call \(\mathscr{H}_1 = H \cap \Sigma_1\), as above. Finally refer to \(\mathrm{H}^{\mu}\) as the mean normal curvature of \(\mathscr{H}_1\).
	The core of the proof is showing that, for the tangent field \(U^{\mu}\) of the null generators of the horizon \(H\), it holds everywhere that:
	\begin{equation}
	\label{eq:exp-null-generators}
		\mathrm{H}^{\mu}U_{\mu} \ge 0.
	\end{equation}
	By contradiction, suppose instead that \(\mathrm{H}^{\mu}U_{\mu} < 0\) at \(p\in \mathscr{H}_1\). We then want to extend the function \(\mathrm{H}^{\mu}U_{\mu}\) in a continuous way on \(\Sigma_1\) (at least in a neighborhood of \(\mathscr{H}_1\)). 
	
	In order to do that, take any deformation of \(\mathscr{H}_1\) outward on \(\Sigma_1\), say \(\mathscr{H}_1'\) and call \(K\) the closed region in \(\Sigma_1\) between \(\mathscr{H}_1\) and \(\mathscr{H}_1'\); the boundary of its future \(\partial J^+(K)\) is a null hypersurface of codimension \(1\), and hence comes with its own null generators, with tangent field \(U'^{\mu}\). This allows us to define the function in \(p' \in\mathscr{H}_1'\) as simply the contraction 
	\(\mathrm{H}'^{\mu}U'_{\mu}\) (with \(\mathrm{H}'^{\mu}\) mean normal curvature of \(\mathscr{H}_1'\)). Different deformations of \(\mathscr{H}_1\) may give different extensions, but that's not important, since we only need that there exists one so that the extension is smooth.
	
	\begin{figure}
		\centering
%		todo: fai figura per capire estensione, vedi iPad.
		\includegraphics[scale=1.7]{example-image-duck}
	\end{figure}
	Given that extension, it exists a neighborhood of \(p\) in which \(\mathrm{H}^{\mu}U_{\mu}(p) < 0\), we can pick a deformation \(\mathscr{H}_1\) outward on \(\Sigma_1\), to \(\mathscr{H}_1'\), such that
	\[
	\begin{cases}
	J^-(\mathscr{I}^+) \cap \mathscr{H}_1' \neq \emptyset; \\
	\mathrm{H}^{\mu}U'_{\mu} < 0 \text{ everywhere on } 	J^-(\mathscr{I}^+) \cap \mathscr{H}_1'
	\end{cases}
	\]
	
	Pick now a point \(q \in \mathscr{I}^+ \cap \partial J^+(K)\):  (\VS{This exists for sth similar to 12.2.6 of Wald})
	the null geodesic generator through \(q\) will meet \(\mathscr{H}_1'\) orthogonally because of \ref{prop:global-existence} and \ref{prop:perp-critical-gamma}; however in \(p' = \gamma \cap \mathscr{H}_1'\) \(U_{\mu}\mathrm{H}^{\mu} < 0\), so by proposition \ref{prop:fp-criteria} we know that a focal point to \(\mathscr{H}_1'\) must develop on \(\gamma\) before reaching \(q\). In fact, it is enough to choose \(f = 1 - \frac{\lambda}{\ell}\), with \(\lambda\) affine parameter such that:
	\[
	\begin{cases}
	\hat{\mathrm{H}}^{\mu}U_{\mu} = 1 \\
	\gamma(\lambda = 0) = p' \\
	\ell \ge  \frac{1}{\vert U_{\mu}\mathrm{H}^{\mu} \vert}
	\end{cases}
	\]
	and 
	\[
	\int_{0}^{\ell} \big((n -2)(\nabla_Uf)^2 - f^2R_{\mu\nu}U^{\mu}U^{\nu} \big)d\lambda\le 
	\frac{n -2}{\ell} \le -(n -2) U_{\mu} \mathrm{H}^{\mu} =
	-(n -2) g(f^2 U, \mathrm{H})\Big\vert_{p},
	\]
	This is impossible, because the null generator cannot contain any focal point, and hence 
	\[
	\forall p \in H \quad \mathrm{H}^{\mu}U_{\mu}(p) \ge 0.
	\]
		
	To conclude, it is enough to observe that each \(p\in \mathscr{H}_1\) lies on a null generator \(\gamma\) contained in \(H\). As \(\Sigma_2\) is a Cauchy hypersurface as well, \(\gamma\) must intersect \(\Sigma_2\) in a point \(q \in \mathscr{H}_2\). Then, the flow along null generators maps \(\mathscr{H}_1\) into a portion of \(\mathscr{H}_2\).
	
	But we know that under deformation along the flow of a vector field, the area of a submanifold evolves as \ref{eq:variation-area}, so:
	\begin{equation*}
		\delta_U\mathcal{A}_{\mathscr{H}_1} = \int_{\mathscr{H}_1} \mathrm{H}^{\mu}(p)U_{\mu} \ge 0
	\end{equation*}
	which is telling us that when we modify \(\mathscr{H}_1\) along the flow of the null generators the area of \(\mathscr{H}_1\) can never decrease.
\end{proof}

\section{The damped Averaged Null Energy Condition -- dANEC}
In section \ref{sec:classical-bh-area} we proved the black holes area theorem under classical hypothesis. As discussed in %todo: scrivi intro a energy conditions e metti una ref
the Null Energy Conditions is violated by any sort of quantum fields, while we wonder in what cases - if any - the theorem can be extended in a semiclassical regime.

One first attempt is obtained replacing the null energy conditions with its damped version; this result was already obtained by Lesourd in \cite{lesourd2018remark}, where it was used the Raychaudhuri's equation once again, while here we would like to propose an alternative proof, in the same spirit of theorem \ref{th:classical-bh-area}.
\begin{definition}
	A spacetime satisfies the \emph{damped averaged null energy condition} if along each future complete null geodesic \(\gamma\), affinely parametrized by \(\lambda\), there exists a non-negative constant \(c\ge 0\) such that:
	\[
	\liminf\limits_{\Lambda\rightarrow \infty} \int_{0}^{\Lambda} e^{-ct}R_{\mu\nu}U^{\mu}U^{\nu}d\lambda - \frac{c}{2} > 0
	\]
	where \(U^{\mu}\) is \(\gamma\)'s tangent field.
\end{definition}

The statement of the theorem is the same as \ref{th:classical-bh-area}, but replacing the Null convergence condition with dANEC; the proof is pretty similar as well but this time we will need to choose a different trial function when using proposition \ref{prop:fp-criteria}.
\begin{proof}
	Define \(\mathscr{H}_1\), and consider \(U_{\mu}\mathrm{H}^{\mu}\) exactly as in \ref{th:classical-bh-area}. Again we want to show that it needs to be \(U_{\mu}\mathrm{H}^{\mu} > 0\) everywhere on the horizon. In order to do it, by contradiction, assume there exists \(p\in \mathscr{H}_1\) for which this doesn't hold, and construct the extension \(U'_{\mu}\mathrm{H}'^{\mu}\) in a neighborhood of \(p\) in the same way as before.
	Now, pick \(f = e^{-c\frac{\lambda}{2}}\): the left hand side of \ref{eq:fp-criteria} in \(n = 4\) dimensions becomes:
	\[
	\int_{0}^{+\infty} (n-2)\frac{c^2}{4} e^{-c\lambda} - e^{-ct}R_{\mu\nu}U^{\mu}U^{\nu}d\lambda = \frac{c}{2} - \int_{0}^{+\infty} e^{-ct}R_{\mu\nu}U^{\mu}U^{\nu}d\lambda 
	\]
	which is negative, for the assumption of dANEC.
	Instead, \(U'_{\mu}\mathrm{H}'^{\mu} < 0 \) implies
	\[
	-(n - 2)U'_{\mu}\mathrm{H}'^{\mu} > 0.
	\]
	From this it follows that \ref{eq:fp-criteria} holds, and hence  a focal point is formed along \(\gamma\), which is, again, absurd. 
	
	By this contradiction we have that along the null generators \(U_{\mu}\mathrm{H}^{\mu} > 0\), and the conclusion is reached exactly ad in \ref{th:classical-bh-area}.
\end{proof}

Assuming the dANEC we are still able to prove that the area cannot decrease: this is a signal that we are still working with a classical condition; we will now move forward to conditions that reach weaker conclusions and hopefully are derived from weaker conditions.

\section{The Sobolev condition}
In this section we will assume the validity of a different energy condition, that uses Sobolev norms, and that's why we call it \emph{Sobolev condition}.
This assumption is motivated by the work in section \(4\) of \cite{fewster2020new}, where an analogous condition is assumed in order to derive singularity theorems.

\begin{definition}
	We say that the \emph{Sobolev condition} is satisfied on a curve \(\gamma\), from a Cauchy surface \(\Sigma\) for a ``time'' \(\ell\), if there exixst some \(m\in \N\) and \(2\) non negative constants \(Q_0\) and \(Q_m\) such that, for any \(\frac{1}{2}-\)density \(f\) on \(\gamma\):
	\[
		\int_0^{\ell} f(\lambda)^2 R_{\mu\nu}U^{\mu}U^{\nu} \ge -Q_m(\gamma) \vert\vert f^{(m)}\vert\vert^2 - Q_0(\gamma) \vert\vert f\vert\vert^2;
	\]
	the affine parametrization is chosen so that \(\gamma(\lambda = 0) \in \Sigma\) and \(\hat{H}_{\mu}\frac{d\gamma^{\mu}}{d\lambda}\Big\vert_{\lambda = 0} \equiv 1\) (\(\hat{H}_{\mu}\) is the versor of the mean normal curvature of \(\Sigma\)) - we will refer to this as the \emph{standard} affine parametrization. Finally, \(\vert\vert \star \vert\vert\) denotes the standard norm in \(L^2(I)\).
\end{definition}

Such a condition has some useful consequences on the functional \(J_{\ell}[f] = \int_0^{\ell} (n - 2)(\nabla_Uf)^2 - f^2R_{\mu\nu}U^{\mu}U^{\nu}(\lambda) d\lambda \).
\begin{lemma}
	\label{lemma:J-sobolev-condition}
	Let 
	\begin{itemize}
		\item[\ding{99}] the Sobolev condition be satisfied on \(\gamma\), from \(\Sigma\) for a ``time'' \(\ell\);
 		 \item[\ding{99}] the quantity \(\rho(\lambda) \coloneqq R_{\mu\nu}U^{\mu}U^{\nu}(\lambda) \) on \(\gamma\) be lower controlled by \(\rho_0\) in \([0, \ell_0] \subseteq [0, \ell]\), or in other words
    	 \[
			R_{\mu\nu}U^{\mu}U^{\nu}(\lambda) \ge \rho_0 \quad\quad \forall\lambda\in [0, \ell_0].
		\]
	\end{itemize}
	Now define \(f\) as:
	\begin{equation}
		f(\lambda) = 
		\begin{cases}
			1 \hfill \lambda\in [0, \ell_0) \\
			I(m, m; \frac{\ell - \lambda}{\ell - \ell_0}) \quad \hfill \lambda\in [\ell_0, \ell].
		\end{cases}
	\end{equation}
	where \(I(m,m;x)\) is the regularized incomplete Beta function.
	Then for this specific choice of \(f\) it holds:
	\begin{equation}
		\label{eq:J-sobolev-condition}
		J[f] \le (n - 2)\mathcal{V} \coloneqq -(1-A_m)\rho_0\ell_0 + \frac{Q_mC_m}{\ell_0^{2m-1}} + Q_0A_m\ell + \frac{(n - 2)B_m}{\ell - \ell_0} + \frac{Q_mC_m}{(\ell-\ell_0)^{2m-1}},
	\end{equation}
	where \(Q_m\), \(Q_0\) have been defined with the Sobolev condition, while \(A_m\), \(B_m\) and \(C_m\) are the sobolev norms of the regularized incomplete \(m-\)Beta function, its first and \(m-\)th derivative. For the sake of completeness their values are:
	\[
	A_m = \frac{1}{2} - \frac{(2m)!^4}{4(4m)!m!^4} \quad 
	B_m= \frac{(2m-2)!^2(2m-1)!^2}{(4m-3)!(m - 1)!} \quad 
	C_m = \frac{(2m-2)!(2m-1)!}{(m-1)!^2}. 
	\]
\end{lemma}

\begin{proof}
	The proof of this lemma can be found in \cite{fewster2020new} as lemma \(4.1\), and it proceeds as follows. Define a piecewise smooth function \(\varphi\) on \([0,\ell]\) by:
	\[
	\varphi(\lambda) = 
	\begin{cases}
		I(m, m;\frac{\lambda}{\ell_0}) \quad \hfill \lambda \in [0, \ell_0) \\
		1 \hfill \lambda \in [\ell_0, \ell]
	\end{cases}	
	\]

	\begin{figure}
		\centering
		\includegraphics[scale=1.7]{example-image-duck}
		%todo: inserisci plot di f e phi
	\end{figure}
	
	It's possible to see that \(\varphi f \in W_0^m([0,\ell])\), so writing \(f^2 = (\varphi f)^2 + (1 - \varphi^2)\) we have:
	\begin{align*}
		\int_0^{\ell} f(\lambda)^2\rho(\lambda) &\ge \int_0^{\ell} (1 - \varphi(\lambda)^2)\rho(\lambda)d\lambda - Q_m(\gamma) \vert\vert (\varphi f)^{(m)}\vert\vert^2 - Q_0(\gamma) \vert\vert \varphi f\vert\vert^2 \\
		&\ge  \rho_0\int_0^{\ell} (1 - \varphi(\lambda)^2)d\lambda - Q_m(\gamma) \vert\vert (\varphi f)^{(m)}\vert\vert^2 - Q_0(\gamma) \vert\vert \varphi f\vert\vert^2.
	\end{align*}

	Moreover, the norms of the functions can be analytically computed for the specific choices that we made:
	\[
		\vert\vert \varphi f\vert\vert^2 = A_m\ell_0 + A_m(\ell - \ell_0)\quad\quad
		\vert\vert (\varphi f)^{(m)}\vert\vert^2 = \frac{C_m}{\ell_0^{2m - 1}} + \frac{C_m}{(\ell - \ell_0)^{2m - 1}}
		\quad\quad 
		\vert\vert f'\vert\vert^2 = \frac{B_m}{\ell - \ell_0}
	\]
	Thus,
	\[
		J[f] \le \frac{(n - 2)B_m}{\ell - \ell_0} + \frac{Q_mC_m}{\ell_0^{2m - 1}} + \frac{Q_mC_m}{(\ell - \ell_0)^{2m - 1}} + Q_0A_m\ell - \rho_0\ell_0(1 - A_m) \coloneqq (n-2)\mathcal{V},
	\]
	and the estimate is complete.
\end{proof}

As in the previous section we are now able to infer some conditions on the area of black holes. However, in this case we don't reach exactly the same conclusion as before:
\begin{theorem}
	\label{th:sobolev-bh-area}
	Let \((M, g_{\mu\nu})\) be a strongly asymptotically predictable spacetime satisfying the Sobolev condition for all normal future-pointing null geodesics leaving from a Cauchy surface \(\Sigma\), and for a ``time'' \(\ell\). Moreover, assume that the quantity \(\rho(\lambda) \coloneqq R_{\mu\nu}U^{\mu}U^{\nu}(\lambda) \) on \(\gamma\) be lower controlled by \(\rho_0\) in \([0, \ell_0] \subseteq [0, \ell]\). 
	
	% Let \(\Sigma_2\) be the image of \(\Sigma_1\) along the flux of normal null geodesics after a standard affine parameter \(\ell\), \(\Sigma_2 = F_{\ell}(\Sigma_1)\); 
	Moreover, if \(H\) is the event horizon, and
	\[
	\mathscr{H} = H \cap \Sigma ;
	% \quad \quad \mathscr{H}_2 = H \cap \Sigma_2
	\]
	then the area of \(\mathscr{H}\) istantaneously cannot decrease at a faster rate than \(\mathcal{V}\), 
	where \(\mathcal{V}\) is some positive constant that depends on \(Q_0\) and \(Q_m\), among others, and is defined in \eqref{eq:J-sobolev-condition}.
\end{theorem}

\begin{proof}
	For this proof we proceed in a similar way as in the previous sections; we want to prove that \(\forall p \in \mathscr{H}\) \(U_{\mu}\mathrm{H}^{\mu}(p) \ge -\mathcal{V}\), where \(U^{\mu}\) is the tangent field of the null generator of \(H\) through \(p\), and \(\mathrm{H}^{\mu}(p)\) is the mean normal curvature of \(\Sigma\) in \(p\). We proceed as follows
	\begin{itemize}
		\item[\ding{99}] By contradiction, assume there exists a \(p\in \mathscr{H}\) so that \(U_{\mu}\mathrm{H}^{\mu}(p) < -\mathcal{V}\).
  		\item[\ding{99}] There exists an outward deformation on \(\Sigma\) of \(\mathscr{H}\), in a neighborhood of \(p\), on which \(U_{\mu}\mathrm{H}^{\mu}(p') < -\mathcal{V}\) holds everywhere. This point can be motivated exactly as it has been done for the proof of \ref{th:classical-bh-area}. In particular this implies that \(-(n -2) U_{\mu}\mathrm{H}^{\mu} > (n-2)\mathcal{V}\) for the null generators leaving of \(\partial J^+(K)\).
    	\item[\ding{99}] By means of lemma \ref{lemma:J-sobolev-condition} we have that \(J[f] \le (n - 2)\mathcal{V}\), at least for the specific choice of \(f\) made in such lemma. This leads to a contradiction because - once again - proposition \ref{prop:fp-criteria} comes into play, and the null generator of \(\partial J^+(K)\) through \(p'\in \mathscr{H}'\) is doomed to contain a focal point, in contradiction with the global hyperbolicity of the region outside the black hole.
	\end{itemize}
	We have then proved that 
	\[
		U_{\mu}\mathrm{H}^{\mu}(p) \ge -\mathcal{V}	\quad\quad \forall p \in \mathscr{H}_1
	\]
	Then, as before, we have that the change of the area of \(\mathscr{H}_1\) along the flow of null generators is controlled by
	\begin{equation*}
		\delta_U\mathcal{A}_{\mathscr{H}_1} = \int_{\mathscr{H}_1} \mathrm{H}^{\mu}(p)U_{\mu} \ge - \mathcal{V}\cdot\mathcal{A}_{\mathscr{H}_1}.
	\end{equation*}
	So far we have proved that the rate of change of the area of the horizon near the reference ``time instance'' given by \(\Sigma_1\) cannot be lower than a certain (negative) rate. 
	
	It is not possible to constrain the average rate of evolution between \(2\) Cauchy surfaces \(\Sigma_1\) and \(\Sigma_2\) directly though, because we would need to assume the Sobolev condition, and the pointwise restriction \(\rho \ge \rho_0\) in the all region in between \(\Sigma_1\) and \(Sigma_2\), which would mean a significant weakening of our result.

\end{proof}

\begin{remark}
	At this point, a couple of remarks are needed. 

	\begin{itemize}
		\item[\ding{99}] Coherently with the weakening on the energy condition hypothesis, we reach a weaker result: in fact we are still able to place a lower bound on the rate of change or the area, but the bound is not as strong as it is in the classical case; this is interesting though, because we expect the classical bound to be violated in a semiclassical context: in order to make sure we have reached a meaningful result ideally we need to shown an example where NEC is violated but our new hypothesis hold, and the area of the horizon shrinks, even for a very short time.
  		\item[\ding{99}] Notice that \(1-A_m \ge 0\), so the lower is \(\rho_0\), the weaker is the hypothesis we ask for, but the weaker is our result as well. In fact \(\mathcal{V}\) increases when \(\rho_0\) diminishes, and so we gain a weaker and weaker bound; at the same time, if \(\rho_0\) was positive and very large, \(\mathcal{V}\) would be negative, and so we would have that the area of the black hole not only cannot decrease, but it has to steadily increase.
    	\item[\ding{99}] In order to estimate the value of the functional \(J[f]\) we required \(\rho (\lambda)\ge\rho_0\) for \(\lambda\in [0, \ell_0]\); this is again a pointwise requirement, so it is not particularly satisfying according to the analysis given in section \ref{ch:energy-conditions}. However, in \cite{levi2016versatile} Levi and Ori have been running some numerical simulations to estimate \(\rho\) in the outer proximity of the horizion, in a background Schwarzschild metric, and found \(\rho \simeq -2.7\cdot 10^{-7} \hbar M^{-4}\) (where \(M\) is the mass of the black hole). This supports the legitimacy of our additinal assumption, at least for large enough black holes, and suggests that such a condition might be commonly satisfied even with a rather small value for \(\vert\rho_0\vert\).
	\end{itemize}
	
\end{remark}

\section{Physical interpretation of \(\mathcal{V}\)}
As extensively argued above, \(\mathcal{V}\) represents the bound on the instantaneous rate of change of the area of the black hole horizon. In order to understand the physical impact of the hypothesis we have assumed, we would like to compare it to some relevant physical quantity; in this scenario we are lucky enough to have a particular appealing one: the rate of black hole evaporation.

With that in mind, we need to translate the mathematical expression of \(\mathcal{V}\) given in \(\eqref{eq:J-sobolev-condition}\) into a function of some physical quantities; we will let us be inspired by the work in section \(5.2\) of \cite*{fewster2020new}, where Fewster and Kontou computed \(\mathcal{V}\) in the specific case of a non-minimally coupled classical Einstein-Klein Gordon theory.

They needed such a computation to compare the hypothesis of some weakened singularity theorems to physical scenarios, while here we would like to analyze the evaporation of black holes, and will need indeed to be make some different approximations.

\subsection{The non-minimally coupled Einstein Klein Gordon Theory}
Non minimally coupled scalar fields is the typical classical example that violates NEC, and thus the original black hole area theorem doesn't apply. 

They are scalar fields described by the Lagrangian density
\[
\mathcal{L}[\phi] = \frac{1}{2}\left[\nabla_{\mu}\phi\nabla^{\mu}\phi   - (m^2 - \xi R)\phi^2\right]
\]
The corresponding stress-energy tensor is:
\begin{equation}
    T_{\mu\nu} = \nabla_{\mu}\phi\nabla_{\nu}\phi - \frac{1}{2}g_{\mu\nu}\left[\nabla_{\rho}\phi\nabla^{\rho}\phi - m^2\phi^2\right] - \xi\left(g_{\mu\nu}\square_g - \nabla_{\mu}\phi\nabla_{\nu} + G_{\mu\nu}\right)\phi^2
\end{equation}
\VS{Controlla che sia corretto con le nostre convenzioni! Ho messo i segni un po' a caso}

The conformal coupling is \(\xi_c = \frac{1}{4}\frac{n - 2}{n - 1}\), and we will only consider \(\xi\in [0,\xi_c]\); \VS{Why?}
now the following observation comes along.

\begin{prop}
    It is physicall reasonable to impose that \(8\pi\xi\phi^2 < 1\).
\end{prop}

\begin{proof}
    The argument can be found in \cite*{kontou2020energy} and starts by rearranging Einstein equations in the following form:
    \[
      \frac{1 - 8\pi\xi\phi^2}{8\pi}G_{\mu\nu} =  \underbrace{\nabla_{\mu}\phi\nabla_{\nu}\phi - \frac{1}{2}g_{\mu\nu}\left[\nabla_{\rho}\phi\nabla^{\rho}\phi - m^2\phi^2\right] - \xi\left(g_{\mu\nu}\square_g - \nabla_{\mu}\phi\nabla_{\nu}\right)\phi^2}_{T^{eff}_{\mu\nu}}.
    \]
    Notice that for \(\xi > 0\) the coefficient in front of \(G_{\mu\nu}\) vanishes for \(8\pi\xi\phi^2 = 1\); for such values of the field \(\phi\) the order of Einstein equations reduces, and so the intial value problem becomes ill-defined.

    When \(8\pi\xi\phi^2 \neq 1\) instead, Einstein equations become:

    \[
        G_{\mu\nu} = \frac{8\pi}{1 - 8\pi\xi\phi^2}T^{eff}_{\mu\nu},
    \]
    and so we can see that for \(8\pi\xi\phi^2 > 1\) we would have a negative ``effective Newton's constant''. 

    This is not a definitive proof that non-minimally coupled fields cannot access such high values (called \emph{trans-Planckian}), but it is more a motivation about why it seems reasonable to assume that bound. 
\end{proof}

From here onwards we will assume that non-minimally coupled scalar fields will always obey the bound:
\[
\phi^2 < \frac{1}{8\pi\xi}.    
\]
Moreover we shall only consider \emph{massless} scalar fields; given that, it was shown in \cite{fewster2011singularity} and \cite{brown2018singularity} for ``reasonable'' functions the Ricci tensor contraction is bounded in a form tha can be recasted in the one of the Sobolev condition.

\begin{prop}
    In the context of a massless non-minimally coupled Einsten-Klein Gordon field theory, the integral of the contraction of the Ricci tensor along any causal geodesic, wighted any compactly supported real valued \(-\frac{1}{2}\)-density \(f\), is bounded by:

    \begin{equation}
        \int_{\gamma}f^2 R_{\mu\nu}U^{\mu}U^{\nu} \le Q\left(\vert\vert\nabla_U f \vert\vert^2 + \tilde{Q}^2 \vert\vert f\vert\vert^2\right).
    \end{equation}
\end{prop}

\begin{proof}
    We start from a bound given as equation \((67)\) in \cite{brown2018singularity}.
\end{proof}



\section{A more general point of view}

The all problem is about maximizing J. %todo: chiarisciti se e' max o min per favore.

A new definition of ``classical'' energy condition: a condition is classical if \(\sup J[f] = 0\)
	
	
	
	
	
	

	
	
	
	
	



