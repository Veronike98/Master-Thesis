\section{Asymptotically flat spacetime}
For this section we refer mainly to chapter \(11\) of \cite{wald1991general}, entirely dedicated to this class of spacetimes.
Here we are interested only to grasp the idea of the definition and understand why it's a fundamental concept, needed to lay the basis for good definitions of black holes and related objects, even if the idea of asymptotic flateness is interesting by itself.

Once again, the definition of this concept is rather subtle: the intuitive idea would be to specify some fall-off rates with which the metric \(g_{\mu\nu}\) ``tends to'' the flat metric  \(\eta_{\mu\nu}\). However, we no longer have a background flat metric, in terms of which the fall-off rates can be specified; in particular, generally there is no global inertial reference frame where to define a preferred radial coordinate \(r\).


\section{Strongly Asymptotically Predictable Spacetimes}

\section{Properties of Black Holes}

\EAK{the definition of black hole in that context might fit well here}

Our statement of the Black Hole Area Theorem comes from \cite{wald2010general} (theorem \(12.2.6\)):
%todo: restate the tehorem only using \mathscr{H}, non serve distinguere tra 1 e 2 secondo me.
\begin{theorem}[Black Hole Area Theorem]
	Let \((M, g_{\mu\nu})\) be a strongly asymptotically predictable spacetime satisfying \(R_{\mu\nu}k^{\mu}k^{\nu} \ge 0\) for all null vectors \(k^{\mu}\) (\emph{null convergence condition}). Let \(\Sigma_1\) and \(\Sigma_2\) be spacelike Cauchy surfaces for the globally hyperbolic region \(\tilde{V}\) such that \(\Sigma_2 \subset I^+(\Sigma_1)\), and given \(H\) the event horizon we define
	\[
	\mathscr{H}_1 = H \cap \Sigma_1 \quad \quad \mathscr{H}_2 = H \cap \Sigma_2
	\]
	Then the area of \(\mathcal{H}_2\) is greater or equal than the area of \(\mathcal{H}_1\).
\end{theorem}

\begin{proof}
	This theorem is usually proven by showing, through a reductio ab absurdum, that the expansion \(\theta\) of the null geodesic generators is everywhere non-negative. Here we wish to follow an alternative path, inspired by the index form methods, that makes use of another object, defined in section \ref{sec:submanifolds}: the mean curvature.
	
	As we have seen the equivalence between the \(2\) definitions of trapped surfaces, %todo: write it down and reference
	using a similar reasoning as for lemma \ref{lemma:charact-trapped}, we can see that the thesis is equivalent to proving that, for all \(p \in H\) and it exists a future-directed null (or, equivalently, causal) vector \(v^{\mu}\) in \(T_pH^{\perp}\):
	\[
	\mathfrak{H}^{\mu}(p)v_{\mu} \ge 0.
	\]
	Conversely, suppose that \(\exists p\in H\) such that \(\forall v^{\mu} \in T_pH^{\perp}\), with \(v^{\mu} \) future directed casusal vector, such that \(\mathfrak{H}^{\mu}(p)v_{\mu} < 0.\) Then, by \ref{lemma:charact-trapped} we know that \(\mathfrak{H}^{\mu}(p)\) is past-pointing timelike.
	
	We can now proceed as usual, following for example \cite{wald2010general}. Take the Cauchy surface \(\Sigma\) for the unphysical spacetime \(\tilde{V}\), that passes through \(p\), and define \(\mathscr{H} = H \cap \Sigma\). Since \(\mathfrak{H}^{\mu}(p)\) is past-pointing timelike at \(p\), for continuity we might deform \(\mathscr{H}\) outward in a neighborhood of \(p\) in \(\Sigma\), to obtain a \(2\)-surface \(\mathscr{H}'\) which intersects \(J^-(\mathscr{I}^+)\) non trivially, and such that is past pointing timelike for any point in the intersection \(\mathscr{H}' \cup J^-(\mathscr{I}^+)\).
	
	Call \(K\) the closed region in \(\Sigma\) between \(\mathscr{H}\) and \(\mathscr{H}'\), and pick a point \(q \in \mathscr{I}^+ \cup \partial J^+(K)\). Because of proposition \ref{prop:perp-critical-gamma} we know that the null geodesic generator of \(\partial J^+(K)\) must meet \(\mathscr{H}'\) orthogonally; however proposition \ref{cor:fp-criteria}, tells us that such generator must have a focal point within a finite value of the affine parameter, and thus gives a contradiction. Hence 
	\[
	\forall p \in H \quad \exists v^{\mu} \in T_pH^{\perp} \text{ causal and future directed, such that }  \mathfrak{H}^{\mu}(p)v_{\mu} \ge 0.
	\]
	In particular we can build a vector field \(V \in TH^{\perp}\) such that \(\mathfrak{H}^{\mu}(p)V_{\mu}(p) \ge 0\) everywhere on \(H\).
	Then, as proved in \cite{kriele1999spacetime} and reported in section \ref{sec:submanifolds}, we have that:
	\begin{equation*}
		\delta_VA_{\mathscr{H}_1} = \int_{\mathscr{H}_1} \mathfrak{H}^{\mu}(p)V_{\mu} \ge 0
	\end{equation*}
	when we modify \(\mathscr{H}_1\) along the flow of \(V\), or, in other words, the area of \(\mathscr{H}_1\) can never decrease.
\end{proof}