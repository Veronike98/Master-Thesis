The name ``black holes'', that is used to indicate what before were called ``frozen stars'', is probably one of the best-suited nomenclature ever assigned in Physics; their existence fascinates many people, not only among physicists but also in the general public, which often eagerly follows any discovery in this field of Nature.

Possibly, this enchantment is due to some atavistic fear of the darkness we had to live with for a long time, darkness that was only won over by the breathtaking wonder of the faint light of the stars. It looks very scary then, that the darkness can instead win the battle, and swallow in light \emph{forever}: the sky had very often -- at least in our culture -- been associated to the place where the supreme justice could finally be carried out, so the idea that instead \emph{eternal} prisons might be sitting there is nothing but terrifying.  
	
Besides this genuine attraction, that in some degree many people share, physicists ironically study those objects to actually shed some light onto the many unsolved mysteries gravity is still hiding.
This ``force'' is the faintest interaction that we know about, and therefore the effective description we developed for it works until energies way beyond the ones we could ever imagine to probe on this planet, even on some very far day. To reach the limit of our knowledge then, we are only left with the possibility of studying those configurations where gravity becomes extreme. That is why we study black holes: in hopes of collecting some additional piece of yearned for knowledge, from monsters we learnt are not really eternal either. 

Among others, Penrose's singularity theorem and Hawking's Area theorem were some of the most general results characterising classical black holes. The tension with the other very famous phenomena of evaporation however, made the area theorem appear weaker than Penrose's result, in the sense of ``less general''. It might be due to this inconsistency then, that while Singularity theorems benefited of a nearly constant effort for being generalised to semi-classical frameworks, the area theorem remained confined within classical boundaries, a nice little gem of mathematical relativity that couldn't shine further.

What we have found, instead, is that it is indeed true that Singularity theorems are ``more general'', as they transfer almost untouched to semi-classical settings, but are ``less generalisable''. There is no weaker notion of singularity they could be weakened to, the spacetime is either or not geodesically incomplete. Instead, that very famous zero that bounds the rate of change of the black hole horizon can be weakened to a negative number, and we proved this is indeed what happens, at least in a semi-classical context. 
Moreover, the energy conditions used to prove this theorem only need to be valid on achronal null geodesics, a similarity to the AANEC that looks very promising if hoping that this version of the theorem could generalise even to a complete theory of quantum gravity (if, of course, AANEC itself resists up to there, as it is now believed).

We have also realised that black holes are a very nice playground where to test energy conditions: depending on whether or not they allow black hole evaporation, and to what extent, it is possible to tell how ``quantum'' the chosen condition is. This is of course not possible by considering singularity theorems, as a singularity should form anyway, and we have learnt that simply allowing negative energy densities might not be enough to obtain a properly ``quantum'' condition (recall the example of the dANEC). With some effort, the computation of our bound \(\mathcal{V}\) might then be turned on a measure onto the space of energy conditions.

Similarly to what very commonly happens in this subject, as soon as a path is chosen, many more seem to open up already after a few steps in.
Minimization problems have a very long history in physics, and the techniques we have applied are very widely known, but definitely don't exhaust the all spectrum of valid ones.
Moreover, in the investigation of state dependent Sobolev condition, we performed the application only to one of many possible field theories that fulfil a condition of that sort, even if computing the coefficients might be more difficult in other settings.
If that wasn't enough, an extra degree of freedom is provided by the quantum state we choose to average the stress tensor on. We have used KMS states, but it shouldn't be too difficult to generalise the result to Hartle-Hawking states.
What would really be interesting however, is managing to perform the computation in the Unruh state, the proper one to describe black hole radiation. Any computation of this sort would immensely strengthen the rigour and liability of our results, and we hope to have a chance to investigate it in the near future.

Finally, let us mention one last possible perspective for future investigation. Immediately after the formulation of the black hole area theorem, it was pointed out the suggestive parallel with the second law of thermodynamics. Together with the formulation of the first law of black hole mechanics (another evocative reformulation of Einstein equation) this inspired the formulation of a thermodynamics theory for black holes~\cite[]{bardeen1973four}, which then found its proper solidity in the discovery of Hawking radiation, once more.
Apart from leading to one of the most active field of today's research, developments along this path lead to a generalisation of the identification between entropy of the black hole and area of the horizon, by means of a generalised notion of entropy.
Of course, with the generalised area theorem treated in this thesis, the suggestive parallel from which everything burst out, ceases to hold; we wonder however, if a deeper effect is going to be retrieved in black hole thermodynamics. 
Actually, a connection between entropy, energy conditions and singularity theorems is already being studied by Wall and Bousso; the former proved AANEC (in two dimension) starting precisely from the second law of thermodynamics, while the latter is author of singularity theorems that set off from energy conditions that exploit the notion of generalised entropy. To understand where exactly our generalised area law would find its place within this framework would be most interesting, and might help in gaining a bigger picture of this long standing puzzle.