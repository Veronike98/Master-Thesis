\documentclass[12pt, a4paper]{article}

% !TeX program = lualatex

\usepackage{pacchetti}

%%%%%%%%%%%%%%%%%%%%%%%%%%%%%%%%%%%%%%% Colori 
\definecolor{darkturquoise}{rgb}{0.0, 0.81, 0.82}
\definecolor{cerisepink}{rgb}{0.93, 0.23, 0.51}
\definecolor{brilliantlavender}{rgb}{0.96, 0.73, 1.0}
\definecolor{fuchsiapink}{rgb}{1.0, 0.47, 1.0}

%%%%%%%%%%%%%%%%%%%%%%%%%%%%%%%%%%%%%%% HEAD COMMANDS	
%\newtheorem{theorem}{Theorem}[section]
%
%\newtheorem{corollary}{Corollary}[theorem]
%
%\newtheorem{lemma}[theorem]{Lemma}
%
%\theoremstyle{definition}
%\newtheorem{definition}{Definition}[section]
%
%\theoremstyle{remark}
%\newtheorem*{remark}{Remark}

%%%%%%%%%%%%%%%%%%%%%%%%%%%%%%%%%%%%%%% MATH SYMBOLS
\newcommand{\R}{{\mathbb{R}}}
\DeclareMathOperator{\Tr}{Tr}

%%%%%%%%%%%%%%%%%%%%%%%%%%%%%%%%%%%%%%%%box carini
\newenvironment<>{ideablock}[1]{%
	\setbeamercolor{block title}{fg=white,bg=idea}%
	\begin{block}#2{#1}}{\end{block}}

\newenvironment<>{defblock}[1]{%
	\setbeamercolor{block title}{fg=white,bg=definition}%
	\begin{block}#2{#1}}{\end{block}}

\newenvironment<>{theoblock}[1]{%
	\setbeamercolor{block title}{fg=white,bg=theorem}%
	\begin{block}#2{#1}}{\end{block}}


%%%%%%%%%%%%%%%%%%%%%%%%%%%%%%%%%%%%%%%%% due nuovi comandi per allineare le cose bene
\newcommand\parallelcontent[2]{
	\begin{columns}[t]
		\column{0.5\textwidth} #1
		\column{0.5\textwidth} #2
	\end{columns}
}
\newcommand\parallelitem[2]{
	\parallelcontent
	{\begin{itemize} \item #1 \end{itemize}}
	{\begin{itemize} \item #2 \end{itemize}}
}

%per fare le parentesi sulla destra di una lista

%%%%%%%%%%%%%%%%%%%%%%%%%%%%%%%%%%%%%%%% Graffe verticali per il testo
\tikzset{My Node Style/.style={midway, right, xshift=3.0ex, align=left, font=\small, draw=none, thin, text=black}}

\newcommand{\VerticalBrace}[4][]{%
	% #1 = draw options
	% #2 = top mark
	% #2 = bottom mark
	% #4 = label
	\begin{tikzpicture}[overlay,remember picture]
	\tikzmath{coordinate \p,\q;\p=(pic cs:#2);\q=(pic cs:#3);\maxx=max(\px,\qx);}
	\draw[xshift=1ex,decorate,decoration={brace, amplitude=1.5ex}, #1] 
	([yshift=1.5ex]{{pic cs:#2} -| \maxx pt,0})  -- ([yshift=-.5ex]{{pic cs:#3} -| \maxx pt,0})
	node[My Node Style] {#4};
	\end{tikzpicture}
}


%todo: settare un nuovo font, ora non ci riesco

\title{Cover Letter}

\author{Veronica Sacchi\thanks{veronica.sacchi@sns.it}}


\begin{document}

\maketitle

My name is Veronica Sacchi, and I am currently a Master Physics student at the Scuola Normale Superiore and the University of Pisa. The former is an institution that offers a scholarship to few selected students, renewed on an annual basis, to allow them to develop their interests in several fields of knowledge, within a demanding but very collaborative environment.
Having been an active part of it in the last \(4.5\) years has allowed me to live fully immersed in a particularly stimulating community, and I enjoyed exploring many different topics with some incredibly lively interactions with very bright colleagues. I loved this fruitful cooperation, I found it incredibly exciting and I'd be looking forward to keeping working in a similar environment.

I'd like to study for a research degree in the group led by professor Gorbenko because I have always been fascinated by some fundamental questions about our universe, and I think this is the best way to keep developing this deep interest.
Up to now I have been studying for a Physics degree (first Bachelor and now Master), and I've found in Theoretical Physics, and specifically in General Relativity, a glimpse of the understanding I want to get about our world. 
I also find exciting the elegant and sometimes rather simple explanations Mathematical Physics has provided us with: the ability to single out the ``few" key-factors one needs, to make so many precise and yet powerful deductions, keeps leaving me astonished and I hope I will be able to contribute to the development of this amazing art in the future.

Specifically at the moment I am working on Singularity Theorems for my Master thesis: rather recently a proof by means of index form methods, instead of the Raychaudhuri's equation, has been shown. This step forward is very important because the \emph{pointwise} energy conditions they require were a major limit for the application of these theorems; developing a proof by means of the Raychaudhuri's equation requires such pointwise restrictions in order to infer properties on the solution of the differential equation. However, such a hypothesis is not something QFT is able to provide us with - and indeed they are violated by even rather innocent-looking quantum fields - making of such conditions the Achilles heel of these theorems.

Developing a proof that passes through index forms only requires some sort of averaged energy conditions instead (roughly because, where before we needed to solve a differential equation, now we have to study an \emph{integral} equation), and this seems very promising for the aim of extending the regime of validity of such important theorems, but also to gain an insight on a quantum theory of gravity.

Finally, last year I also worked for about a month on Classical Singularity Theorems and you can find the slides of the talk I have given afterwards at this \href{https://uz.sns.it/~ver22albireo/ext-file/colloquio/slides.pdf}{link}.

Besides, I have been a former contestant in the EGMO (European Girls' Mathematical Olympiads): this is what got me involved into Mathematics at a more formal level, and after that I have always been very active in trying to promote STEM subjects among girls. I believe girls still have a lot of potential that needs to be developed, and I try to encourage many high school students in joining me in a scientific career, by organizing several stages and training specifically targeted to them.

Finally, I believe I have a very passionate personality, I am very curious and I love asking questions and discuss problems with friends and acquaintances. I always try to gain a deep understanding of the concepts I am working with, sometimes at the price of asking a friend to explain an idea again; I need to work on my own sometimes but I try never to be too detouched from what my collegues are doing, and the enthusiasm I contribute to inject into the work can usually spread around, making some lovely memories out of many challenging afternoons.
\end{document}          
