\documentclass[12pt, a4paper]{article}

% !TeX program = lualatex

\usepackage{pacchetti}



\definecolor{turquoise}{RGB}{0, 247, 230}
\definecolor{goldenyellow}{RGB}{255, 218, 66}
\definecolor{fuchsia}{RGB}{255, 0, 172}

%%%%%%%%%%%%%%%%%%%%%%%%%%%%%%%%%%%%%%% HEAD COMMANDS	
\newtheorem{theorem}{Theorem}[section]

\newtheorem{corollary}[theorem]{Corollary}

\newtheorem{lemma}[theorem]{Lemma}

\newtheorem{prop}[theorem]{Proposition}

\theoremstyle{definition}
\newtheorem{definition}{Definition}[section]

\theoremstyle{remark}
\newtheorem*{remark}{Remark}

\newcommand{\EAK}[1]{\textcolor{red}{EAK: #1}}
\newcommand{\VS}[1]{\textcolor{cyan}{VS: #1}}

%%%%%%%%%%%%%%%%%%%%%%%%%%%%%%%%%%%%%%% MATH SYMBOLS
\newcommand{\R}{{\mathbb{R}}}
\newcommand{\N}{{\mathbb{N}}}

 \newcommand{\pprec}{\prec\mathrel{\mkern-5mu}\prec}






%todo: settare un nuovo font, ora non ci riesco

\title{Cover Letter}

\author{Veronica Sacchi\thanks{veronica.sacchi@sns.it}}


\begin{document}

\maketitle

My name is Veronica Sacchi, and I am currently a Master Physics student at the Scuola Normale Superiore and the University of Pisa. The former is an institution that offers a scholarship to few selected students, renewed on an annual basis, to allow them to develop their interests in several fields of knowledge, within a demanding but very collaborative environment.
Having been an active part of it in the last \(4.5\) years has allowed me to live fully immersed in a particularly stimulating community, and I enjoyed exploring many different topics with some incredibly lively interactions with very bright colleagues. I loved this fruitful cooperation, I found it incredibly exciting and I'd be looking forward to keeping working in a similar environment.

I'd like to study for a research degree in the group led by professor Gorbenko because I have always been fascinated by some fundamental questions about our universe, and I think this is the best way to keep developing this deep interest.
Up to now I have been studying for a Physics degree (first Bachelor and now Master), and I've found in Theoretical Physics, and specifically in General Relativity, a glimpse of the understanding I want to get about our world. 
I also find exciting the elegant and sometimes rather simple explanations Mathematical Physics has provided us with: the ability to single out the ``few" key-factors one needs, to make so many precise and yet powerful deductions, keeps leaving me astonished and I hope I will be able to contribute to the development of this amazing art in the future.

Specifically at the moment I am working on Singularity Theorems for my Master thesis: rather recently a proof by means of index form methods, instead of the Raychaudhuri's equation, has been shown. This step forward is very important because the \emph{pointwise} energy conditions they require were a major limit for the application of these theorems; developing a proof by means of the Raychaudhuri's equation requires such pointwise restrictions in order to infer properties on the solution of the differential equation. However, such a hypothesis is not something QFT is able to provide us with - and indeed they are violated by even rather innocent-looking quantum fields - making of such conditions the Achilles heel of these theorems.

Developing a proof that passes through index forms only requires some sort of averaged energy conditions instead (roughly because, where before we needed to solve a differential equation, now we have to study an \emph{integral} equation), and this seems very promising for the aim of extending the regime of validity of such important theorems, but also to gain an insight on a quantum theory of gravity.

Finally, last year I also worked for about a month on Classical Singularity Theorems and you can find the slides of the talk I have given afterwards at this \href{https://uz.sns.it/~ver22albireo/ext-file/colloquio/slides.pdf}{link}.

Besides, I have been a former contestant in the EGMO (European Girls' Mathematical Olympiads): this is what got me involved into Mathematics at a more formal level, and after that I have always been very active in trying to promote STEM subjects among girls. I believe girls still have a lot of potential that needs to be developed, and I try to encourage many high school students in joining me in a scientific career, by organizing several stages and training specifically targeted to them.

Finally, I believe I have a very passionate personality, I am very curious and I love asking questions and discuss problems with friends and acquaintances. I always try to gain a deep understanding of the concepts I am working with, sometimes at the price of asking a friend to explain an idea again; I need to work on my own sometimes but I try never to be too detached from what my collegues are doing, and the enthusiasm I contribute to inject into the work can usually spread around, making some lovely memories out of many challenging afternoons.
\end{document}          
