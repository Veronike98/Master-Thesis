
Now that we know what a submanifold is, an important question we might like to answer is \emph{given a submanifold \(P\subset M\) and a point \(q \notin P\), what is the ``best'' path from \(P\) to \(q\)?} 

Here``best'' means a path that maximizes (or minimizes) some relevant functional of the curves joining \(P\) to \(q\). The most important functional we would like to study is the curve-lenght \(L[\gamma]\): it carries a powerful physical meaning but it turns out to be well-defined and sufficiently differentiable only for families of timelike curves. 

%Let's not worry for a moment about it: for the study of critical points of the lenght functional the relevant objects are focal points.

\section{Definitions and general properties}

In this work we would like to study null, or even causal curves: \(L\) is not smooth though so a better choice for what functional to study is the \emph{energy} or \emph{action}. Given a curve \(\gamma : [0, l] \rightarrow M\) it is defined as
\begin{equation*}
E[\gamma] \coloneqq \frac{1}{2}\int_{0}^{l} g(\gamma'(\lambda), \gamma'(\lambda))d\lambda;
\end{equation*}
this quantity can be proven to vary smoothly for any picewise smooth variation of \(\gamma\), regardless of its causal nature.

%Let \(L[\gamma]\) be the lenght functional of the curve, defined as:
%\begin{equation*}
%	L[\gamma] \coloneqq \int_{a}^{b} \vert\vert \dot{\gamma}(\tau) \vert\vert d\tau;
%\end{equation*}
%where \(\vert\vert \dot{\gamma}(\tau) \vert\vert = \sqrt{g_{\mu\nu}U^{\mu}U^{\nu}}\) and \(\tau\) is the proper time of the curve.
%
%Given a picewise smooth family of timelike curves \(\zeta(t,s) \equiv \gamma_s(t)\) (with constant speed \(\vert U \vert = 1\)) it is only a matter of simple algebra to compute the first and second variation of \(L\) along the family.
%\begin{lemma}
%	\label{lemma:first-L-var}
%	If \(L\) is the lenght functional of the family \(\zeta:[a,b] \times (-\delta, \delta) \rightarrow M\) of timelike picewise smooth curves, then (take \(U\) and \(V\) as defined in \ref{eq:tang-trans-fields})
%	\begin{align*}
%		\frac{dL}{ds}\Big\vert_{s = 0} &= \int_{a}^{b} g_{\mu\nu} U^{\mu}\frac{D V^{\nu}}{D t} dt  \\
%		&=- \int_{a}^{b}  g_{\mu\nu} \frac{D U^{\mu}}{D t} V^{\nu} dt - \sum_{i = 1}^{k} g_{\mu\nu} \Delta U^{\mu}V^{\nu}\Big\vert_{t_i} + g_{\mu\nu} U^{\mu}V^{\nu}\Big\vert^b_a.
%	\end{align*}
%where \(t_i\) with \(i\in \{1, \ldots, k\}\) are the breaks of \(\zeta\).
%\end{lemma}

We are now interested in studying the set of all picewise smooth curves joining \(P\) to \(q\), \(\Omega(P, q)\). By means of a rather simple computation we can work out the formula for the first and second variation of \(E\) on a family of curves \(\gamma_s(\lambda) \coloneqq
\zeta(\lambda, s)\) in \(\Omega(P, q)\) (which adds the restriction that \(\forall s \quad \zeta(0, s) \in M\)).

\begin{lemma}
	Let \(\zeta\) be a picewise smooth variation of the curve \(\gamma: [0, l] \rightarrow M\)  in \(\Omega(P, q)\), and call \(U\) and \(V\) the tangent and the transverse fields, as in section \ref{sec:geodesics}. Then
	\begin{align}
		E'[\gamma_s]\vert_{s = 0} \coloneqq \frac{dE[\gamma_s]}{ds}\Big\vert_{s = 0} &= 
		\int_{0}^{l} U_{\mu}\nabla_UV^{\mu}(\lambda) d\lambda = \\
		&= - \int_{0}^{l}  V_{\mu}\nabla_UU^{\mu}(\lambda) d\lambda - \sum_{i = 1}^{k} \Delta U^{\mu}V_{\mu}\Big\vert_{\lambda_i} + U^{\mu}V_{\mu}\Big\vert^l_0.
		\label{eq:first-E-var}
	\end{align}
	If \(\gamma\) is a geodesic we also have
	\begin{equation}
		\label{eq:second-E-var}
		E''[\gamma_s]\vert_{s = 0} \coloneqq \frac{d^2E[\gamma_s]}{ds^2}\Big\vert_{s = 0} = 
		\int_{0}^{l} \left[(\nabla_UV_{\mu})(\nabla_UV^{\mu}) - R_{\mu\nu\alpha\beta}U^{\mu}V^{\nu}V^{\alpha}U^{\beta}\right] d\lambda + (U_{\mu}\nabla_UV^{\mu})\Big\vert_0^l.
	\end{equation}
	
\end{lemma}

\begin{proof}
	For the first identity we have
	\begin{equation*}
		\frac{dE[\gamma_s]}{ds}\Big\vert_{s = 0}  = \int_{0}^{l} U_{\mu}\nabla_VU^{\mu}(\lambda) d\lambda = \int_{0}^{l} U_{\mu}\nabla_UV^{\mu}(\lambda) d\lambda
	\end{equation*}
Then, integrating by parts we get 
\begin{equation*}
	\frac{dE[\gamma_s]}{ds} = \int_{0}^{l} \nabla_U\left(U_{\mu}(\lambda)V^{\mu}\right)(\lambda) d\lambda - \int_{0}^{l} V_{\mu}\nabla_UU^{\mu}(\lambda) d\lambda
\end{equation*}
 and remembering that the transversial field is always continuous (all the curves are unbroken) we get \eqref{eq:first-E-var}.
 
 For the second variation we can keep on deriving, and remembering the definition of the Riemann tensor:
 \begin{align*}
 	\frac{d^2E[\gamma_s]}{ds^2} &= \frac{d}{ds} \left[\int_{0}^{l} U_{\mu}\nabla_UV^{\mu}(\lambda) d\lambda\right] = \\
 	&= \int_{0}^{l} (\nabla_VU^{\mu})(\nabla_UV_{\mu})(\lambda) + U^{\mu} \nabla_V\nabla_UV_{\mu} (\lambda) d\lambda=\\
 	&= \int_{0}^{l} \left[(\nabla_UV_{\mu})(\nabla_UV^{\mu}) - R_{\mu\nu\alpha\beta}U^{\mu}V^{\nu}U^{\alpha}V^{\beta}\right] d\lambda + U^{\mu} \nabla_V\nabla_UV_{\mu} \Big\vert_0^l +\\
 	&+  \int_{0}^{l} \nabla_UU^{\mu} \nabla_VV_{\mu} (\lambda) d\lambda
 \end{align*}
	the last term being zero when we reduce to \(s = 0\), if \(\gamma\) is a geodesic.

\end{proof}

We are now able to formally prove a rather intuitive property:
\begin{prop}
	The critical points of \(E\), defined as the curves \(\gamma\) such that for any variation \(\gamma_s\) \(E'[\gamma]\Big\vert_{s = 0} = 0\), are exactly the normal geodesics from \(P\) to \(q\).
\end{prop}
\begin{proof}
	If \(\gamma\) is an (unbroken) null geodesic we have \(\nabla_UU = 0\) and there are no breaking points. Moreover for fixed endpoint variations \(V^{\mu}(l) = 0\), while for orthogonal geodesics
	\(U^{\mu}V_{\mu}(0) = 0\), hence the last term of \eqref{eq:first-E-var} null, leaving \(E'[\gamma] = 0\) for any variation.
	
	Conversely, if we suppose that \(E'(s =0) = 0\) for every variation of \(\gamma\) in \(\Omega (P,q)\), it is possible to show that each segment \(\gamma[\lambda_i, \lambda_{i + 1}]\) is a geodesic.
	In fact, take \(\textbf{v}\) any tangent vector in \(T_{\gamma(\lambda_i)}M\)  extended on \(\gamma\) by parallel transport, and \(f\) a bump function with support in \([\lambda_i, \lambda_{i + 1}]\); \(V = f\textbf{v}\) produces a fixed endpoint variation of \(\gamma\) and hence for any \(f\) we have 
	\[
	0 = E'[\gamma_s]\vert_{s = 0} = -\int_{\lambda_i}^{\lambda_{i+1}} g_{\mu\nu}\nabla_UU^{\mu} f v^{\nu} d\lambda
	\]
	which leads to \(\nabla_UU^{\mu} = 0\).
	
	We then need to show that the breaks are trivial: let \(v\) be an arbitrary tangent vector in \(\gamma(\lambda_i)\) extended on \(\gamma[\lambda_{i - 1}, \lambda_{i + 1}]\) and \(f\) is a bump function in \([\lambda_{i - 1}, \lambda_{i + 1}]\); we get for any \(v\):
	\[
	0 = E'[\gamma_s]\vert_{s = 0} = -g_{\mu\nu}v^{\nu}\Delta U^{\mu}(\lambda_i)
	\]
	so \(\Delta U^{\mu}(\lambda_i) = 0\).
	
	Finally we are left to show that \(\gamma\) needs to be orthogonal to \(P\). Take any vector \(v \in T_{\gamma(0)}P\) and extend it to a field \(V\) on \(\gamma\) by parallel transport, inducing a variation of \(\gamma\).
	
	\noindent As \(E'[\gamma] = 0\) for any variation in \(\Omega(P, q)\), and since \(\gamma\) is an unbroken null geodesic, we have
	\[
	0 = E'[\gamma_s] \Big\vert_{s = 0} = U^{\mu}V_{\mu} (0) = U^{\mu}v_{\mu},
	\]
	which concludes.
\end{proof}
