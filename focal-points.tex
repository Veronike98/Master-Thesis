
Now that we know what a submanifold is, an important question we might like to answer is \emph{given a submanifold \(P\subset M\) and a point \(q \notin P\), what is the ``best'' path from \(P\) to \(q\)?} 

Here ``best'' means a path that maximizes (or minimizes) some relevant functional of the curves joining \(P\) to \(q\). The most important functional we would like to study is the curve-lenght \(L[\gamma]\): it carries a powerful physical meaning but it turns out to be well-defined and sufficiently differentiable only for families of timelike curves. 

%Let's not worry for a moment about it: for the study of critical points of the lenght functional the relevant objects are focal points.

\section{Variations of \(E\)}

In this work we would like to study null, or even causal curves: \(L\) is not smooth though so a better choice for what functional to study is the \emph{energy} or \emph{action}. Given a curve \(\gamma : [0, l] \rightarrow M\) it is defined as
\begin{equation*}
E[\gamma] \coloneqq \frac{1}{2}\int_{0}^{l} g(\gamma'(\lambda), \gamma'(\lambda))d\lambda;
\end{equation*}
this quantity can be proven to vary smoothly for any picewise smooth variation of \(\gamma\), regardless of its causal nature.

%Let \(L[\gamma]\) be the lenght functional of the curve, defined as:
%\begin{equation*}
%	L[\gamma] \coloneqq \int_{a}^{b} \vert\vert \dot{\gamma}(\tau) \vert\vert d\tau;
%\end{equation*}
%where \(\vert\vert \dot{\gamma}(\tau) \vert\vert = \sqrt{g_{\mu\nu}U^{\mu}U^{\nu}}\) and \(\tau\) is the proper time of the curve.
%
%Given a picewise smooth family of timelike curves \(\zeta(t,s) \equiv \gamma_s(t)\) (with constant speed \(\vert U \vert = 1\)) it is only a matter of simple algebra to compute the first and second variation of \(L\) along the family.
%\begin{lemma}
%	\label{lemma:first-L-var}
%	If \(L\) is the lenght functional of the family \(\zeta:[a,b] \times (-\delta, \delta) \rightarrow M\) of timelike picewise smooth curves, then (take \(U\) and \(V\) as defined in \ref{eq:tang-trans-fields})
%	\begin{align*}
%		\frac{dL}{ds}\Big\vert_{s = 0} &= \int_{a}^{b} g_{\mu\nu} U^{\mu}\frac{D V^{\nu}}{D t} dt  \\
%		&=- \int_{a}^{b}  g_{\mu\nu} \frac{D U^{\mu}}{D t} V^{\nu} dt - \sum_{i = 1}^{k} g_{\mu\nu} \Delta U^{\mu}V^{\nu}\Big\vert_{t_i} + g_{\mu\nu} U^{\mu}V^{\nu}\Big\vert^b_a.
%	\end{align*}
%where \(t_i\) with \(i\in \{1, \ldots, k\}\) are the breaks of \(\zeta\).
%\end{lemma}

We are now interested in studying the set of all picewise smooth curves joining \(P\) to \(q\), \(\Omega(P, q)\). By means of a rather simple computation we can work out the formula for the first and second variation of \(E\) on a family of curves \(\gamma_s(\lambda) \coloneqq
\zeta(\lambda, s)\) in \(\Omega(P, q)\) (which adds the restriction that \(\forall s \quad \zeta(0, s) \in M\)).

\begin{lemma}
	Let \(\zeta\) be a picewise smooth variation of the curve \(\gamma: [0, l] \rightarrow M\)  in \(\Omega(P, q)\), and call \(U\) and \(V\) the tangent and the transverse fields, as in section \ref{sec:geodesics}. Then
	\begin{align}
		E'[\gamma_s]\vert_{s = 0} \coloneqq \frac{dE[\gamma_s]}{ds}\Big\vert_{s = 0} &= 
		\int_{0}^{l} U_{\mu}\nabla_UV^{\mu}(\lambda) d\lambda = \\
		&= - \int_{0}^{l}  V_{\mu}\nabla_UU^{\mu}(\lambda) d\lambda - \sum_{i = 1}^{k} \Delta U^{\mu}V_{\mu}\Big\vert_{\lambda_i} + U^{\mu}V_{\mu}\Big\vert^l_0.
		\label{eq:first-E-var}
	\end{align}
	If \(\gamma\) is a geodesic we also have
	\begin{equation}
		\label{eq:second-E-var}
		E''[\gamma_s]\vert_{s = 0} \coloneqq \frac{d^2E[\gamma_s]}{ds^2}\Big\vert_{s = 0} = 
		\int_{0}^{l} \left[(\nabla_UV_{\mu})(\nabla_UV^{\mu}) + R_{\mu\nu\alpha\beta}U^{\mu}V^{\nu}V^{\alpha}U^{\beta}\right] d\lambda + (U_{\mu}\nabla_VV^{\mu})\Big\vert_0^l.
	\end{equation}
	
\end{lemma}

\begin{proof}
	For the first identity we have
	\begin{equation*}
		\frac{dE[\gamma_s]}{ds}\Big\vert_{s = 0}  = \int_{0}^{l} U_{\mu}\nabla_VU^{\mu}(\lambda) d\lambda = \int_{0}^{l} U_{\mu}\nabla_UV^{\mu}(\lambda) d\lambda
	\end{equation*}
Then, integrating by parts we get 
\begin{equation*}
	\frac{dE[\gamma_s]}{ds} = \int_{0}^{l} \nabla_U\left(U_{\mu}(\lambda)V^{\mu}\right)(\lambda) d\lambda - \int_{0}^{l} V_{\mu}\nabla_UU^{\mu}(\lambda) d\lambda
\end{equation*}
 and remembering that the transversial field is always continuous (all the curves are unbroken) we get \eqref{eq:first-E-var}.
 
 For the second variation we can keep on deriving, and remembering the definition of the Riemann tensor:
 \begin{align*}
 	\frac{d^2E[\gamma_s]}{ds^2} &= \frac{d}{ds} \left[\int_{0}^{l} U_{\mu}\nabla_UV^{\mu}(\lambda) d\lambda\right] = \\
 	&= \int_{0}^{l} (\nabla_VU^{\mu})(\nabla_UV_{\mu})(\lambda) + U^{\mu} \nabla_V\nabla_UV_{\mu} (\lambda) d\lambda=\\
 	&= \int_{0}^{l} \left[(\nabla_UV_{\mu})(\nabla_UV^{\mu}) + R_{\mu\nu\alpha\beta}U^{\mu}V^{\nu}V^{\alpha}U^{\beta}\right] d\lambda + U^{\mu} \nabla_V\nabla_UV_{\mu} \Big\vert_0^l +\\
 	&+  \int_{0}^{l} \nabla_UU^{\mu} \nabla_VV_{\mu} (\lambda) d\lambda
 \end{align*}
	the last term being zero when we reduce to \(s = 0\), if \(\gamma\) is a geodesic.

\end{proof}

We are now able to formally prove a rather intuitive property:
\begin{prop}
	The critical points of \(E\), defined as the curves \(\gamma\) such that for any variation \(\gamma_s\), \(E'[\gamma_s]\Big\vert_{s = 0} = 0\), are exactly the normal geodesics from \(P\) to \(q\).
\end{prop}
\begin{proof}
	If \(\gamma\) is an (unbroken) null geodesic we have \(\nabla_UU = 0\) and there are no breaking points. Moreover for fixed endpoint variations \(V^{\mu}(l) = 0\), while for orthogonal geodesics
	\(U^{\mu}V_{\mu}(0) = 0\), hence the last term of \eqref{eq:first-E-var} null, leaving \(E'[\gamma] = 0\) for any variation.
	
	Conversely, if we suppose that \(E'(s =0) = 0\) for every variation of \(\gamma\) in \(\Omega (P,q)\), it is possible to show that each segment \(\gamma[\lambda_i, \lambda_{i + 1}]\) is a geodesic.
	In fact, take \(\textbf{v}\) any tangent vector in \(T_{\gamma(\lambda_i)}M\)  extended on \(\gamma\) by parallel transport, and \(f\) a bump function with support in \([\lambda_i, \lambda_{i + 1}]\); \(V = f\textbf{v}\) produces a fixed endpoint variation of \(\gamma\) and hence for any \(f\) we have 
	\[
	0 = E'[\gamma_s]\vert_{s = 0} = -\int_{\lambda_i}^{\lambda_{i+1}} g_{\mu\nu}\nabla_UU^{\mu} f v^{\nu} d\lambda
	\]
	which leads to \(\nabla_UU^{\mu} = 0\).
	
	We then need to show that the breaks are trivial: let \(v\) be an arbitrary tangent vector in \(\gamma(\lambda_i)\) extended on \(\gamma[\lambda_{i - 1}, \lambda_{i + 1}]\) and \(f\) is a bump function in \([\lambda_{i - 1}, \lambda_{i + 1}]\); we get for any \(v\):
	\[
	0 = E'[\gamma_s]\vert_{s = 0} = -g_{\mu\nu}v^{\nu}\Delta U^{\mu}(\lambda_i)
	\]
	so \(\Delta U^{\mu}(\lambda_i) = 0\).
	
	Finally we are left to show that \(\gamma\) needs to be orthogonal to \(P\). Take any vector \(v \in T_{\gamma(0)}P\) and extend it to a field \(V\) on \(\gamma\) by parallel transport, inducing a variation of \(\gamma\).
	
	\noindent As \(E'[\gamma] = 0\) for any variation in \(\Omega(P, q)\), and since \(\gamma\) is an unbroken null geodesic, we have
	\[
	0 = E'[\gamma_s] \Big\vert_{s = 0} = U^{\mu}V_{\mu} (0) = U^{\mu}v_{\mu},
	\]
	which concludes.
\end{proof}

By definition \(E''[\gamma]\) is exactly the Hessian \(\textbf{H}_{\gamma}\): the latter in fact is the unique \(\R\)-bilinear from \(T_{\gamma}\Omega\) such that 
\[
\textbf{H}_{\gamma}(V, V) = \frac{d^2E[\gamma_s]}{ds^2}\Big\vert_{s = 0}
\]
where \(\gamma_s\) is any variation of \(\gamma\) in \(\Omega(P, q)\) with transverial field \(V\).
As we now know that \(\gamma\) needs to be orthogonal to \(P\) we could also write the second variation of \(E\) employing the second fundamental form:
\begin{equation}
	\label{eq:hessian}
	\textbf{H}(V) \equiv\frac{d^2E[\gamma_s]}{ds^2}\Big\vert_{s = 0} = 
	\int_{0}^{l} \left[(\nabla_UV_{\mu})(\nabla_UV^{\mu}) - R_{\mu\nu\alpha\beta}U^{\mu}V^{\nu}V^{\alpha}U^{\beta}\right] d\lambda - (U_{\mu}\mathrm{I\!I}^{\mu})(V, V)\Big\vert_{\gamma(0)}.
\end{equation}

\section{Focal points along null geodesics}
We see then that the ``relevant'' set of curves to consider in \(\Omega(P, q)\) when studying \(E\) is the one of geodesics normal to \(P\); however this is not enough to guarantee that we found a minimum of \(E\) and therefore we want to develop a method to better distinguish elements in this subset of curves. Besides the study of the second derivative, which is what an analytic approach would suggest, we find a deep relationship with a new geometrical object: focal points.

\vskip 4pt

First of all we need to define the analogous of Jacobi fields for variations of curves where one of the endpoints is a submanifold: the \(P\)-Jacobi fields.
\begin{definition}
	Given a submanifold \(P\) in \(M\), a \(P\)-Jacobi field is the variation vector field of a geodesic \(\gamma \perp P\) through normal geodesics.
\end{definition}

%%%% pag.281 O'Neill ma secondo me non serve davvero
%If \(V\) is such a field it is characterized by the fact that 
%\[
%\begin{cases}
%V(0) \text{ is tangent to } P; \\
%\Pi_{P}^{\parallel} \frac{dV}{d\lambda} = \Pi_{P}^{\parallel}
%\end{cases}
%\]

\begin{definition}
	Let \(\gamma\) be a geodesic of \(M\) normal to \(P \subset M\). Then \(\gamma(r)\), where \(r \neq 0\) is a \emph{focal point} of \(P\) along \(\gamma\) provided there is a nonzero \(P\)-Jacobi field \(V\) on \(\gamma\), with \(V(r) = 0\).
\end{definition}

A focal point of a submanifold \(P\) along a normal geodesic \(\gamma\) is an \emph{almost}-meeting point of nearby \(P\)-normal geodesics \underline{of the same causal character} of \(\gamma\). If \(\gamma\) is nonnull this is obvious by continuity, while if \(\gamma\) is null this is assured by the following:
\begin{corollary}
	Let \(\gamma\) be a null geodesic normal to a submanifold \(P\) of a Lorentizian manifold \(M\). A \(P\)-Jacobi field \(V\) on \(\gamma\) is the vector field of a variation of \(\gamma\) through null geodesics through \(P\) if and only if \(V \perp \gamma\).
\end{corollary}

\begin{proof}
	Suppose \(\gamma_s\) is such a variation: then \(U_{\mu}U^{\mu}\equiv 0\) by definition, and then
	\[
	0 = \frac{1}{2}\nabla_V(U_{\mu}U^{\mu}) = \nabla_UV^{\mu}U_{\mu}
	\]
	Hence \(\nabla_UV(0) \perp \gamma\); but \(V(0)\) is tangent to \(P\), then we also have \(V(0) \perp \gamma\), which implies, by \ref{lemma:Jacobi-fields-properties}, that \(V \perp \gamma\).
	
	The converse is also true but a bit more complicated to prove, and a formal proof is provided by O'Neill in \cite{o1983semi} in Corollary 10.40.
\end{proof}

As we have that\(V(0) \perp P\) and \(V(l) = 0\) because we are taking a family of normal geodesics in \(\Omega(P, q)\) we have that \(V \perp \gamma\), so we know that when we study variation of \(E\) on a null geodesic \(\gamma\) we can restrict our attention to normal null geodesics.

We are finally ready to prove the following key result:
\begin{prop}
	Let \(P\) be a spacelike sumbanifold of a Lorentz manifold. If there are no focal points of \(P\) along a normal null geodesic \(\gamma\in\Omega(P,q)\), then \(\textbf{H}_\gamma\) is positive semidefinite on \(T_{\gamma}^{\perp}\Omega = \{V \in T_{\gamma}\Omega \vert V \perp \gamma\}\). Furthermore if \(V \in T_{\gamma}^{\perp}\Omega \) and \(\textbf{H}_\gamma(V, V) = 0\) then \(V\) is tangent to \(\gamma\).
\end{prop}

\begin{proof}
	Let \(e_1, \ldots, e_{n - 1}\) be a basis for the space of perpendicular \(P\)- Jacobi fields on \(\gamma\), Without loss of generality we can suppose \(e_1\) to be tangent to \(\gamma\), and as there are no focal points on \(\gamma\) we can write \(V = \sum_{i = 1}^{n - 1} f_ie_i\).
	Now, call \(A \coloneqq \sum_{i = 1}^{n - 1} f'_ie_i\) and \(B \coloneqq \sum_{i = 1}^{n - 1} f_i\nabla_Ue_i\); then
	\[
	\nabla_UV = A + B
	\]
	so
	\begin{align*}
		\nabla_U(V_{\mu}B^{\mu}) &= (\nabla_UV_{\mu})B^{\mu} + V_{\mu}(\nabla_UB^{\mu}) = \\
		 &= A_{\mu}B^{\mu} + B_{\mu}B^{\mu} + \sum_{i = 1}^{n - 1}f'_iV_{\mu}\nabla_Ue_i^{\mu} + \sum_{i = 1}^{n - 1}f_iV_{\mu}\nabla_U\nabla_Ue_i^{\mu}.
	\end{align*}
	By the Jacobi equation \eqref{eq:Jacobi} we have \(\nabla_U\nabla_Ue_i^{\mu} = R\indices{^{\mu}_{\nu\alpha\beta}}U^{\nu}U^{\alpha}e_i^{\beta}\). Moreover a simple computation leads to \((e_i)_{\mu}\nabla_Ue_j^{\mu} - (e_j)_{\mu}\nabla_Ue_i^{\mu} = 0\) and \(e_i\) is tangent to \(P\) so
	\[
	(\Pi^{\parallel}\nabla_Ue_i)^{\mu}e_j^{\mu} = - \mathrm{I\!I}(e_i, e_j)^{\mu}U_{\mu}
	\]
	and by symmetry of the shape tensor we have \((e_i)_{\mu}e_j^{\mu} = (e_j)_{\mu}e_i^{\mu}\). This idenity implies:
	\[
	\sum_{i = 1}^{n - 1}f'_iV_{\mu}\nabla_Ue_i^{\mu} = \sum_{i, j = 1}^{n - 1}f'_if_j(e_j)_{\mu}\nabla_Ue_i^{\mu} = \sum_{i, j = 1}^{n - 1}f'_if_j(e_i)_{\mu}\nabla_Ue_j^{\mu} = A_{\mu}B^{\mu}
	\]
	So, as \((\nabla_UV^{\mu})(\nabla_UV_{\mu}) = (A^{\mu} + B^{\mu}) (A_{\mu} + B_{\mu})\)
	\[
	\implies \nabla_U(V_{\mu}B^{\mu}) = 2A_{\mu}B^{\mu} + B_{\mu}B^{\mu} - R_{\mu\nu\alpha\beta}V^{\mu}U^{\nu}V^{\alpha}U^{\beta}
	\]
	and we finally have
	\begin{equation}
	\label{eq:V-identity}
		(\nabla_UV^{\mu})(\nabla_UV_{\mu}) + R_{\mu\nu\alpha\beta}U^{\mu}V^{\nu}V^{\alpha}U^{\beta} =  \nabla_U(V_{\mu}B^{\mu})  + A_{\mu}A^{\mu}.
	\end{equation}
	Now, substituting \eqref{eq:V-identity} into \eqref{eq:second-E-var} we get
	\[
	\textbf{H}_{\gamma} = \int_{0}^{l} A_{\mu}A^{\mu}(\lambda) d\lambda + V_{\mu}B^{\mu}\Big\vert_{s = 0} - U_{\mu}\mathrm{I\!I}^{\mu}(V, V)\vert_{\gamma(0)}.
	\]
	The last \(2\) terms cancel with each other because \(V(l) = 0\) and 
	\begin{align*}
		-V_{\mu}B^{\mu}(0) &= - \sum_{i = 1}^{n - 1}f_i V_{\mu}\nabla_Ue_i^{\mu}(0) =\\
		& = - \sum_{i = 1}^{n - 1}f_i V_{\mu}\Pi^{\parallel}\left(\nabla_Ue_i^{\mu}\right)(0) =\\
		& = \sum_{i = 1}^{n - 1}f_i U_{\mu}\mathrm{I\!I}^{\mu}(e_i, V)(0) = U_{\mu}\mathrm{I\!I}^{\mu}(V, V)(0).
	\end{align*}
	At this point we can observe that, as \(A^{\mu}\) is orthogonal to the null vector \(U^{\mu}\) \(A_{\mu}A^{\mu} \le 0\) (\(A^{\mu}\) needs to be spacelike, it's a simple consequence of Cauchy-Schwarts inequality), and also
	\[
	A_{\mu}A^{\mu} = 0 \iff A^{\mu}\parallel U^{\mu}
	\]
\end{proof}



