\section{Geodesics}
Let's start off with some basic concepts derived from differential geometry. We will generally address the spacetime manifold as \(M\), \(g\) its Lorentizan metric and \(\nabla\) the Levi-Civita connection.
The first fundamental definiton we need is:
\begin{definition}
	Let \((M, g)\) be a spacetime and \(I \subseteq \R\) an interval; a smooth curve \(\gamma : I \rightarrow M\) is a \emph{geodesic} if its tangent field is parallel transported along \(\gamma\).\\
\end{definition}    

Calling \(U\indices{^{\mu}} \coloneqq \dot{\gamma}^{\mu}\) the tangent field, this definition is equivalent to the equation:
\[
\nabla_U U^{\mu} \coloneqq \frac{D\tensor{U}{^\mu} }{dt} \equiv 0.
\]


A \emph{curve} is a 1-parameter map; similarly we can define a \emph{family of curves} as a 2-parameter map \(\zeta: I \times J \rightarrow M\), where \(I, J \subseteq \R\) are intervals. This gives us \(2\) tangent fields, which we call respectively \emph{longitudinal} and \emph{transverse}:
\[
U^{\mu} \coloneqq \frac{\partial}{\partial t} \zeta^{\mu}(t,s); \quad \quad \quad 
V^{\mu} \coloneqq \frac{\partial}{\partial s} \zeta^{\mu}(t,s). 
\]

\begin{remark}
	Commutation of partial derivatives immediately gives 
	\[
	[U, V] = 0 \implies \nabla_U V^{\mu} = \nabla_V U^{\mu}
	\]
	and, by definition of the Riemann tensor
	\[
	[\nabla_U, \nabla_V]W^{\mu} = R\indices{^{\mu}_{\nu\alpha\beta}}W^{\nu}U^{\alpha}V^{\beta}
	\]
\end{remark}

By this remark we can compute the second derivative of the transverse field:
\[
\frac{D^2\tensor{V}{^\mu} }{dt^2} = \nabla_U\nabla_V U^{\mu} = \nabla_V\nabla_U U^{\mu} + R\indices{^{\mu}_{\nu\alpha\beta}}U^{\nu}U^{\alpha}V^{\beta}
\]

which his leads us to the second fundamental objects we need to define. 
\begin{definition}
	Given a family of \emph{geodesics} \(\gamma_s(t) \coloneqq \zeta(s,t)\), the transversal field \(V^{\mu} \coloneqq \frac{\partial}{\partial s} \zeta^{\mu}(t,s)\) is said to be a \emph{Jacobi field}.
\end{definition}

This vector field has a very important characterization, which here we'll only state:
\begin{lemma}
\(V^{\mu}\) is a \emph{Jacobi field} if and only if it satisfies the equation
	\[
	\frac{D^2\tensor{V}{^\mu} }{dt^2} = \nabla_U\nabla_V U^{\mu} =  R\indices{^{\mu}_{\nu\alpha\beta}}U^{\nu}U^{\alpha}V^{\beta}
	\].
\end{lemma}







