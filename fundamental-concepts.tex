\section{Geodesics}\label{sec:geodesics}
\EAK{General comment: it might be useful to number your equations so you are able to cite them}

The main source of inspiration for this first chapter is the book by B. O'Neill \cite{o1983semi}. We shall refer mainly to chapters \(10\) and \(14\); when needed, more specific references will be made.

Let's start off with some basic concepts derived from differential geometry. We will generally address the spacetime manifold as \(M\), \(g\) its Lorentizan metric and \(\nabla\) the Levi-Civita covariant derivative.
We will generally assume that a \emph{spacetime} is a smooth, time-oriented Lorentzian manifold; we will use the terms manifold and spacetime intercheangeably, unless otherwise stated.
The first fundamental definiton we need is:
\begin{definition}
	Let \((M, g)\) be a spacetime and \(I \subseteq \R\) an interval; a smooth curve \(\gamma : I \rightarrow M\) is a \emph{geodesic} if its tangent field is parallel transported along \(\gamma\).\\
\end{definition}    

%\EAK{Below you are talking about timelike geodesics, clarify it and mention that the parametrization is proper time}

Restricting to the case of timelike geodesics parametrized by proper time, we may call \(U\indices{^{\mu}} \coloneqq \dot{\gamma}^{\mu}\) the tangent field, and the previous definition is equivalent to the equation:

\begin{equation}
\label{eq:tang-trans-fields}
\nabla_U U^{\mu} \coloneqq \frac{D\tensor{U}{^\mu} }{dt} \equiv 0.
\end{equation}



A \emph{curve} is a \(1\)-parameter map; similarly we can define a \emph{family of curves} as a \(2\)-parameter map \(\zeta: I \times J \rightarrow M\), where \(I, J \subseteq \R\) are intervals. This gives us \(2\) tangent fields, which we call respectively \emph{longitudinal} and \emph{transverse}:
\[
U^{\mu} \coloneqq \frac{\partial}{\partial t} \zeta^{\mu}(t,s); \quad \quad \quad 
V^{\mu} \coloneqq \frac{\partial}{\partial s} \zeta^{\mu}(t,s). 
\]

\begin{definition}
	Given a curve \(\gamma\), a \emph{(smooth) variation of \(\gamma\)} is any (smooth) \(2\)-parameter map \(\zeta(s,t)\) such that 
	\[
	\left. \zeta(s, t) \right\vert_{s = 0} = \gamma(t).
	\]
\end{definition}

%todo: metti un bel disegno di una famigla di curve

%\EAK{Small comment: if the equation is at the end of a sentence we usually have a full stop at the end. }

\begin{remark}
	Commutation of partial derivatives immediately gives 
	\[
	[U, V] = 0 \implies \nabla_U V^{\mu} = \nabla_V U^{\mu}
	\]
	and, by definition of the Riemann tensor
	\[
	[\nabla_U, \nabla_V]W^{\mu} = R\indices{^{\mu}_{\nu\alpha\beta}}W^{\nu}U^{\alpha}V^{\beta}
	\].
\end{remark}

By this remark we can compute the second derivative of the transverse field:
\[
\frac{D^2\tensor{V}{^\mu} }{dt^2} = \nabla_U\nabla_V U^{\mu} = \nabla_V\nabla_U U^{\mu} + R\indices{^{\mu}_{\nu\alpha\beta}}U^{\nu}U^{\alpha}V^{\beta}
\]

which his leads us to the second fundamental object we need to define. 
\begin{definition}
	Given a family of \emph{geodesics} \(\gamma_s(t) \coloneqq \zeta(s,t)\), the transversal field \(V^{\mu} \coloneqq \frac{\partial}{\partial s} \zeta^{\mu}(t,s)\) is said to be a \emph{Jacobi field}.
\end{definition}

This vector field has a very important characterization, which here we'll only state:
\begin{lemma}
\(V^{\mu}\) is a \emph{Jacobi field} if and only if it satisfies the equation
	\begin{equation}
	\label{eq:Jacobi}
		\frac{D^2\tensor{V}{^\mu} }{dt^2} = \nabla_U\nabla_V U^{\mu} =  R\indices{^{\mu}_{\nu\alpha\beta}}U^{\nu}U^{\alpha}V^{\beta}.
	\end{equation}
\end{lemma}

Jacobi fields also have some nice properties we will find helpful in the prosecution of our work;
\begin{lemma}
	\label{lemma:Jacobi-fields-properties}
	Let \(V\) be a Jacobi field on \(\gamma\); then
	\[
	V \perp \gamma \iff \exists a\neq b \quad\vert\quad V\perp \gamma(a),\gamma(b) \iff \exists a \quad\vert\quad Y, Y' \perp \gamma(a).
	\]
\end{lemma}
\begin{proof}
	The proof is nearly straightforward from the characterization of Jacobi field and the symmetries of the Riemann tensor:
	\[
	\frac{d^2}{dt^2} (U^{\mu}V_{\mu}) = R_{\mu\nu\alpha\beta}U^{\mu}U^{\nu}U^{\alpha}V^{\beta} = 0
	\]
	hence \(U^{\mu}V_{\mu} = As + B\), from which the result is immediate.
\end{proof}
%\EAK{The previous subsection seems the relevant one to also define focal points and index forms}

\section{Submanifolds}
\label{sec:submanifolds}

\begin{definition}
	Let \(M\) be a submanifold of the semi-Riemannian manifold \(\bar{M}\), and \(j:M\rightarrow\bar{M}\) the inclusion map. Then \(M\) is a \emph{semi-Riemannian submanifold} if the pullback of the metric tensor \(j^*(g)\) is a metric tensor on \(M\).
\end{definition}

This definiton makes it clear that observers on the submanifold \(M\) agree on the notion of distance as if they were outside the submanifold. Nevertheless, observers within \(M\) see the world differently than observers from the outside: they in fact will inherit \(2\) different connections from their respective notion of metric. 

The comparison between these \(2\) different Levi-Civita connections gives surge to the \emph{second fundamental form}, which provides an infinitesimal description of the shape of \(M\) within \(\bar{M}\).

In order to introduce this important object notice first of all that:
\[
\forall p \in M \quad T_p\bar{M} = \underbrace{T_pM}_{\text{vectors tangent to }M}+ \underbrace{T_p(M)^{\perp}}_{\text{vectors orthogonal to } M}
\]

\noindent and we will generally call \(\Pi_p^{\parallel}\coloneqq d_pj\) and \(\Pi_p^{\perp}\coloneqq \mathbb{1} - d_pj\) the projecive maps on these \(2\) subspaces.
Adopting a similar notation, we can decompose the set of all smooth fields in \(T\bar{M}\) restricted to \(M\) as:
\[
\bar{\mathfrak{X}}(M) = \mathfrak{X}(M) + \mathfrak{X}(M)^{\perp}.
\]

Now, the connection \(\bar{D}\) on \(\bar{M}\) will naturally give raise to a connection on \(M\)
\begin{align*}
\bar{D} : \mathfrak{X}(M) \times \bar{\mathfrak{X}}(M) & \rightarrow \bar{\mathfrak{X}}(M) \\
	 V \times X &\mapsto \bar{D}_V X
\end{align*}

by taking any smooth extension of the fields \(V\) and \(X\) to \(\mathfrak{X}(\bar{M})\), and then restricting again on \(M\). It can be proved that \(\bar{D}_V X\) is a well-defined smooth vector field on \(M\). In particular
\begin{lemma} 
	if \(V, W \in \mathfrak{X}(M)\) and \(D\) is the Levi-Civita connection on \(M\), it holds that
	\[
	D_V W = \Pi^{\parallel}\left(\bar{D}_V W\right).
	\]
\end{lemma}

It's evident then that the Levi-Civita connection on \(M\) is missing something with respect to the induced connection from \(\bar{M}\). This is precisely the object we have been looking for:

%\EAK{Math symbol for second fundamental form: $\mathrm{I\!I}$}
\begin{definition}
	given \(M \subset \bar{M}\) a semi-Riemannian submanifold, the \emph{shape tensor} (or \emph{second fundamental form}) is defined as:
	\begin{align*}
		\mathrm{I\!I} : \mathfrak{X}(M) \times \mathfrak{X}(M) &\longrightarrow \mathfrak{X}(M)^{\perp}\\
							V \times W &\mapsto \Pi^{\perp}\left(\bar{D}_V W\right)
	\end{align*}
	\noindent and in particular is bilinear and symmetric.
\end{definition}

In order to gain a better understanding of what this object encodes, let's think for a moment about the following example. Given a field \(Y \in \mathfrak{X}(M)\) tangent to a curve \(\alpha\) of \(M\), parametrized by \(s\), we indicate:
\[
\dot{Y} \coloneqq \frac{\bar{D}Y}{ds} \quad \quad Y' \coloneqq \frac{DY}{ds}
\]
It is then easy to prove that:
\[
\ddot{\alpha} = \alpha'' + \mathrm{I\!I}(\alpha', \alpha')
\]

and hence, we can think of \(\mathrm{I\!I}\) as the additional external ``force'' needed to keep a point moving on \(M \subset \bar{M}\), a sort of costraining force. An equivalent interpretation is by how much a vector paralleled transported in \(M\) according to \(D\) is actually changing from the point of view of \(\bar{D}\).

%todo: inserisci disegno pag.103 O'Neill

We now prove an identity which will be useful to remember in the future. First of all we need to define the analogous of Jacobi fields for variations of curves where one of the endpoints is a submanifold: the \(M\)-Jacobi fields.
\begin{definition}
	Given a submanifold \(M\) in \(bar{M}\), a \(M\)-Jacobi field is the variation vector field of a geodesic \(\gamma \perp M\) through normal geodesics.
\end{definition}
\begin{lemma}
	\label{lemma:shape-identity}
	Let \(M\) be a submanifold of \(\bar{M}\), \(\gamma\) a curve leaving \(M\) orthogonally in \(p\), with tangent vector field \(U^{\mu}\), and \(e_1, \ldots, e_m\) a basis for the space of perpendicular \(M-\)Jacobi fields on \(\gamma\).
	Then 
	\begin{equation}
		(\Pi^{\parallel}\nabla_Ue_i)_{\mu}e_j^{\mu} = - \mathrm{I\!I}(e_i, e_j)^{\mu}U_{\mu}
	\end{equation}
\end{lemma}

	\begin{proof}
		As \(\gamma\) and its variations are orthogonal to \(M\) \(e_j^{\mu}U_{\mu} \equiv 0\), hence:
		\[
		0 = \nabla_{e_i}\left(e_j^{\mu}U_{\mu}\right) = \left(nabla_{e_i}U_{\mu}\right)e_j^{\mu} + U_{\mu}\left(nabla_{e_i}e_j^{\mu}\right)
		\]
		Again, \(U_{\mu}\) is orthogonal to \(M\), then \(U_{\mu}\left(nabla_{e_i}e_j^{\mu}\right) = U_{\mu}\left(\Pi^{\perp}nabla_{e_i}e_j^{\mu}\right)\), while \(e_i\) is a Jacobi field so \(nabla_{e_i}U^{\mu} = \nabla_Ue_i^{\mu}\). Plugging this back into the previous equation we get exactly
		\[
		(\Pi^{\parallel}\nabla_Ue_i)_{\mu}e_j^{\mu} = (\nabla_Ue_i)_{\mu}e_j^{\mu} = (nabla_{e_i}U)_{\mu} e_j^{\mu} = - U_{\mu}\left(\Pi^{\perp}nabla_{e_i}e_j^{\mu}\right) = - \mathrm{I\!I}(e_i, e_j)^{\mu}U_{\mu}
		\]
	\end{proof}


The second fundamental form induces a vector field on \(M\) called \emph{mean curvature}.
\EAK{Maybe use $H^\mu$ for mean normal curvature? It is simpler and it goes with the other papers}
\VS{It used to be like that, but then I realized it might be confused with the event horizon in the black hole area theorem and I wanted to make them different: is that a terrible idea?}
\begin{definition}
		Let\(M \subset \bar{M}\) be an  \(n\)-semi-Riemannian submanifold, with \emph{shape tensor} \(\mathrm{I\!I}\), and \(\{\textbf{e}_i\}_{i \in \{1, \ldots, n\}}\) any frame on \(M\) at \(p\). Then we call \emph{mean curvature vector field} the field \(\mathfrak{H} \in \mathfrak{X}(M)^{\perp} \) defined by:
		\[
		\mathfrak{H}^{\mu}(p) = \frac{1}{n} \sum_{i=1}^{n} \epsilon_i \mathrm{I\!I}^{\mu}(\textbf{e}_i, \textbf{e}_i).
		\]
		where 
		\[
		\epsilon_i = 
		\begin{cases}
		+1 \quad \text{if } \textbf{e}_i \text{ is \emph{timelike}} \\
		-1 \quad \text{if } \textbf{e}_i \text{ is \emph{spacelike}}
		\end{cases}
		\]
\end{definition}

This definition will be particularly important for the proof of Null Singularity Theorems, as they are instrumental to define the concept of a \emph{trapped surface}, the initial condition needed to eventually develop a singularity (in the classical limit). 

\section{Trapped surfaces}

\EAK{This is the O'Neill definition but maybe we should opt for something simpler. Senovilla definition instead? Also not a fan of the parenthesis notation for inner product. Generally you should think what notation you prefer for your whole thesis and stick to that for consistency}

\begin{definition}
	We say that a spacelike submanifold \(M\) is \emph{future-converging} provided its mean curvature vetor field \(\mathfrak{H}\) is past-pointing timelike.
\end{definition}



It is only a matter of simple linear algebra in each normal space \(T_p(M)^{\perp}\) to show that
\begin{lemma} \label{lemma:charact-trapped}
	assuming \(M\) is a \emph{spacelike} submanifold, the follwoing statements are equivalent:
	\begin{enumerate}
		\item  \(k(v) =g_{\mu\nu}\mathfrak{H}^{\mu} v^{\nu} < 0 \) for all future pointing \emph{null} vectors \(v\) normal to M.
		\item  \(k(w) =g_{\mu\nu}\mathfrak{H}^{\mu} w^{\nu} < 0 \) for all future pointing \emph{causal} vectors \(w\) normal to M.
		\item \(\mathfrak{H}\) is past-pointing timelike.
	\end{enumerate}
\end{lemma}

\begin{proof}
	We will prove a chain of implications to gain the equivalence.
	
	\(\mathbf{2) \implies 1)]}\) this is trivial because any null vector is a causal vector.
	
	\(\mathbf{1) \implies 3)]}\) Let's write \(\mathfrak{H}^{\mu} = (\mathfrak{H}^0, \vec{\mathfrak{H}})\). It is possible to choose a spacelike \(3-\)vector \(\vec{v}\) such that \(\vert\vec{v}\cdot\vec{\mathfrak{H}}\vert = - \vert\vec{v}\vert\cdot\vert\vec{\mathfrak{H}}\vert\) and  then complete it to the future-pointing null vector \(v^{\mu} = (v^0, \vec{v})\) (where \(v^0 = \vert \vec{v}\vert > 0\)).
	Now:
	\[
	\mathfrak{H}^{\mu}v_{\mu} = \mathfrak{H}^0v^0 - \vec{\mathfrak{H}}\cdot\vec{v} < 0 \quad \implies 
	\quad v^0 \mathfrak{H}^0 < - \vert\vec{v}\vert\cdot\vert\vec{\mathfrak{H}}\vert < 0
	\]
	which, for the positivity of \(v^0\), implies that \(H^{\mu}\) is timelike and past-pointing.
	
	\(\mathbf{3) \implies 2)]}\) first of all observe that any causal vector can be written as the sum of \(2\) future-pointing vectors such that \(w^{\mu}= v^{\mu} + t^{\mu}\) where \(v\) is null, and \(t^{\mu} = (t^0, \vec{0})\) (with \(t^0 \ge 0\)). At this point:
	\[
	w^{\mu}\mathfrak{H}^{\mu} = v^{\mu}\mathfrak{H}^{\mu} + \underbrace{t^0 \mathfrak{H}^0}_{\le0}
	\]
	Moreover, by Cauchy-Schwartz we have:
	\[
	v_{\mu}\mathfrak{H}^{\mu} = v^0 \mathfrak{H}^0 - \vec{\mathfrak{H}}\cdot\vec{v} \le v^0\mathfrak{H}^0 +\vert \vec{\mathfrak{H}}\vert\cdot\underbrace{\vert\vec{v}\vert}_{v^0} = v^0\underbrace{(\mathfrak{H}^0 + \vert \vec{\mathfrak{H}}\vert)}_{<0} < 0.
	\]
\end{proof}

	%todo: write prooof of variation of area, see Kriele.

\section{Causality conditions}

The notion of \emph{causality} has to do with the fact that in a Lorentzian manifold, starting from a  point \(p\) it is not true that an observer or a light ray can travel to any other point \(q\), as they are allowed only to move along (future pointing) causal geodesics (i.e. whose tangent field is never spacelike).


\EAK{Why $\guillemotleft$ and not $\ll$? Similarly why $\prec$ and not $<$?}
We will denote causality relations with the following notation, following \cite{o1983semi}.
\begin{enumerate}
	\item  \(p \guillemotleft q\) means there is a future-pointing \emph{timelike} curve in \(M\) connecting \(p\) to \(q\).
	\item \(p \prec q\) means there is a future-pointing \emph{causal} curve from \(p\) to \(q\).
\end{enumerate}

Evidently \(p\guillemotleft q\) implies \(p\prec q\) but not vice versa. For a subset \(A \subseteq M\) we also define:
\begin{enumerate}
	\item the \emph{chronological future} of \(A\)
	\[
	I^+(A) =\{ q\in M :\quad\exists p \in A\text{ with } p\guillemotleft q\}.
	\]
	\item the \emph{causal future} of \(A\)
	\[
	I^+(A) =\{ q\in M : \quad\exists p \in A\text{ with } p\prec q\}.
	\]
\end{enumerate}
\EAK{Might want to add the domain of dependence, Cauchy horizon, spatial/timelike/null infinities we discussed last time (also look at Wald). A conformal diagram might be nice too (something to think about later perhaps).}