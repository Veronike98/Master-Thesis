
%%%%%%%%%%%%%%%%%%%%%%%%%%%%%%%%%%%%%%% COLORI

\definecolor{turquoise}{RGB}{0, 247, 230}
\definecolor{goldenyellow}{RGB}{255, 218, 66}
\definecolor{fuchsia}{RGB}{255, 0, 172}
\definecolor{electric-blue}{RGB}{55, 0, 255}

\definecolor{blond}{RGB}{251,231,161}
\definecolor{skincolor}{RGB}{224,177,132}
\definecolor{idea}{RGB}{254,231,2} 	        %giallo ideee

%%%%%%%%%%%%%%%%%%%%%%%%%%%%%%%%%%%%%%% HEAD COMMANDS	
\newcommand{\inghead}{% 
	\textcolor{darkturquoise}{\LARGE\textbf{Ingredienti}\ }
}

\newcommand{\mathead}{% 
	\textcolor{darkturquoise}{\LARGE\textbf{Materiale utile}\ }
}
\newcommand{\prephead}{% 
	\textcolor{fuchsiapink}{\LARGE\textbf{Preparazione}\ } }
\newcommand{\hinthead}{% 
	\textcolor{cerisepink}{\huge{Tip:}}
}

%%%%%%%%%%%%%%%%%%%%%%%%%%%%%%%%%%%%%% Math commands

% Comandi per i teoremi, definizioni ed esmpi
\theoremstyle{plain}
\newtheorem{thm}{Teorema} % reset theorem numbering for each chapter
\newtheorem{post}{Postulato} % reset theorem numbering for each chapter
\newtheorem{lem}{Lemma}

\theoremstyle{definition}
\newtheorem{defn}{Definizione} % definition numbers are dependent on theorem numbers
\newtheorem{exmp}{Esempio} % same for example numbers
\newtheorem*{ach}{Attenzione!}

% Shortcut
\newcommand{\SigmaX}{\hat{\sigma}^{(x)}}
\newcommand{\SigmaY}{\hat{\sigma}^{(y)}}
\newcommand{\SigmaZ}{\hat{\sigma}^{(z)}}

\newcommand{\Hilb}{\mathcal{H}}
\newcommand{\Id}{\hat{\mathcal{I}}}
\newcommand{\LinSet}{\mathscr{L}}
\newcommand{\ChoiState}[1]{\hat{\rho}_\text{CJ}^{#1}}

\DeclareMathOperator{\Tr}{tr}

\newcommand{\VerticalBrace}[4][]{%
	% #1 = draw options
	% #2 = top mark
	% #2 = bottom mark
	% #4 = label
	\begin{tikzpicture}[overlay,remember picture]
	\tikzmath{coordinate \p,\q;\p=(pic cs:#2);\q=(pic cs:#3);\maxx=max(\px,\qx);}
	\draw[xshift=1ex,decorate,decoration={brace, amplitude=1.5ex}, #1] 
	([yshift=1.5ex]{{pic cs:#2} -| \maxx pt,0})  -- ([yshift=-.5ex]{{pic cs:#3} -| \maxx pt,0})
	node[My Node Style] {#4};
	\end{tikzpicture}
}


%box carini
\newenvironment<>{ideablock}[1]{%
	\setbeamercolor{block title}{fg=white,bg=fuchsia}%
	\begin{block}#2{#1}}{\end{block}}

\newenvironment<>{defblock}[1]{%
	\setbeamercolor{block title}{fg=white,bg=goldenyellow}%
	\begin{block}#2{#1}}{\end{block}}

\newenvironment<>{theoblock}[1]{%
	\setbeamercolor{block title}{fg=white,bg=turquoise!80!black}%
	\begin{block}#2{#1}}{\end{block}}


% due nuovi comandi per allineare le cose bene
\newcommand\parallelcontent[2]{
	\begin{columns}[t]
		\column{0.5\textwidth} #1
		\column{0.5\textwidth} #2
	\end{columns}
}
\newcommand\parallelitem[2]{
	\parallelcontent
	{\begin{itemize} \item #1 \end{itemize}}
	{\begin{itemize} \item #2 \end{itemize}}
}

%Citazioni
\newcommand{\easycite}[2]{
	[\textcolor{electric-blue}{#1 - #2}]
}

%%%%%%%%%%%%%%%%%%%%%%%%%%%%%%%%%%%%%%% MATH SYMBOLS
\newcommand{\R}{{\mathbb{R}}}
\newcommand{\N}{{\mathbb{N}}}

 \newcommand{\pprec}{\prec\mathrel{\mkern-5mu}\prec}