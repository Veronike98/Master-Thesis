In \cite[]{visser1997gravitational} Visser sets up a semi-analytical model in order to compute some contractions of the stress energy tensor of scalar fields non-minimally coupled to a Schwarzschild background metric, averaged on the Unruh Vacuum.

We are again using the Unruh vacuum - as in \cite[]{levi2016versatile} - which is the appropriate state to describe black hole evaporation, but we expand the domain of validity of such a result to non minimally coupled scalar fields, which is the case we have been treating in the analysis of \(mathcal{V}\).

Moreover, the advantage of a semi-analytical computation allows us to understand the robustness of our result, which is what we are eventually interested in.

For the \((3 + 1)\) Unruh vacuum we don't have any analytic approximate expression of the stress energy tensor, and hence the need of a numerical approach. These results are obtained  using the Jensen-McLaughlin-Ottewill numerical data for the spin \(0\) Unruh vacuuum \cite[]{jensen1991renormalized}. By spherical symmetry, for any \(s\)-wave quantum state \(\vert \psi\rangle\) 
\begin{equation}
    \langle\psi\vert T^{\mu\nu}\vert\psi\rangle = \begin{pmatrix}
        \rho & f & 0 & 0 \\
        f & -\tau & 0 & 0 \\
        0 & 0 & p & 0 \\
        0 & 0 & 0 & p
    \end{pmatrix}
\end{equation}

Here \(\rho\), \(f\), \(p\) and  \(\tau\) are functions of the radial coordinate \(r\), the mass of the black hole \(M\) and \(\hbar\), and we have set \(G = 1\). This expression is valid in the local-Lorentz basis attached to the fiducial static observer (FIDOS); to be more precise, if the Schwarzschild metric is written as 
\[
ds^2 =  \left(1 - \frac{R_s}{r}\right)dt^2  - \frac{dr^2}{1 - \frac{R_s}{r}} - r^2d\Omega^2
\]

then the local Lorentz basis attached to the fiducial observer at coordinate \(t, r\) will be written as 
\[
d\tilde{t} = \sqrt{1 - \frac{R_s}{r}} dt \quad \quad d\tilde{r} = \frac{dr}{\sqrt{1 - \frac{R_s}{r}}}, 
\]

so that the metric locally diagonalizes to the flat one.
For compactness of notation we shall define \(z = \frac{R_s}{r}\), where \(R_s = 2M\) is the Schwarzschild radius.

Visser relies on the decomposition of the stress energy tensor into \(4\) separately conserved quantities (in the Schwarzschild spacetime), as was shown in \cite[]{christensen1977trace}:

\[
    \langle\psi\vert T^{\mu\nu}\vert\psi\rangle = \left[T_{trace}\right]^{\mu\nu} + \left[T_{pressure}\right]^{\mu\nu} +
    \left[T_{+}\right]^{\mu\nu} + \left[T_{-}\right]^{\mu\nu}.
\]

These quantities are defined to be:
\[
    \left[T_{trace}\right]^{\mu\nu} = \begin{pmatrix}
        -T(z) + \frac{z^2}{1 - z}H(z) & 0 & 0 & 0 \\
        0 & + \frac{z^2}{1 - z}H(z) & 0 & 0 \\
        0 & 0 & 0 & 0 \\
        0 & 0 & 0 & 0
    \end{pmatrix} 
\]
where 
\[
    H(z)\equiv \frac{1}{2}\int_{z}^{1} \frac{T(\bar{z})}{\bar{z}^2}d\bar{z}    
\]
and 
\[
    \left[T_{pressure}\right]^{\mu\nu} = 
    \begin{pmatrix}
        2p(z) + \frac{z^2}{1 - z}G(z) & 0 & 0 & 0 \\
        0 & + \frac{z^2}{1 - z}G(z) & 0 & 0 \\
        0 & 0 & p(z) & 0 \\
        0 & 0 & 0 & p(z)
    \end{pmatrix} 
\]
with 
\[
    \quad \quad G(z)\equiv \int_{z}^{1} \left(\frac{2}{\bar{z}^3} - \frac{3}{\bar{z}^2}\right)p(\bar{z})d\bar{z}.    
\]
The conserved tensor \(\left[T_{trace}\right]^{\mu\nu}\) depends only on the trace \(T(z)\) of the total stress energy tensor, while \(\left[T_{trace}\right]^{\mu\nu}\) is traceless and only depends on the transverse pressure \(p(z)\). In the Hartle-Hawking and Unruh vacuum states \(T\vert_{z = 1}\) and \(p\vert_{z = 1}\) are finite \cite[]{christensen1977trace,jensen1991renormalized}, so \(G\) and \(H\) not only are well defined, but near the horizion can be expanded as 
\begin{align*}
    H(z) &= \frac{1}{2}T\vert_{z = 1}(1 - z) + O[(1 - z)^2],\\
    G(z) &= -p\vert_{z = 1}(1 - z) + O[(1 - z)^2].\\
\end{align*}

Visser chooses to rearrange \(G(z)\) into \(F(z)\) through an integration by parts
\[
G(z) = - \frac{1 - z}{z^2}p(z) - F(z) \quad \quad F(z) \equiv \int_z^1 \bar{z}^2(1 - \bar{z})\frac{d}{d\bar{z}}\left(\frac{p(\bar{z})}{\bar{z}^4}\right) d\bar{z}.   
\]

Finally the outgoing and ingoing flux parts are defined as 
\[
    \left[T_{\pm}\right]^{\mu\nu} = f_{\pm} \frac{z^2}{1 - z}
    \begin{pmatrix}
        1 & \pm 1 & 0 & 0 \\
        \pm & 1 & 0 & 0 \\
        0 & 0 & 0 & 0 \\
        0 & 0 & 0 & 0
    \end{pmatrix},
\]
where \(f_+\) and \(f_-\) are \(2\) \emph{constants} that the determine the overall flux 
\[
f(z) = \left(f_+ - f_-\right) \frac{z^2}{1 - z}. 
\]

Up to now everything has been carried out with minimal assumptions on the background geometry and the matter content of the system. Now assume that we are in presence of a \emph{conformally coupled} field; then the trace of the stress tensor is known exactly and in Schwarzschild takes the form:
\[
T(z) = \Upsilon p_{\infty}z^6 \quad \quad p_{\infty} \equiv \frac{\hbar}{90(16\pi)^2(2M)^4}   
\]
The number \(\Upsilon\) depends on the quantum field under consideration and for a conformally coupled scalar field it is \(\Upsilon = 96\). Instead \(p_{\infty}\) can be interpreted as the pressure at spatial infinity, and its dependence on the mass \(M\) is the crucial feature that will give us our desired power law.

Moreover, we will set our analysis in the Unruh vacuum, where there is a lot of additional information available; first of all regularity on the future horizion implies \(f_+ = 0\), though it is allowed \(f_- \neq 0\).

Notice that nothing stops \(f_-\) from being negative (which corresponds to an outgoing flux, as we expect for the Hawking radiation), so it makes sense to define \(f_- = -f_0\). 

Now, at asymptotic spatial infinity we want the stress tensor to look like that of an outgoing flux of positive radiation (the Hawking radiation in fact, see \cite[]{christensen1977trace}). This corresponds to asking for
\[
\rho(z) \rightarrow f(z) \quad\quad \text{ as } z \rightarrow 0;    
\]
substituiting the expressions for \(\rho\) and \(f\) in terms of \(f_0\) and \(F(0)\), and matching the leading terms, we find:

\[
f_0 = \frac{\Upsilon}{20}p_{\infty} - \frac{F(0)}{2}.
\]

Notice then that the all physical details of the physic near the horizion is encoded in \(F(0)\). It is precisely for the determination of \(F\) that the numerical approach shall be used. However, a rough idea can be already given by matching the total luminosity of a black hole
\[
    L = 4\pi f_0(2M)^2
\]
with the Stefan-Boltzmann's law of geometric optics:
\[
L_{\text{geometric optics}} = \frac{1}{2}\sigma S T^4.  
\]
In \(L_{\text{geometric optics}}\) there is a factor \(\frac{1}{2}\) due to the single polarization of the scalar field, \(\sigma = \frac{\pi^2}{60}\), \(T\) is taken to be the Hawking temperature, and \(S\) is the effective radiating surface area \(S = 4\pi(3\sqrt{3}M)^2\). Pulling this together Visser obtains
\[
 f_{\text{geometric optics}} = \frac{81}{16}p_{\infty}.
\]

More precise computation of that proportionality coeffiecient might be obtained by more careful analysis and with the help of numerical simulations - as done in \cite[]{visser1997gravitational} for example - but this won't change the mass dependence of the final result, as that is fully contained in \(p_{\infty}\), and therefore we shall not be concerned by that.

We have then all the information on the stress energy tensor that we need, and can therefore proceed to compute the contraction we are intereted in, namely \(\langle T^{\mu\nu}\rangle U_{\mu}U_{\nu}\), where \(U^{\mu}\) are the tangent vectors of the generators of the event horizion.

The coordinate system used up to now is not well suited anymore for this type of computation: it is good to analyze the behaviour of observer close to spatial infinity, but near the horizion it makes appear the well-known coordinate singluarity, and it is not clear how to well define \(U^{\mu}\). Inspired by the classical treatment of the coordinate singularity, we proceed to careful change coordinate system, and express \(T^{\mu\nu}\) in terms of Kruskal coordinates.

To make the notation more compact, and the computation easier, we should disregard the angular components, as we are only going to be interested in radial geodesics.

\section{Local Lorentzian coordinates}
As we have seen, in the coordinates \(\tilde{t}, \tilde{r}\) the stress tensor takes the form:

\[
    \langle\psi\vert T^{\mu\nu}\vert\psi\rangle = 
    \begin{pmatrix}
        \rho & f \\
        f & -\tau
    \end{pmatrix} ,
\]
where
\[
\tilde{t} = \sqrt{1 - \frac{R_s}{r}} t + \tilde{t}_0 \quad \quad d\tilde{r} = \frac{dr}{\sqrt{1 - \frac{R_s}{r}}}.
\]
The Jacobian then looks like:
\[
\tilde{J} = \begin{pmatrix}
    \frac{\partial \tilde{t}}{\partial t} = \sqrt{1 - z} & \frac{\partial \tilde{t}}{\partial r} = \frac{t}{2R_s} \frac{z^2}{\sqrt{1 - z}} \\
    \frac{\partial \tilde{r}}{\partial t} = 0 & \frac{\partial \tilde{r}}{\partial r} = \frac{1}{\sqrt[]{1 - z}}
\end{pmatrix},   
\]
and the inverse reads
\[
\tilde{J}^{- 1} = \begin{pmatrix}
    \frac{\partial t}{\partial \tilde{t}} = \frac{1}{\sqrt{1 - z}} & \frac{\partial t}{\partial \tilde{r}} = -\frac{t}{2R_s} \frac{z^2}{\sqrt{1 - z}} \\
    \frac{\partial r}{\partial \tilde{t}} = 0 & \frac{\partial r}{\partial \tilde{r}} = \sqrt{1 - z}
\end{pmatrix}.   
\]

Changing to the usual coordinates \((t, r)\) is now easy, and brings to (we drop the angle parethesis for compactness of notation)
\[
   T^{\mu\nu} = \begin{pmatrix}
    \frac{1}{1 - z}\left[\rho - f z^2 \frac{t}{2R_s} - \tau z^4\frac{t^2}{4R_s^2}\right] & f + \tau \frac{t}{2R_s}z^2 \\
    f + \tau \frac{t}{2R_s}z^2  & -\tau(1 - z)
   \end{pmatrix}.
\]

\section{Kruskal coordinates}
The Kruskal coordinates \((U, V)\) are defined as 
\[
   U = - e^{-\frac{u}{2R_s}} \quad \quad V = e^{\frac{v}{2R_s}}
\]
where \((u,v)\) are the so called Eddington-Finkelstein coordinates
\[
u = t - r - R_s\ln\left(\frac{r - R_s}{R_s}\right)    \quad \quad 
v = t + r + R_s\ln\left(\frac{r - R_s}{R_s}\right).
\]

The Jacobian is then
\[
J = \begin{pmatrix}
    \frac{\partial U}{\partial t} = - \frac{U}{2R_s} & \frac{\partial U}{\partial r} = \frac{1}{1 - z} \frac{U}{2R_s} \\
    \frac{\partial V}{\partial t} = \frac{V}{2R_s} & \frac{\partial V}{\partial r} = \frac{1}{1 - z} \frac{V}{2R_s}
\end{pmatrix}    
\]

so that the metric becomes
\[
    ds^2 = \frac{32M^3e^{-\frac{r}{R_s}}}{r}dUdV
\]
and the stress tensor
\[
    \scriptsize
    T^{\mu\nu} = \frac{1}{4R_s^2}\frac{1}{1 - z} 
    \begin{pmatrix}
        U^2 \left[\rho - 2f - \tau - z^2 \frac{t}{2R_s} \left(f + \tau\right) - z^4\frac{t^2}{4R_s^2}\tau \right] &
        UV\left[\tau - \rho + fz^2 \frac{t}{R_s} + \tau z^4\frac{t^2}{4R_s^2}\right] \\ 
        UV\left[\tau - \rho + fz^2 \frac{t}{R_s} + \tau z^4\frac{t^2}{4R_s^2}\right] &
        V^2 \left[\rho + 2f - \tau - z^2 \frac{t}{2R_s} \left(f - \tau\right) - z^4\frac{t^2}{4R_s^2}\tau \right]
    \end{pmatrix}.
\]

\begin{figure}
    \centering
    \includegraphics[scale=1.17]{example-image-duck}
    \caption{Penrose diagram of the maximally extended Schwarzschild solution}
\end{figure}

The outgoing null geodesics, generating the future horizions are given by \((U \equiv 0, V)\), and the tangent vector is 
\[
U^{\mu} = \frac{2rR_s}{16M^3e^{-\frac{r}{R_s}}} \begin{pmatrix}
    0 \\ 1
\end{pmatrix},
\]
so
\[
   T^{\mu\nu}g_{\mu\alpha}g_{\nu\beta}U^{\alpha}U^{\beta}  = 
    \frac{U^2}{1 - z}\left[\rho - 2f - \tau - z^2 \frac{t}{2R_s} \left(f + \tau\right) - z^4\frac{t^2}{4R_s^2}\tau \right].
\]
We need to be careful in taking the limit towards the horizion to compute correctly \(\frac{U^2}{1 - z}\). We choose to look at the family of null geodesics given by \(U \equiv U_0\) constant, and then taking \(U_0 \rightarrow 0\). 

Recall the identity
\[
UV = - \frac{r}{R_s} (1 - z) e^(\frac{r}{R_s}) \implies U^2 = \frac{(1 - z)^2}{z^2}  \frac{e^z}{V^2}
\]
and indeed
\begin{equation}
    \label{eq:contraction-nearly-done}
    \langle T^{\mu\nu}\rangle U_{\mu}U_{\nu} = \frac{e}{V^2}(1 - z)\left[\rho - 2f - \tau - z^2 \frac{t}{2R_s} \left(f + \tau\right) - z^4\frac{t^2}{4R_s^2}\tau \right]. 
\end{equation}
Finally, we can substitute the expressions for \(\rho\), \(f\) and \(\tau\); according to Visser:
\begin{align*}
    \rho(z) &= -f_0 \frac{z^2}{1 - z} - F(z) \frac{z^2}{1 - z} + \text{reg} \\
    \tau(z) &= f_0 \frac{z^2}{1 - z} + F(z) \frac{z^2}{1 - z} + \text{reg} \\
    f(z) &= f_0 \frac{z^2}{1 - z}
\end{align*}
where we didn't bother writing terms regular on the horizion, as they get suppressed by the extra factor \((1 - z)\) in \(\eqref{eq:contraction-nearly-done}\). By recalling the definition of \(F(z)\) we can observe that near the horizion \(F(z) = O[(1 - z)^2]\), and hence \(F(z)\) also provides regular terms that will be suppressed for \(z \simeq 1\).

In the end we find
\[
    \langle T^{\mu\nu}\rangle U_{\mu}U_{\nu} =  -\frac{e}{V^2}f_0 \left[4 + \frac{t}{R_s} + \frac{t^2}{4R_s^2}\right]
\]
The contraction becomes singular on the past horizion, at \(V = 0\), as expected, but it remains regular for anywhere else on the future horizion, it is always negative (so we have everywhere a violation of the null energy condition), and its value goes as \(M^{-4}\). 

Notice that these two last features only derive from the very minimal requirements of positive luminosity, and the matching with the expected well-known behaviour of the stress tensor in flat background (at spatial infinity). It is exactly the minimality of these requirements that provides such robustness to our result.

